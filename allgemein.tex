\newglossaryentry{initiativephase}
{
    name=Initiativephase,
    description={In deiner Initiativephase kannst du handeln. Eine Minute entspricht 16 Initiativephasen einer Person.}
}

\newglossaryentry{schwierigkeit}
{
	name={Schwierigkeit},
	description={Wenn dein \gls{erfolgswert} größer gleich der Schwierigkeit ist, gelingt die Probe.}
}

\newglossaryentry{attributsproben}
{
    name={Attributsproben},
    description={Der \gls{probenwert} beträgt das doppelte des \glslink{attribut}{Attributs}.}
}

\newglossaryentry{fertigkeitsproben}
{
    name={Fertigkeitsproben},
    description={Der \gls{Probenwert} beträgt die Summe aus \gls{basiswert} und \gls{fertigkeitswert}.}
}
\newglossaryentry{basiswert}
{
    name={Basiswert},
    description={}}

\newglossaryentry{erschwernis}
{
    name={Erschwernis},
    description={Verringert den \gls{erfolgswert} (bspw. -2).}}
    
\newglossaryentry{erleichterung}
{
    name={Erleichterung},
    description={Erhöht den \gls{erfolgswert} (bspw. +2).}}
\newglossaryentry{vier stufen}
{
    name={Vier Stufen},
    description={\glslink{erschwernis}{Erschwernisse} bzw. \glslink{erleichterung}{Erleichterungen} betragen immer 2/4/8/16.}}
\newglossaryentry{freiwillig erschweren}
{
    name={freiwillig erschweren},
    description={Das freiwillige Erschweren gibt dir verschiedene Boni (siehe z.B. \gls{mächtige magie}).}}
\newglossaryentry{hohe qualität}
{
    name={Hohe Qualität},
    description={Mit Hohe Qualität kannst du schärfere Schwerter schmieden, tödlichere Gifte brauen oder deine Gegner stärker einschüchtern. Jede Verbesserung erschwert deine Probe um –4. Die genauen Auswirkungen von Hohe Qualität findest du bei der jeweiligen Handlung.}}
\newglossaryentry{glück}
{
    name={Glück},
    description={Proben, in denen Glück eine große Rolle spielt, werden mit \gls{1w20} statt mit \gls{3w20} abgelegt.}}
\newglossaryentry{umgebung}
{
    name={Umgebung},
    description={Proben, in denen die Umgebung eine große Rolle spielt, werden mit \gls{1w20} statt mit \gls{3w20} abgelegt.}}
\newglossaryentry{triumph}
{
    name={Triumph},
    description={Wenn deine Probe gelingt und der gewertete Würfel eine 20 zeigt, hast du einen Triumph erzielt.Und deinem Charakter gelingt ein außergewöhnlicher, ja herausragender, Erfolg.}}
\newglossaryentry{patzer}
{
    name={Patzer},
    description={Wenn deine Probe misslingt und der Würfel eine 1 zeigt, bedeutet das einen Patzer – dein Charakter hat sich in eine unangenehme oder sogar lebensbedrohliche Situation gebracht.}}
\newglossaryentry{zusammenarbeit}
{
    name={Zusammenarbeit},
    description={Bei manchen Aufgaben ist es sinnvoll, wenn mehrere Charaktere zusammenarbeiten. Der Spielleiter entscheidet,
wie viele Charaktere helfend eingreifen können und wie groß ihr \gls{einfluss} ist. Helfer können eine um 4 Punkte leichtere Probe ablegen, um die entscheidende Probe um +2 (kleiner Einfluss), +4 (großer Einfluss) oder selten sogar +8 (enormer Einfluss) pro Helfer zu erleichtern. Misslingt der Helferin die Probe, ist die entscheidende Probe aber um –2 erschwert.}}
\newglossaryentry{einfluss}
{
    name={Einfluss},
    description={Einfluss kann klein (+2), groß (+4) oder selten sogar enorm (+8) sein. Einfluss ist besonders für \gls{zusammenarbeit} relevant.}}
\newglossaryentry{gruppenprobe}
{
    name={Gruppenprobe},
    description={Dabei würfelt ein Spieler eine Probe, deren Ergebnis für die ganze Gruppe bindend ist.}}

\newglossaryentry{offene probe}
{
    name={Offene Probe},
    description={Dabei würfelt der Spieler gegen eine vom Spielleiter offen bestimmte \gls{schwierigkeit}.}}

\newglossaryentry{vergleichende probe}
{
    name={Vergleichende Probe},
    description={Die Probe mit dem höheren \gls{ew} setzt sich durch. Dabei kann \glslink{aktive probe}{aktiv} oder \glslink{passive probe}{passiv} gewürfelt werden.}}
        
\newglossaryentry{ausgedehnte probe}
{
    name={Ausgedehnte Probe},
    description={In einer ausgedehnten Probe bestimmt der Spielleiter den Detailgrad (DG). Um erfolgreich zu sein, müssen dir insgesamt so viele Einzelproben gelingen, wie der Detailgrad beträgt – und zwar, bevor dir dieselbe Anzahl an Einzelproben
misslungen ist. (Eine gewöhnliche Probe entspricht also einem Detailgrad von 1)}
}

\newglossaryentry{detailgrad}
{
    name={Detailgrad},
    description={Bestimmt wie viele Proben bei einer ausgedehnten Probe gewürfelt werden. Normalerweise liegt der Detailgrad bei 2 bis 4.}}

\newglossaryentry{eigenheit}
{
    name={Eigenheit},
    description={Eigenheiten beschreiben die Stärken und Schwächen deines Charakters, seine Herkunft, Ausbildung und Weltanschauung. Außerdem lassen sie sich mit Schicksalspunkten \glslink{eigenheiten einsetzen}{einsetzen} oder \gslink{eigenheiten ausnutzen}{ausnutzen}, um Schicksalspunkte zu generieren.}}

\newglossaryentry{status}
{
    name={status},
    description={Der Status bestimmt, ob dein Charakter in seinem bisherigen Leben Wachteleier gespeist hat oder gerade der menschen­verachtenden Sklavenarbeit in einer Mine entkommen ist: Elite, Oberschicht, Mittelschicht,Unterschicht und Abschaum. Außerdem bestimmt er deine \gls{Lebenserhaltungskosten}.}}

\newglossaryentry{Erfahrungspunkte}
{
    name={Erfahrungspunkte},
    description={Erfahrungspunkte (EP) sind die Währung, mit der du die Werte deines Charakters weiterentwickelst, sodass dein Charakter immer größere Herausforderungen bestehen kann.}}

\newglossaryentry{attributsteigerungskosten}
{
    name={Attributsteigerungskosten},
    description={Attribute werden Punkt für Punkt gesteigert. Jeder Punkt
kostet Zielwert x 16 Erfahrungspunkte.}}

\newglossaryentry{fertigkeitswertsteigerungskosten}
{
    name={Fertigkeitswertsteigerungskosten},
    description={Fertigkeits­werte Punkt für Punkt gesteigert. Jeder Punkt kostet Zielwert x Steigerungsfaktor Erfahrungspunkte. Der Steigerungsfaktor hängt von der Fertigkeit ab und liegt zwischen 2 und 4.}}

\newglossaryentry{talent}
{
    name={Talent},
    description={Jeder Fertigkeit sind mehrere Talente zugeordnet, die für einen Teilbereich dieser Fertigkeit stehen. Ohne das passende Talent darfst du zwar Proben ablegen, dabei aber nur den halben Fertigkeitswert einsetzen.}}

\newglossaryentry{talentkosten}
{
    name={Talentkosten},
    description={Ein Talent kostet dich 20 x Steigerungsfaktor Erfahrung­punkte. Selten anwendbare Talente sind verbilligt und kosten nur die Hälfte. Da sich jeder Charakter mit den Gebräuchen seiner eigenen Kultur auskennt erhält er das entsprechende Talent gratis.}}

\newglossaryentry{freie fertigkeit}
{
    name={Freie Fertigkeit},
    description={Freie Fertigkeiten sind verschiedene seltene Handwerks­künste, aber auch Sprachen und Schriften.
Freie Fertigkeiten sind in drei Stufen geteilt. Mit der ersten Stufe bist du unerfahren in dieser Freien Fertigkeit, mit der zweiten erfahren und mit der dritten ein Meister dieses Faches. Du kannst diese Stufen I/II/III für nur 4/8/16 EP erwerben, wobei die vorhergehende Stufe vorausgesetzt wird. Selbstverständlich spricht jeder Charakter seine eigene Muttersprache.}}

\newglossaryentry{energie}
{
    name={Energie},
    description={Über \glslink{vorteil}{Vorteile} werden übernatürliche Energien zur Verfügung gestellt. Beispielsweise müssen Zauberer den \gls{vorteil} \gls{zauberer} kaufen, der ihnen \glslink{astralpunkt}{Astralpunkte} (\gls{asp}) verleiht.}}

\newglossaryentry{traditionsvorteil}
{
    name={Traditionsvorteil},
    description={}}

\newglossaryentry{übernatürliche fertigkeit}
{
    name={Übernatürliche Fertigkeit},
    description={Kategorie des übernatürlichen Wirkens (bspw. Einfluss, Antimagie).}}

\newglossaryentry{übernatürliches talent}
{
    name={Übernatürliches Talent},
    description={Sammelbegriff für Liturgien, Zauber und Anrufungen (bspw. Böser Blick, Herr über das Tierreich).}}

\newglossaryentry{energiesteigerungskosten}
{
    name={Energiesteigerungskosten},
    description={Jeder Punkt kostet Zielwert \acronym{ep}.}}

\newglossaryentry{schicksalspunkt}
{
    name={Schicksalspunkt},
    description={Schicksalspunkte bieten verschiedene Boni: \gls{glückliche fügung}, \gls{nur ein kratzer} und \gls{von der schippe springen}. Außerdem wird ihr Einsatz zusammen mit passenden \glslink{eigenheit}{Eigenheiten} noch wirkungsvoller.}}

\newglossaryentry{glückliche fügung}
{
    name={Glückliche Fügung},
    description={Du kannst einen Schicksalspunkt einsetzen, der dir einen zusätzlichen Würfel für die nächste Probe verleiht. Statt mit drei Würfeln würfelst du mit vier und der zweithöchste Wert zählt. Statt mit einem Würfel würfelst du mit zwei und der höchste Wert zählt.}}


\newglossaryentry{nur ein kratzer}
{
    name={Nur ein Kratzer},
    description={Unmittelbar nachdem du Wunden erlitten hast, kannst du einen Schicksalspunkt einsetzen. Dadurch werden die gerade erlittenen Wunden halbiert.}}


\newglossaryentry{von der schippe springen}
{
    name={Von der Schippe springen},
    description={Du kannst zwei Schicksalspunkte aufwenden, um den Tod deines Charakters abzuwenden. Dazu musst du deinem Spielleiter einen plausiblen Vorschlag machen, wie dein Charakter die tödliche Situation überlebt. Von der Schippe springen lässt deinen Charakter übrigens nicht ungeschoren davonkommen. Eher überlebt er einen eigentlich tödlichen Sturz schwer verletzt oder wird von den Feinden für tot gehalten und ausgeplündert.}}
                

\newglossaryentry{startschicksalspunkte}
{
    name={Startschicksalspunkte},
    description={Du startest mit sehr sehr 2(sehr reich)/3(reich)/4(normal)/5(arm)/6(sehr arm) Schicksalspunkten. Du kannst Schicksalspunkte über dem Maximum (normalerweise 4) nicht zurückgewinnen. Dies wirkt sich auch auf dein Startkapital aus.}}
        

\newglossaryentry{startkapital}
{
    name={Startkapital},
    description={Du startest mit 256(sehr reich)/128(reich)/32(normal)/16(arm)/4(sehr arm) Dukaten. Dies wirkt sich auch auf deine Startschicksalspunkte aus.}}

\newglossaryentry{eigenheiten einsetzen}
{
    name={Eigenheiten einsetzen},
    description={Du kannst \glslink{eigenheit}{Eigenheiten} einsetzen, um deinem Charakter einen \gls{vorteil} zu verschaffen. Passt eine deiner \glslink{eigenheit}{Eigenheiten} zu deinem Vorhaben, kannst du Schicksalspunkte effektiver nutzen: Erstens erhältst du bei einer glücklichen Fügung sogar zwei zusätzliche Würfel. Zweitens kannst du für einen Schicksalspunkt einen Probenwurf wiederholen. Der Spielleiter hat bei all diesen Varianten das Recht auf ein Veto.}}

\newglossaryentry{eigenheiten ausnutzen}
{
    name={Eigenheiten ausnutzen},
    description={\glslink{eigenheit}{Eigenheiten} stellen außerdem die Schwächen und Marotten deines Charakters dar und können ihm das Leben schwer machen. Als Gegenleistung erhältst du einen \gls{schicksalspunkt}, außer du verfügst schon über die maximale Zahl von \glslink{schicksalspunkt}{Schicksalspunkten}. Dieses Ausnutzen von \glslink{eigenheit}{Eigenheiten} kann auf drei Arten geschehen: \gls{vom spielleiter ausgehend}, \gls{von dir ausgehend}, und \gls{aus dem spiel heraus}}}

\newglossaryentry{vom spielleiter ausgehend}
{
    name={Vom Spielleiter ausgehend},
    description={Der Spielleiter bietet dir an, eine deiner \glslink{eigenheit}{Eigenheiten} auszunutzen. Wenn du das zulässt, kann er eine Handlung deines Charakters scheitern lassen oder ihn auf eine andere Art und Weise in Schwierigkeiten bringen. Dafür bekommst du einen \gls{Schicksalspunkt}. Du kannst das Ausnutzen einer \gls{Eigenheit} auch ablehnen, wenn du dafür einen deiner \glslink{schicksalspunkt}{Schicksalspunkte} abgibst. Besitzt du keinen \gls{schicksalspunkt}, kannst du nicht ablehnen.}}

\newglossaryentry{von dir ausgehend}
{
    name={Von dir ausgehend},
    description={Du schlägst dem Spielleiter eine Handlung vor, die deiner Meinung nach einen \gls{schicksalspunkt} wert ist und zu deinen \glslink{eigenheit}{Eigenheiten} passt. Nimmt der Spielleiter an, erhältst du einen Schicksalspunkt. Hier gibt es kein Risiko, Eigeninitiative wird also belohnt!}}

\newglossaryentry{aus dem spiel heraus}
{
    name={Aus dem Spiel heraus},
    description={Manchmal bringst du dich durch das Ausspielen deiner \glslink{eigenheiten}{Eigenheiten} ganz von selbst in haarige Situationen, ohne dass irgendjemand an \glslink{schicksalspunkt}{Schicksalspunkte} gedacht hätte. In diesem Fall sollte dir der Spielleiter nachträglich einen \gls{schicksalspunkt} zugestehen.}}

\newglossaryentry{einsatz erhöhen}
{
    name={Einsatz erhöhen},
    description={Wenn du in einer wirklich entscheidenden Situation das \glslink{eigenheiten ausnutzen}{Ausnutzen} ablehnst, kann der Spielleiter auch den Einsatz erhöhen. Statt einem \gls{Schicksalspunkt} erhältst du nun zwei, wenn du das Ausnutzen der \gls{eigenschaft} zulässt.}}

\newglossaryentry{lebenserhaltungskosten}
{
    name={Lebenserhaltungskosten},
    description={Diese geben den ungefähren Betrag an, den dein Charakter pro Monat für Unterkunft, Verpflegung, Kleidung, Unterhaltung und alle übrigen Ausgaben des täglichen Lebens ausgibt, wenn er nicht auf Selbstversorgung zurückgreift oder auf fremde Kosten lebt. Ein höherer \gls{status} bedeutet höhere Lebenserhaltungskosten. Der Vorteil \gls{einkommen} kann dabei helfen auch hohe Lebenserhalungskosten zu bestreiten.}}

\newglossaryentry{attribut}
{
    name={Attribut},
    description={Die insgesamt acht Attribute (\gls{ch},\gls{ff},\gls{ge},\gls{in},\gls{kk},\gls{kl},\gls{ko},\gls{mu}) stellen die geistige und körperliche Grundlage deines Charakters dar. Manchmal legst du \glslink{attributprobe}{Attributproben} ab. Darüber hinaus bestimmen sie deine \glslink{abgeleiteter Wert}{abgeleiteten Werte} (s.u.), beeinflussen den \gls{basiswert} für \glslink{fertigkeitsprobe}{Fertigkeitsproben} und werden für viele \glslink{vorteil}{Vorteile} vorausgesetzt.}}

\newglossaryentry{charisma}
{
    name={Charisma},
    description={ist die natürliche Ausstrahlung deines Charakters auf seine Umgebung. Es steht für Führungsqualitäten, Selbstbewusstsein, Überzeugungskraft und ein gewinnendes Wesen. Alle gesellschaftlichen Fertigkeiten hängen von deinem Charisma ab und auch Elementaristen benötigen ein hohes Charisma.}}

\newglossaryentry{fingerfertigkeit}
{
    name={Fingerfertigkeit},
    description={ist die manuelle Geschicklichkeit deines Charakters. Er verfügt über eine gute Hand-Augen-Koordination und kann feine Bewegungen schnell und fehlerfrei ausführen. Handwerker, Fernkämpfer und Hersteller von Artefakten sollten deswegen nicht auf eine hohe Fingerfertigkeit verzichten.}}

\newglossaryentry{gewandtheit}
{
    name={Gewandtheit},
    description={ist die Beweglichkeit und Gelenkigkeit deines Charakters. Gewandte Charaktere bewegen sich geschmeidig und können ihre Bewegungen gut abschätzen. Das fördert deine körperlichen und kämpferischen Fertigkeiten und erhöht deine \gls{Geschwindigkeit}.}}

\newglossaryentry{intuition}
{
    name={Intuition},
    description={ermöglicht es deinem Charakter, Personen und Sachverhalte schnell richtig einzuschätzen. Sie steht auch für Wahrnehmung und Einfühlungsvermögen deines Charakters. Intuition beeinflusst eine breite Palette an Fertigkeiten und bestimmt deine \gls{Initiative} im Kampf.}}

\newglossaryentry{körperkraft}
{
    name={Körperkraft},
    description={ist ein Maß für die Stärke deines Charakters. Kräftige Charaktere können schwere Lasten heben und ihre Angriffe richten durch den höheren \gls{Schadensbonus} mehr Schaden an. Dadurch ist eine hohe Körperkraft gerade für kämpferische Charaktere hilfreich.}}

\newglossaryentry{klugheit}
{
    name={Klugheit},
    description={Klugheit (KL) ist das logische Denkvermögen deines Charakters und seine Fähigkeit, komplizierte Zusammenhänge zu erkennen und zu analysieren. Kluge Charaktere verfügen auch über ein gutes Allgemeinwissen. Eine hohe Klugheit ist bei vielen gesellschaftlichen und allen Wissensfertigkeiten unverzichtbar.}}

\newglossaryentry{konstitution}
{
    name={Konstitution},
    description={beschreibt die Widerstandsfähigkeit deines Charakters gegen äußere Einflüsse wie Strapazen, Gifte, Krankheiten und Verwundungen. Die Konstitution beeinflusst nur wenige Fertigkeiten, bestimmt aber deine \gls{Wundschwelle}, den wohl wichtigsten \glslink{abgeleiteter Wert}{abgeleiteten Wert}.}}

\newglossaryentry{mut}
{
    name={Mut},
    description={Mut (MU): Ein mutiger Charakter bewahrt in kritischen Situationen einen kühlen Kopf und schreckt nicht vor Gefahren zurück, was gerade im Nahkampf unerlässlich ist. Zusätzlich stärkt Mut den Widerstand gegen magische Beeinflussungen, indem er deine \gls{Magieresistenz} erhöht. Auch Dämonologen benötigen einen hohen Mut.}}

\newglossaryentry{abgeleiteter wert}
{
    name={Abgeleiteter Wert},
    description={Die abgeleiteten Werte berechnen sich aus deinen \glslink{attribut}{Attributen}.
Du kannst sie nicht direkt steigern, sondern musst entweder die \glslink{attribut}{Attribute} erhöhen oder spezielle \glslink{vorteil}{Vorteile} erwerben. Die abgeleiteten Werte sind \glslink{wundschwelle}, \glslink{magieresistenz}, \glslink{geschwindigkeit}, \glslink{initiative} und \glslink{schadensbonus}}}

\newglossaryentry{wundschwelle}
{
    name={Wundschwelle},
    description={bestimmt, wie gut dein Charakter Schaden widerstehen kann. Schaden bis zu deiner Wundschwelle ist zwar schmerzhaft, aber noch nicht wirklich gefährlich. Erst Schadensmengen über deiner Wundschwelle können deinen Charakter beeinträchtigen oder sogar töten. Die Wundschwelle beträgt 4 + 1 für je 4 volle Punkte \gls{ko}. Mit dem Vorteil \gls{unverwüstlich} kannst du die \gls{WS} um +1 erhöhen.}}

\newglossaryentry{magieresistenz}
{
    name={Magieresistenz},
    description={ist die Widerstandsfähigkeit gegen Zauberei. Viele Zauber wirken nur auf deinen Charakter, wenn sie seine Magieresistenz in einer vergleichenden Probe überwinden. Die Magieresistenz beträgt 4 + 1 für je 4 volle Punkte \gls{mu}. Die Vorteile \gls{willensstark I} und \glslink{willensstark II}{II} und \gls{unbeugsam} steigern die \gls{mr} weiter.}}

\newglossaryentry{geschwindigkeit}
{
    name={Geschwindigkeit},
    description={stellt die Schnelligkeit und Beweglichkeit deines Charakters dar. Sie bestimmt, wie weit er sich im Kampf bewegen kann und hilft auch bei Verfolgungsjagden zu Fuß. Die Geschwindigkeit beträgt 4 + 1 für je 4 volle Punkte \gls{ge}. Eine noch höhere \gls{gs} verleihen ihm \gls{flink I} und \glslink{flink II}{II}.}}

\newglossaryentry{initiative}
{
    name={Initiative},
    description={steht für Reaktionsgeschwindigkeit und Übersicht im Kampf. Kämpfer mit hoher Initiative können zu Beginn eines Kampfes schneller handeln und so den Erstschlag führen. Die Initiative entspricht der \gls{in}. Der \gls{vorteil} \gls{kampfreflexe} erhöht die \gls{ini} zusätzlich.}}

\newglossaryentry{schadensbonus}
{
    name={Schadensbonus},
    description={erhöht den Waffenschaden bei allen Nahkampfangriffen. Kopflastige Waffen profitieren sogar doppelt vom Schadensbonus. Für je 4 volle Punkte \gls{kk} erhältst du +1 Schadensbonus.}}

\newglossaryentry{vorteil}
{
    name={Vorteil},
    description={Vorteile sind besondere Fähigkeiten, die deinem Charakter im Spiel nützen können. Sie können entweder schon bei der Generierung oder bei späteren Steigerungen gekauft werden, solange du die Voraussetzungen erfüllst (oft ein \gls{attribut} in einer bestimmten Höhe). Deswegen ist es auch kein Problem,}}
        
        
        
        
        
    
        
        
\newacronym{pw}{PW}{Probenwert}
\newacronym{ew}{EW}{Erfolgswert}
\newacronym{3w20}{3w20}{3 20-seitige Würfel}
\newacronym{1w20}{1w20}{1 20-seitiger Würfel}
\newacronym{I}{I}{1w20 Probe}
\newacronym{dg}{DG}{Detailgrad}
\newacronym{ep}{EP}{Erfahrungspunkte}
\newacronym{ch}{CH}{Charisma}
\newacronym{ff}{FF}{Fingerfertigkeit}
\newacronym{ge}{GE}{Gewandtheit}
\newacronym{in}{IN}{Intuition}
\newacronym{kk}{KK}{Körperkraft}
\newacronym{kl}{KL}{Klugheit}
\newacronym{ko}{KO}{Konstitution}
\newacronym{mu}{MU}{Mut}
\newacronym{ws}{WS}{Wundschwelle}
\newacronym{mr}{MR}{Magieresistenz}
\newacronym{gs}{GS}{Geschwindigkeit}
\newacronym{ini}{INI}{Initiative}
\newacronym{asp}{AsP}{Astralpunkte}

%\newglossaryentry{#1}
%{
%    name=#1,
%    description={#2}
%}