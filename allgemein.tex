\newglossaryentry{initiativephase}
{
    name=Initiativephase,
    description={In deiner Initiativephase kannst du handeln. Eine Minute entspricht 16 Initiativephasen einer Person.}
}

\newglossaryentry{schwierigkeit}
{
	name={Schwierigkeit},
	description={Wenn dein Erfolgswert größer gleich der Schwierigkeit ist, gelingt die Probe.}
}

\newglossaryentry{attributsproben}
{
    name={Attributsproben},
    description={Der Probenwert beträgt das doppelte des Attributs.}
}

\newglossaryentry{fertigkeitsproben}
{
    name={Fertigkeitsproben},
    description={Der Probenwert beträgt die Summe aus Basiswert und Fertigkeitswert.}
}
\newglossaryentry{basiswert}
{
    name={Basiswert},
    description={}}

\newglossaryentry{erschwernis}
{
    name={Erschwernis},
    description={Verringert den Erfolgswert (bspw. -2).}}
    
\newglossaryentry{erleichterung}
{
    name={Erleichterung},
    description={Erhöht den Erfolgswert (bspw. +2).}}
\newglossaryentry{vier stufen}
{
    name={Vier Stufen},
    description={Erschwernisse bzw. Erleichterungen betragen immer 2/4/8/16.}}
\newglossaryentry{freiwillig erschweren}
{
    name={freiwillig erschweren},
    description={Das freiwillige Erschweren gibt dir verschiedene Boni (siehe z.B. Mächtige Magie).}}
\newglossaryentry{hohe qualität}
{
    name={Hohe Qualität},
    description={Mit Hohe Qualität kannst du schärfere Schwerter schmieden, tödlichere Gifte brauen oder deine Gegner stärker einschüchtern. Jede Verbesserung erschwert deine Probe um –4. Die genauen Auswirkungen von}}
\newglossaryentry{glück}
{
    name={Glück},
    description={Proben, in denen Glück eine große Rolle spielt, werden mit 1w20 statt mit 3w20 abgelegt.}}
\newglossaryentry{umgebung}
{
    name={Umgebung},
    description={Proben, in denen die Umgebung eine große Rolle spielt, werden mit 1w20 statt mit 3w20 abgelegt.}}
\newglossaryentry{triumph}
{
    name={Triumph},
    description={Wenn deine Probe gelingt und der gewertete Würfel eine 20 zeigt, hast du einen Triumph erzielt.Und deinem Charakter gelingt ein außergewöhnlicher, ja herausragender, Erfolg.}}
\newglossaryentry{patzer}
{
    name={Patzer},
    description={Wenn deine Probe misslingt und der Würfel eine 1 zeigt, bedeutet das einen Patzer – dein Charakter hat sich in eine unangenehme oder sogar lebensbedrohliche Situation gebracht.}}
\newglossaryentry{zusammenarbeit}
{
    name={Zusammenarbeit},
    description={Bei manchen Aufgaben ist es sinnvoll, wenn mehrere Charaktere zusammenarbeiten. Der Spielleiter entscheidet,
wie viele Charaktere helfend eingreifen können und wie groß ihr Einfluss ist. Helfer können eine um 4 Punkte leichtere Probe ablegen, um die entscheidende Probe um +2 (kleiner Einfluss), +4 (großer Einfluss) oder selten sogar +8 (enormer Einfluss) pro Helfer zu erleichtern. Misslingt der Helferin die Probe, ist die entscheidende Probe aber um –2 erschwert.}}
\newglossaryentry{einfluss}
{
    name={Einfluss},
    description={Einfluss kann klein (+2), groß (+4) oder selten sogar enorm (+8) sein.}}
\newglossaryentry{gruppenprobe}
{
    name={Gruppenprobe},
    description={Dabei würfelt ein Spieler eine Probe, deren Ergebnis für die ganze Gruppe bindend ist.}}

\newglossaryentry{offene probe}
{
    name={Offene Probe},
    description={Dabei würfelt der Spieler gegen eine vom Spielleiter offen bestimmte Schwierigkeit.}}

\newglossaryentry{vergleichende Probe}
{
    name={Vergleichende Probe},
    description={Die Probe mit dem höheren EW setzt sich durch. Dabei kann aktiv oder passiv gewürfelt werden.}}
        
\newglossaryentry{ausgedehnte Probe}
{
    name={Ausgedehnte Probe},
    description={In einer ausgedehnten Probe bestimmt der Spielleiter den Detailgrad (DG). Um erfolgreich zu sein, müssen dir insgesamt so viele Einzelproben gelingen, wie der Detailgrad beträgt – und zwar, bevor dir dieselbe Anzahl an Einzelproben
misslungen ist. (Eine gewöhnliche Probe entspricht also einem Detailgrad von 1)}
}

\newglossaryentry{detailgrad}
{
    name={Detailgrad},
    description={Bestimmt wie viele Proben bei einer ausgedehnten Probe gewürfelt werden. Normalerweise liegt der Detailgrad bei 2 bis 4.}}

\newglossaryentry{eigenheit}
{
    name={Eigenheit},
    description={Schritt 1: Eigenheiten wählen Eigenheiten beschreiben die Stärken und Schwächen deines Charakters, seine Herkunft, Ausbildung und Weltanschauung.}}

\newglossaryentry{status}
{
    name={status},
    description={Der Status bestimmt, ob dein Charakter in seinem bisherigen Leben Wachteleier gespeist hat oder gerade der menschen­verachtenden Sklavenarbeit in einer Mine entkommen ist: Elite, Oberschicht, Mittelschicht,Unterschicht und Abschaum.}}

\newglossaryentry{Erfahrungspunkte}
{
    name={Erfahrungspunkte},
    description={Erfahrungspunkte (EP) sind die Währung, mit der du die Werte deines Charakters weiterentwickelst, sodass dein Charakter immer größere Herausforderungen bestehen kann.}}

\newglossaryentry{attributsteigerungskosten}
{
    name={Attributsteigerungskosten},
    description={Attribute werden Punkt für Punkt gesteigert. Jeder Punkt
kostet Zielwert x 16 Erfahrungspunkte.}}

\newglossaryentry{fertigkeitswertsteigerungskosten}
{
    name={Fertigkeitswertsteigerungskosten},
    description={Fertigkeits­werte Punkt für Punkt gesteigert. Jeder Punkt kostet Zielwert x Steigerungsfaktor Erfahrungspunkte. Der Steigerungsfaktor hängt von der Fertigkeit ab und liegt zwischen 2 und 4.}}

\newglossaryentry{talent}
{
    name={Talent},
    description={Jeder Fertigkeit sind mehrere Talente zugeordnet, die für einen Teilbereich dieser Fertigkeit stehen. Ohne das passende Talent darfst du zwar Proben ablegen, dabei aber nur den halben Fertigkeitswert einsetzen.}}

\newglossaryentry{talentkosten}
{
    name={Talentkosten},
    description={Ein Talent kostet dich 20 x Steigerungsfaktor Erfahrung­punkte. Selten anwendbare Talente sind verbilligt und kosten nur die Hälfte. Da sich jeder Charakter mit den Gebräuchen seiner eigenen Kultur auskennt erhält er das entsprechende Talent gratis.}}

\newglossaryentry{freie fertigkeit}
{
    name={Freie Fertigkeit},
    description={Freie Fertigkeiten sind verschiedene seltene Handwerks­künste, aber auch Sprachen und Schriften.
Freie Fertigkeiten sind in drei Stufen geteilt. Mit der ersten Stufe bist du unerfahren in dieser Freien Fertigkeit, mit der zweiten erfahren und mit der dritten ein Meister dieses Faches. Du kannst diese Stufen I/II/III für nur 4/8/16 EP erwerben, wobei die vorhergehende Stufe vorausgesetzt wird. Selbstverständlich spricht jeder Charakter seine eigene Muttersprache.}}

\newglossaryentry{energie}
{
    name={Energie},
    description={Über Vorteile werden übernatürliche Energien zur Verfügung gestellt. Beispielsweise müssen Zauberer den Vorteil Zauberer kaufen, der ihnen Astralpunkte (AsP) verleiht.}}

\newglossaryentry{traditionsvorteil}
{
    name={Traditionsvorteil},
    description={}}

\newglossaryentry{übernatürliche fertigkeit}
{
    name={Übernatürliche Fertigkeit},
    description={Kategorie des übernatürlichen Wirkens (bspw. Einfluss, Antimagie).}}

\newglossaryentry{übernatürliches talent}
{
    name={Übernatürliches Talent},
    description={Sammelbegriff für Liturgien, Zauber und Anrufungen (bspw. Böser Blick, Herr über das Tierreich).}}

\newglossaryentry{energiesteigerungskosten}
{
    name={Energiesteigerungskosten},
    description={Jeder Punkt kostet Zielwert EP.}}
        
        
        
        
    

\newacronym{pw}{PW}{Probenwert}
\newacronym{ew}{EW}{Erfolgswert}
\newacronym{3w20}{3w20}{3 20-seitige Würfel}
\newacronym{1w20}{1w20}{1 20-seitiger Würfel}
\newacronym{I}{I}{1w20 Probe}
\newacronym{dg}{DG}{Detailgrad}
\newacronym{ep}{EP}{Erfahrungspunkte}
%\newglossaryentry{#1}
%{
%    name=#1,
%    description={#2}
%}