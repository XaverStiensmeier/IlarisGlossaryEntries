\newglossaryentry{hinterhalt}
{
    name={Hinterhalt},
description={Für einen erfolgreichen Hinterhalt muss dir eine vergleichende Pirschen- oder Untertauchen-Probe gegen die Wachsamkeit deiner Opfer gelingen. Gelingt sie, gelten die Opfer als überrascht und die Initiativephasen aller Angreifer werden einmal durchlaufen, ohne dass die überraschten Opfer Aktionen oder Reaktionen durchführen können. Danach beginnt der Kampf wie gewöhnlich. Gerade bei vielen Beteiligten, wie bei einem Räuberüberfall, sollten \glslink{gruppenprobe}{Gruppenproben} gewürfelt werden. Werden die Spielercharaktere überfallen, sollte die Probe gegen den passiven Wachsamkeitswert der Charaktere abgelegt werden, um sie nicht vorzuwarnen.}}

\newglossaryentry{attacke und verteidigung}
{
    name={Attacke und Verteidigung},
    description={In einer Aktion \gls{konflikt} oder \gls{volle offensive} kann dein Charakter einen Feind in Reichweite angreifen. Normalerweise versucht der Angegriffene einen Treffer mit einer Verteidigung zu verhindern und es wird eine vergleichende Probe (I) auf die verwendeten Kampffertigkeiten abgelegt. Gewinnst du den Vergleich, kannst du deinem Gegner Schaden zufügen, der von deiner Waffe abhängt. Kann sich der Angegriffene nicht verteidigen, etwa weil er kampfunfähig oder überrascht ist oder Aktionen Konzentration aufwendet, ist der Angriff Routine (12, I). Auch hier gelten dieselben Modifikatoren wie für eine normale Verteidigung.}}

\newglossaryentry{triumph kampf}
{
    name={Triumph (Kampf)},
    description={Mit einem Triumph bei einem Kampfwurf darfst du nach dem Wurf noch ein zusätzliches Kampfmanöver ansagen (bei variabler Erschwernis bis maximal –4) oder ein Manöver mit variabler Schwierigkeit um bis zu 4 Punkte verbessern. Dein Erfolgswert sinkt dadurch nicht.}}

\newglossaryentry{patzer kampf}
{
    name={Patzer (Kampf)},
    description={Ein Patzer lässt deinen Angriff oder deine Verteidigung spektakulär misslingen und gilt als \gls{triumph kampf} für deinen Gegner. Dieser darf wie bei einem Triumph seinen Angriff oder seine Verteidigung verbessern und sein \gls{ew} wird berechnet, als ob er eine 20 geworfen hätte. Nach Spielleiterentscheid können Triumph und Patzer auch andere, zur Situation passende, Effekte haben. Wenn sowohl Attacke als auch Verteidigung aktiv gewürfelt werden, treten Triumphe und Patzer nur bei Attacken auf.}}

\newglossaryentry{chaotischer kampf}
{
    name={Chaotischer Kampf},
    description={In absoluter Dunkelheit oder unter ähnlich schwierigen Bedingungen wird ein Kampf zu einem chaotischen Hauen und Stechen. Das nennen wir einen chaotischen Kampf. Dabei kann der Spielleiter die Patzerchance auf 2 oder sogar 3 auf dem W20 erhöhen; diese Würfe lassen die Probe in jedem Fall misslingen und bedeuten, dass ein zufälliger Kämpfer in Reichweite getroffen wurde – dich selbst eingeschlossen.}}

\newglossaryentry{meucheln}
{
    name={Meucheln},
    description={Angriffe auf ein überraschtes Opfer haben nur eine Schwierigkeit von 12, die noch durch Positionsmodifikatoren verringert werden kann.  Das ermöglicht dir einen vernichtenden Erstschlag. Außerdem umgehen geeignete Waffen wie Dolche bei einem Meuchelangriff den Rüstungsschutz des Getroffenen. Erfahrene Meuchler nutzen die Aktion volle Offensive zusammen mit den Manövern Todesstoß und Wuchtschlag (oder gezielter Schlag auf Brust, Bauch oder Kopf im Zonensystem).}}

\newglossaryentry{betäuben}
{
    name={Betäuben},
    description={Betäubungsangriffe mit einer stumpfen Waffe werden mit einer Kombination aus voller Offensive, Stumpfer Schlag und Hammerschlag oder Wuchtschlag (oder gezielter Schlag auf den Kopf im Zonensystem) durchgeführt.}}

\newglossaryentry{modifikatoren im kampf}
{
    name={Modifikatoren im Kampf},
    description={Häufig modifiziert die Umgebung Proben auf deine Kampffertigkeiten. Der Einfachheit halber sind die häufigsten Modifikatoren in Stufen eingeteilt. Diese Stufen können durch Vorteile verändert werden; zum Beispiel senkt Angepasst I (Dunkelheit) die Modifikatoren durch schlechte Lichtverhältnisse um eine Stufe. Modifikatoren aus unterschiedlichen Quellen (z.B. Lichtverhältnisse und Untergrund) sind immer kumulativ.\\
    Manchmal betrifft ein Modifikator alle Beteiligten – dann könnt ihr ihn ignorieren. Das ändert nichts an den Trefferchancen, aber spart wertvolle Zeit. Noch öfter wirkt ein Modifikator aber auf fast alle Beteiligten. In diesem Fall solltet ihr nicht die Mehrheit mit einem Malus belegen, sondern stattdessen der Minderheit einen Bonus in gleicher Höhe verleihen.
\begin{tabularx}{\linewidth}{Xp{0.5cm}Xp{0.5cm}}
\begriff{Position} & &  \begriff{Kämpfer} & \\
Gegner sind abgewandt & +4 & ideale Reichweite & +2\\
vorteilhaft & +2 & unpassende Reichweite & -2\\
normal & 0 & je zusätzliche Aktion & -4\\
unvorteilhaft, kniend & -2 & Nebenwaffe & -4\\
liegend & -4 & &\\
& & &\\
\begriff{Untergrund} & & \begriff{Licht} & \\
unsicher, knietiefes Wasser & -2 & Dämmerung & -2\\
eisig, hüfttiefes Wasser & -4 & Mondlicht & -4\\
schultertiefes Wasser & -8 & Sternenlicht & -8
\end{tabularx}}}

\newglossaryentry{reichweite}
{
    name={Reichweite},
    description={Ein Zweihänder ist eine mächtige Waffe, doch der Glaube an ihre grundsätzliche Überlegenheit hat schon manch unerfahrene Abenteurerin im Gewühl des Enterkampfs den Kürzeren ziehen lassen. Spieltechnisch unterscheiden wir zwischen Kämpfen auf offenem Feld, wie einer Feldschlacht oder einem Duell, und Kämpfen auf engem Raum, wie in der Garether Kanalisation oder dichtem Unterholz. Auf offenem Feld bringt eine lange Waffe (Reichweite 2) einen Bonus von +2, eine kurze Waffe (Reichweite 0) einen Malus von –2. Auf engem Raum ist es genau umgekehrt. Mittellange Waffen (Reichweite 1) erleiden in beiden Fällen weder Bonus noch Malus. Führst du mehrere Waffen, dann zählt deine \gls{hauptwaffe}. Siehe auch \gls{modifikatoren im kampf}.}}

\newglossaryentry{hauptwaffe}
{
    name={Hauptwaffe},
    description={Zu Beginn des Kampfes wählst du eine Hauptwaffe. Alle anderen Waffen zählen als Nebenwaffen. Attacken und Verteidigungen mit Nebenwaffen sind um -4 erschwert. Du kannst deine Hauptwaffe mit einer Aktion \gls{bereit machen} ändern.}}

\newglossaryentry{nebenwaffe}
{
    name={Nebenwaffe},
    description={Siehe \gls{hauptwaffe}.}

\newglossaryentry{kontrollbereich}
{
    name={Kontrollbereich},
    description={Der Kontrollbereich ist die Umgebung deines Charakters bis zur Reichweite seiner längsten Waffe. Wann immer sich ein Gegner in diesem Bereich bewegt, ohne dich zu beachten, darfst du sofort einen \gls{passierschlag} mit einer ausreichend langen Waffe ausführen.}}

\newglossaryentry{passierschlag}
{
    name={Passierschlag},
    description={Ein Passierschlag ist eine Attacke, die in einer Reaktion ausgeführt wird und gegen die wie gewöhnlich verteidigt werden kann. Typische Situationen für Passierschläge sind, wenn sich ein Gegner rückwärts bewegt, deinen \gls{kontrollbereich} passiert - ohne dich zu beachten - oder aus dem Nahkampf flüchtet, ohne das Manöver \gls{entfernung verändern} zu nutzen.}}

\newglossaryentry{bodenplan}
{
    name={Bodenplan},
    description={Diese optional Regel ersetzt die Basisregeln zu Reichweiten. mit Hexfeldern. Sie ersetzen die Basisregeln zur Reichweite.
Die Bewegung in ein benachbartes Feld entspricht einem Schritt. Normalerweise darfst du dich auch auf das Feld eines Gegners oder Verbündeten bewegen, aber nicht durch ein solches Feld. Die Reichweite gibt die Entfernung von dir in Hexfeldern an, in der deine Waffe am effektivsten ist. Sind deine Gegner ein Feld näher oder weiter entfernt, erleidest du einen Malus von –2, bei zwei Feldern einen Malus von –4. Du kannst keinen Gegner angreifen, der mehr als zwei Felder entfernt ist.\\
Der \gls{kontrollbereich} ist auch hier die Umgebung eines Kämpfers bis zur Reichweite seiner längsten Waffe. Bewegungen darin sind gefährlich: Wann immer deine Bewegung auf einem Feld im Kontroll­bereich eines Gegners beginnt und die Entfernung zu diesem verändert, darf dieser einen \gls{passierschlag} gegen dich ausführen. Dabei kann der Gegner wählen, ob er den Passier­schlag vor oder nach deiner Bewegung ausführt. Ungefährlicher ist die Bewegung zusammen mit dem Manöver \gls{entfernung verändern}: Wenn dir das Manöver gelingt, erhält der getroffene Gegner in dieser Runde keine Passierschläge durch deine Bewegung. Andere Gegner können natürlich immer noch Passierschläge ausführen – weswegen eng gestaffelte Speerformationen auch so tödlich sind.}}

\newglossaryentry{aufstehen}
{
    name={Aufstehen},
    description={Das Aufstehen kostet im Kampf eine Aktion Konflikt. Allerdings kann ein Kontrahent versuchen, das mit einer Reaktion zu verhindern, dann muss dir zusätzlich eine vergleichende GE-Probe (I) gelingen. Um nur von der liegenden in eine kniende oder von der knienden in eine stehende Position zu kommen, ist keine Probe nötig – wohl aber die Aktion Konflikt.}}
    

\newglossaryentry{manöver}
{
    name={Manöver},
    description={Du kannst deinen Gegner auf verschiedenste Arten bezwingen, sei es durch rohe Gewalt oder raffiniertes Klingenspiel. Spieltechnisch kannst du vor deinen Attacken und Verteidigungen Manöver ansagen, die deine Proben häufig erschweren, aber auch eine besondere Manöverwirkung versprechen. Diese Wirkung tritt ein, wenn du dich bei der vergleichenden Kampfprobe durchsetzt. Manche Manöver erlauben noch eine Gegenprobe (I) auf das angegebene Attribut gegen den EW des Manövers, um die Manöverwirkung abzuwenden.}}

\newglossaryentry{basismanöver}
{
    name={Basismanöver},
    description={Basismanöver kannst du jederzeit durchführen.}}

\newglossaryentry{eingeschränktes basismanöver}
{
    name={Eingeschränktes Basismanöver},
    description={Eingeschränkte Basismanöver erfordern eine bestimmte Situation oder Waffe.}}

\newglossaryentry{aufbauendes manöver}
{
    name={Aufbauendes Manöver},
    description={Aufbauende Manöver setzen einen bestimmten Vorteil voraus.}}