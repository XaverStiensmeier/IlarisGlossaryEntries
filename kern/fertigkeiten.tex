%Fertigkeit
\newglossaryentry{fertigkeit}
{
    name={Fertigkeit},
    description={Fertigkeiten setzen sich aus verwandten Bereichen zusammen, von denen dein Charakter grundsätzlich etwas versteht, aber die er nicht alle gleich gut beherrscht – sogenannten \gls{talent}{Talenten}.}}

\newglossaryentry{nahkampffertigkeit}
{
    name={Nahkampffertigkeit},
    description={Erste Nahkampferfahrung muss sich dein Charakter mühsam erarbeiten. Sobald er jedoch ein Gefühl für die Dynamik eines Kampfes besitzt, wird er auch andere Arten des Kampfes schnell begreifen. Deswegen steigerst du Nahkampffertigkeiten normalerweise nach \gls{steigerungsfaktor} 4. Besitzt du aber vor dem Steigern Nahkampffertigkeiten mit höheren \glslink{fertigkeitswert}{Fertigkeitswerten}, gilt ein Faktor von 2. Für Talente gilt immer \gls{steigerungsfaktor} 2. Durch diese Regel ändert sich übrigens nichts an den Gesamtkosten, egal in welcher Reihenfolge du steigerst.}}

\newglossaryentry{handgemenge}
{
    name={Handgemenge (GE/KK/MU, 4/2)},
    description={Auf Handgemenge würfelst du bei allen waffenlosen Nahkampftechniken, im Umgang mit Handgemengewaffen wie Messern, Dolchen, Schlagstöcken oder Kettenstäben und bei der Verwendung von Schilden. Talente: Handgemengewaffen, Unbewaffnet, Schilde.}}

\newglossaryentry{hiebwaffen}
{
    name={Hiebwaffen (GE/KK/MU, 4/2)},
    description={Mit der Fertigkeit Hiebwaffen führst du alle Arten von stumpfen und scharfen Hiebwaffen, wie Keulen, Äxte und Hämmer. Talente: Einhandhiebwaffen, Zweihandhiebwaffen.}}

\newglossaryentry{klingenwaffen}
{
    name={Klingenwaffen (GE/KK/MU, 4/2)},
    description={Im Kampf mit allen Varianten von Schwertern, Säbeln und Fechtwaffen werden Proben auf die Fertigkeit Klingenwaffen abgelegt. Dolche gehören nicht zu den Klingenwaffen – sie werden mit der Fertigkeit \gls{handgemenge} genutzt. Talente: Einhandklingenwaffen, Zweihandklingenwaffen.}}

\newglossaryentry{stangenwaffen}
{
    name={Stangenwaffen (GE/KK/MU, 4/2)},
    description={Alle Waffen mit einem langen Schaft gelten als Stangenwaffen. Zu den klassischen Stangenwaffen zählen Speere, Hellebarden und der Schnitter. Zusätzlich werden Kampfstäbe und die Lanzen berittener Kämpfer mit dieser Fertigkeit genutzt. Talente: Infanteriewaffen und Speere, Lanzenreiten (\gls{verbilligt})}}

\newglossaryentry{fernkampffertigkeiten}
{
    name={Fernkampffertigkeiten},
    description={}}

\newglossaryentry{schusswaffen}
{
    name={Schusswaffen (FF/IN/KK, 3)},
    description={Mit der Fertigkeit Schusswaffen bedienst du alle Waffen, mit denen Geschosse auf den Gegner abgefeuert werden. Dazu gehören so verbreitete Schusswaffen wie der Bogen oder die Armbrust, aber auch die Schleuder oder das Blasrohr. Talente: Armbrüste, Blasrohre (verbilligt), Bögen}}

\newglossaryentry{wurfwaffen}
{
    name={Wurfwaffen (FF/IN/KK, 2)},
    description={Als Wurfwaffen gelten sämtliche geworfenen Geschosse wie Diskusse oder Wurfspeere. Zusätzlich umfassen sie kurze Wurfwaffen, zu denen Dolche, Wurfsterne und improvisierte Geschosse wie einen Stein oder eine Flasche gehören. Talente: Diskusse, kurze Wurfwaffen, Schleudern (\gls{verbilligt}), Wurfspeere.}}

\newglossaryentry{profane fertigkeiten}
{
    name={Profane Fertigkeiten},
    description={Profane Fertigkeiten können von jedem Charakter genutzt werden.}}

\newglossaryentry{athletik}
{
    name={Athletik (GE/KK/KO, 3)},
    description={Athletik umfasst alle Aktivitäten, bei denen der Charakter seinen ganzen Körper kurz- oder längerfristig koordiniert einsetzen muss. Talente: Laufen, Klettern, Schwimmen, Reiten, Akrobatik.
    \begin{description}
\item Laufen kommt bei Verfolgungsjagden zu Fuß zum Einsatz, wenn du einen Verbrecher stellen oder einem Raubtier entkommen möchtest.
\item Mit Klettern überwindest du alle Arten von Hindernissen.
\item Schwimmen erlaubt eine schnellere Fortbewegung im Wasser und längere Tauchgänge.
\item Reiten ist die Fähigkeit, ein Pferd, ein Kamel oder einen Hippogriff zu kontrollieren und als schnelles Reisemittel oder im Kampf einzusetzen.
\item Akrobatik wird für gewagte Kunststücke und Balanceakte verwendet und kann auch den Fallschaden reduzieren.
\end{description}}}

\newglossaryentry{heimlichkeit}
{
    name={Heimlichkeit (GE/IN/MU, 2)},
    description={Heimlichkeit ermöglicht deinem Charakter, ungehört und ungesehen zu bleiben. Die Probe wird häufig vergleichend gegen die Wachsamkeit deines Kontrahenten abgelegt. Talente: Pirschen, Untertauchen.
\begin{description}
\item Pirschen ermöglicht das Schleichen, Verstecken und Lauern in der freien Natur. Du pirschst dich auf der Jagd an einen wilden Hirsch heran, legst einen Hinterhalt an einer Reichsstraße oder beobachtest ungesehen das Lager einer Orkbande.
\item Untertauchen erlaubt dir das Schleichen, Verstecken und Beschatten in der Zivilisation. Du verschwindest in der Menschenmenge am Marktplatz, steigst ungehört in die Villa eines Ratsherren ein oder beschattest unauffällig einen Hehler.
\end{description}}}

\newglossaryentry{selbstbeherrschung}
{
    name={Selbstbeherrschung (KO/MU/MU, 3)},
    description={Selbstbeherrschung erlaubt dir, widrige Umstände, körperliche Schmerzen oder Ablenkungen zu ignorieren und dich auf das einzig Wichtige zu konzentrieren. Talente: Willenskraft, Zähigkeit.
\begin{description}
\item Willenskraft erlaubt dir, Beeinflussungen und Versuchungen zu widerstehen und dich nicht ablenken zu lassen. Du kannst dich im Chaos des Gefechts auf einen Zauber konzentrieren oder einen Heiligen Befehl abschütteln. 
\item Zähigkeit hilft dir, Schmerzen und Strapazen wegzustecken. So kannst du trotz zahlreicher Wunden noch handlungsfähig bleiben und einen Kampf zu euren Gunsten wenden oder eine ganze Nacht hindurch wachen.
\end{description}}}

\newglossaryentry{wahrnehmung}
{
    name={Wahrnehmung (IN/IN/KL, 4)},
    description={Wahrnehmung ist die Fähigkeit, selbst kleinste Sinneseindrücke zu erfassen und richtig zu interpretieren. Talente: Menschenkenntnis, Sinnenschärfe, Wachsamkeit.
\begin{description}
\item Menschenkenntnis ist deine Fähigkeit, die Absichten deines Gegenübers zu durchschauen. Damit enttarnst du Lügen und falsche Annäherungsversuche.
\item Sinnenschärfe ist die aktive Verwendung deiner Sinne, um einen Kollaborateur in einem geschäftigen Wirtshaus zu belauschen, die Flagge eines nahenden Schiffes zu erkennen oder die Nadel im Heuhaufen zu finden.\item Wachsamkeit fasst den passiven Einsatz deiner Sinne zusammen. Mit ihr entdeckst du Hinterhalte, bemerkst eine Unstimmigkeit an einem Tatort oder eine verborgene Notiz am Wegesrand.
\end{description}}}

\newglossaryentry{autorität}
{
    name={Autorität (CH/KL/MU, 3)},
    description={Autorität erlaubt es deinem Charakter, sich Respekt zu verschaffen und seinen Willen durchzusetzen. Talente: Anführen, Einschüchtern, Rhetorik.
\begin{description}
\item Mit Anführen leitest und motivierst du Untergebene. Du kannst mit Anführen Löscharbeiten oder einen Trupp von Kämpfern koordinieren, um ihnen so Vorteile zu verschaffen (S. 46). Gerade in großen Schlachten entscheidet der Heerführer über Sieg oder Niederlage.
\item Einschüchtern jagt dem Gegenüber Angst ein und bringt ihn so zu einer gewünschten Handlung. Die Probe wird oft vergleichend gegen den MU oder die Menschenkenntnis des Gegenübers abgelegt.
\item Rhetorik beinhaltet zahlreiche Fähigkeiten und Kniffe, um die eigenen Argumente wirkungsvoll einzusetzen und die des Gegenübers zu entkräften. Du kannst Rhetorik nur einsetzen, wenn dein Charakter von seinem Standpunkt überzeugt ist – dreiste Lügen fallen unter Überreden.
\end{description}}}

\newglossaryentry{beeinflussung}
{
    name={Beeinflussung (CH/CH/IN, 3)},
    description={Hierunter fallen sämtliche Techniken, um das Gegenüber zu einer gewünschten Handlung zu bewegen. Dabei werden Betören und Überreden oft als vergleichende Proben gegen die Menschenkenntnis des Opfers gewürfelt. Talente: Betören, Überreden.
\begin{description}
\item Beim Betören nutzt du deine persönliche Ausstrahlung, um zu bekommen, was du willst. Das reicht von Kleinigkeiten wie einem Freibier bis hin zum Verrat geheimer Staatsinformationen.
\item Überreden bedeutet den geschickten Einsatz von Übertreibungen, Unwahrheiten oder Lügen, um das Gegenüber zumindest kurzfristig zu beeinflussen. Mit Überreden feilschst du am Marktplatz, überlistest eine Stadtwache oder infiltrierst ein Borbaradianerkloster ein.
\end{description}}}

\newglossaryentry{gebräuche}
{
    name={Gebräuche (CH/IN/KL, 2)},
    description={Gebräuche stellt das Wissen um fremde und vertraute Kulturen sowie die dort herrschenden gesellschaftlichen Normen dar. Wie heißt der lokale Markgraf und wie sprichst du ihn an? In welchem Viertel findest du eine vertrauenswürdige Herberge, verschwiegene Schläger oder die neuesten Gerüchte? Wie funktioniert das Kamelspiel und auf welches Getränk könntest du deinen Gegner einladen? Herrscht in der nächsten Stadt ein Waffenverbot und wie streng wird es ausgelegt? Gebräuche beantwortet all diese Fragen und verhindert, dass du unangenehm auffällst. Talente: nach Kultur; Mittelreich, Horasreich, Tulamidenlande, Südaventurien, Bornland (verbilligt), Thorwal (verbilligt), Maraskan (verbilligt), Elfen (verbilligt), Zwerge (verbilligt) und weitere. Das Talent der eigenen Heimat ist kostenlos.}}

\newglossaryentry{derekunde}
{
    name={Derekunde (FF/IN/KL, 2},
    description={Darunter fällt das theoretische Wissen über Natur, Tiere und Pflanzen sowie die Geographie. Talente: Geographie, Pflanzenkunde, Tierkunde.
\begin{description}
\item Geographie befasst sich nicht nur mit Reiserouten und fernen Ländern, sondern auch mit dem Sternenhimmel, der Kartographie und Navigation. Ein Geograph kann Schatzkarten lesen und anfertigen, die Position eines Schiffes bestimmen und Sternenkonstellationen deuten.
\item Pflanzenkundige erforschen die vielen nützlichen, gefährlichen oder wundersamen Gewächse Aventuriens. Sie sammeln die Kräuter für eine Heilsalbe, kennen die Gefahr von Jagdgras und finden essbare Früchte oder Wurzeln.
\item Tierkunde bedeutet die Kenntnis der aventurischen Fauna. Dein Charakter kann einen Schleimfleck als Spur einer Riesenamöbe erkennen, einen gereizten Bären beruhigen und kennt die Schwachstelle eines Tatzelwurms.
\end{description}}}

\newglossaryentry{mythenkunde}
{
    name={Mythenkunde (IN/KL/KL, 2)},
    description={Mythenkunde ist das Wissen über die Götter und ihre Diener, sowohl die eigene Religion, als auch – weniger genau – fremde Religionen betreffend. Zusätzlich gehört dazu die Kenntnis der Geschichte sowie bekannter Sagen und Legenden, was meist schwer voneinander zu trennen ist. Talente: Geschichten und Legenden, Götter und Kulte.
\begin{description}
\item Geschichten und Legenden ist das Wissen um alte Überlieferungen. Damit kannst du das Alter von Grabmälern und Artefakten bestimmen oder aus einer Sage die Vorlieben und Schwachpunkte eines Riesen ableiten.
\item Götter und Kulte befasst sich mit dem Wesen der Götter, ihrer Schöpfung und ihren sterblichen Dienern. Dein Charakter kennt den Aufbau und die Ziele der Kirchen und weiß, wie man ihren Angehörigen gegenübertritt. Genauso kann er eine alte Kultstätte einer Gottheit oder vielleicht einem Erzdämonen zuordnen.
\end{description}}}

\newglossaryentry{magiekunde}
{
    name={Magiekunde (KL/KL/MU, 2)},
    description={Die astrale Kraft durchzieht die ganze Welt und wird seit Jahrtausenden von Zauberern genutzt. Magiekunde ist das theoretische Wissen über diese Kraft und ihre Verwendung. Talente: Dämonenkunde, Elementarkunde, Magietheorie, Zauberpraxis.
\begin{description}
\item Dämonenkunde beschäftigt sich mit den Erzdämonen, ihren Dienern und dämonischen Zaubern. Du kannst Dämonen benennen und kennst ihre Schwachstellen.
\item Elementarkunde befasst sich mit den sechs Elementen, der Elementarbeschwörung und den elementaren Zaubern.
\item Magietheorie hilft dir bei der Einschätzung und der Analyse von magischen Phänomenen.
\item Zauberpraxis umfasst die Kenntnis verbreiteter Zauber und Rituale, ihrer Wirkung und möglicher Gegenmaßnahmen.
\end{description}}}

\newglossaryentry{überleben}
{
    name={Überleben (GE/FF/KO, 3)},
    description={Hierunter fallen alle Fähigkeiten, die zum Überleben in der Wildnis erforderlich sind. Du kannst dich orientieren oder einen Lagerplatz, Nahrung und Wasser finden. Außerdem bist du mit typischen Gefahren der Wildnis vertraut und kannst tierische oder menschliche Fährten deuten und ihnen folgen. Dieses Wissen ist rein praktisch; exotische Tiere, Monster oder seltene Heilpflanzen fallen nicht in diesen Bereich. Talente: nach Region; Hoher Norden (nördlich der Salamandersteine), Nordaventurien (nördlich von Steineichenwald und Ysilsee), Mittelaventurien (nördlich von Eisenwald, Phecanowald und Raschtulswall), Südaventurien (nördlich von Drôl und Thalusien), Tiefer Süden (südlich davon) sowie Maraskan (verbilligt), Wüste (verbilligt), Gebirge (verbilligt) und Meer (verbilligt). Die Gebiete schließen sich gegenseitig aus, Maraskan gehört zum Beispiel nicht zum Süden.}}

\newglossaryentry{verschlagenheit}
{
    name={Verschlagenheit (FF/IN/MU, 2)},
    description={Verschlagenheit ist das Handwerk von zwielichtigen Gestalten, die Diebstahl, Falschspiel und Einbruch nicht scheuen. Talente: Falschspielen (verbilligt), Schlösser knacken, Stehlen.
\begin{description}
\item Falschspieler helfen dem Glück etwas nach, zum Beispiel beim Hütchenspiel oder beim Boltan. Proben auf Falschspielen werden oft vergleichend gegen die Wachsamkeit der Mitspieler abgelegt.
\item Schlösser knacken öffnet mit Hilfe eines Dietrichs oder einer Haarnadel Schatzkisten und Tresorräume. Außerdem kann dein Charakter Fallen entschärfen.
\item Mit Stehlen kannst du zur richtigen Zeit am richtigen Ort sein, um die richtige Person unauffällig um ihren Besitz zu erleichtern. Außerdem können Diebe den Wert ihrer Beute einschätzen und Kontakt zu einem Hehler aufnehmen.
\end{description}}}

\newglossaryentry{Alchemie (FF/FF/KL, 3)}
{
    name={Alchemie (FF/FF/KL, 3)},
    description={Alchemie ist die Wissenschaft von der Umwandlung der Stoffe. Mit ihr kannst du nützliche Tränke herstellen oder unbekannte Gebräue analysieren; für beides ist aber zumindest ein einfacher Alchemiekoffer notwendig. Talente: Analyse (verbilligt), Magische Elixiere, Profane Alchemika.
\begin{description}
\item Mit einer alchemistischen Analyse kannst du feststellen, ob eine verstaubte Phiole einen Heiltrank oder ein tödliches Gift enthält.
\item Magische Elixiere beinhaltet Heiltränke, Stärkungsmittel und Verwandlungselixiere, kurz die hohe Kunst der Alchemie.
\item Profane Alchemika ist das weit bodenständigere Handwerk der Meuchler, Waldläufer und Kräuterfrauen, mit dem du Gifte, Heilsalben oder Rauchbomben herstellen kannst.
\end{description}}}

\newglossaryentry{heilkunde}
{
    name={Heilkunde (CH/FF/KL, 3)},
    description={Ein Heilkundiger bekämpft tödliche Blutungen, heimtückische Gifte sowie langwierige Krankheiten und hilft seinen Gefährten so, wieder auf die Beine zu kommen. Talente: Gifte und Krankheiten, Wundheilung
\begin{description}
\item Mit dem Talent Gifte und Krankheiten stoppst du eine Tulmadron-Vergiftung, erkennst die ersten Anzeichen für Zorganpocken und kannst einen Ghulbiss behandeln.
\item Wundheilung ermöglicht es dir, im Kampf entstandene Blutungen zu stoppen und die Heilung von Verletzungen zu fördern. Zusätzlich kannst du durch die Untersuchung von Wunden Rückschlüsse auf die Tatwaffe und den Tathergang ziehen.
\end{description}}}

\newglossaryentry{handwerk}
{
    name={Handwerk (FF/FF/KK)},
    description={Mit Handwerk schmiedest du Waffen und Rüstungen, fertigst filigrane Mechaniken und stabile Truhen. Talente: Holzbearbeitung (verbilligt), Mechanik (verbilligt), Schmieden.
\begin{description}
\item Mit Holzbearbeitung fertigst du Schilde, Bögen oder ein improvisiertes Floß, reparierst eine Zugbrücke, schnitzt aus einem Knochen eine Pfeilspitze oder flickst ein Leck in einem Schiff.
\item Mit Mechanik stellst du Apparaturen wie Armbrüste, Flaschenzüge oder Südweiser her und wartest sie.
\item Schmieden ist die Herstellung und Reparatur von Schwertern, Äxten oder Plattenpanzern.
\end{description}}}
