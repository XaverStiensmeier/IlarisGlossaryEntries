%CH
\newglossaryentry{eindrucksvoll I}
{
    name={Eindrucksvoll I},
    description={In Rededuellen sind Proben auf Betören und Einschüchtern um 2 erleichtert.}}

\newglossaryentry{eindrucksvoll II}
{
    name={Eindrucksvoll II},
    description={In Rededuellen sind Proben auf Betören und Einschüchtern um +2 erleichtert. Bei Proben auf Betören, Einschüchtern und Gebräuche gelten Patzer als gewöhnlich misslungen, außer sie entstehen durch eine ungewohnte Umgebung.}}

\newglossaryentry{soziale anpassungsfähigkeit}
{
    name={Soziale Anpassungsfähigkeit},
    description={Du bist immun gegen alle Auswirkungen durch ungewohnte Umgebung.}}

\newglossaryentry{starke aura}
{
    name={Starke Aura},
    description={Während eines Rededuells kannst du eine CH-Probe gegen die Willenskraft deines Gegenübers ablegen. Wenn die Probe gelingt, kannst du eine Probe in diesem Rededuell wiederholen.}}

%FF

\newglossaryentry{routiniert I}
{
    name={Routiniert I},
    description={Die Dauer zur Fertigung handwerklicher Produkte, zur Wundversorgung und zum Schlösserknacken verkürzt sich um ein Viertel des unmod. Werts.}}

\newglossaryentry{routiniert II}
{
    name={Routiniert II},
    description={Die Dauer zur Fertigung handwerklicher Produkte, zur Wundversorgung und zum Schlösser knacken verkürzt sich um ein weiteres Viertel des unmodifizierten Werts. Bei Proben auf Alchemie, Handwerk, Heilkunde, und Verschlagenheit gelten Patzer als gewöhnlich misslungen.}}

\newglossaryentry{improvisation}
{
    name={Improvisation},
    description={Unzureichendes Werkzeug oder minderwertige
Verbrauchsmaterialien gelten um eine Stufe höher.}}

\newglossaryentry{meisterwerk}
{
    name={Meisterwerk},
    description={Mit hervorragenden Materialien und dem doppelten Zeitaufwand kannst du ein Meisterwerk schaffen. Dieses erhält zusätzlich 2x die Modifikation hohe Qualität und verbesserte Eigenschaften nach Meisterentscheid. Solche Werke sind extrem selten und Gegenstand von Sagen und Legenden. Der Vorteil kann nur mit Fertigkeiten mit einem unmod. Probenwert von mindestens 16 angewandt werden.}}

%GE
\newglossaryentry{flink I}
{
    name={Flink I},
    description={\gls{gs}+1. Bei \gls{athletik}{Athletikproben} gelten Patzer als gewöhnlich misslungen.}}

\newglossaryentry{flink II}
{
    name={Flink II},
    description={\gls{gs}+1.}}

\newglossaryentry{katzenhaft}
{
    name={Katzenhaft},
    description={Erschwernisse durch unsicheren Untergrund sinken um eine Stufe und gegen den Schaden aus Stürzen, Zusammenstößen usw. kannst du deine \gls{ge} als \gls{ws} verwenden.}}

\newglossaryentry{körperbeherrschung}
{
    name={Körperbeherrschung},
    description={Du kannst Nah- oder Fernkampfangriffen, elementaren Schadenszaubern oder ähnlichen Schadensquellen mit einer \gls{ge}-Gegenprobe entgehen. Der Einsatz von Körperbeherrschung wird angesagt, nachdem übliche Verteidigungsmöglichkeiten (wie eine Verteidigung) versagt haben, aber bevor der Schaden bestimmt wird. Anschließend erleidest du einen Punkt \gls{erschöpfung}.}}

%IN
\newglossaryentry{vorausschauend I}
{
    name={vorausschauend I},
    description={In Rededuellen sind Proben auf Rhetorik und Überreden um +2 erleichtert.}}

\newglossaryentry{vorausschauend II}
{
    name={Vorausschauend II},
    description={In Rededuellen sind Proben auf Rhetorik und Überreden um +2 erleichtert. Bei Proben auf Überreden, Rhetorik und Menschenkenntnis gelten Patzer als gewöhnlich misslungen, außer sie entstehen durch eine ungewohnte Umgebung.}}

\newglossaryentry{bedächtig}
{
    name={Bedächtig},
    description={Wenn du in einem Rededuell abwartest, ist deine Probe um +4 erleichtert.}}

\newglossaryentry{empathie}
{
    name={Empathie},
    description={Du kannst eine \gls{in}-Probe gegen die \gls{willenskraft} deines Gegenübers ablegen. Wenn die Probe gelingt, erfährst du eine seiner Schwächen. Eine einmal abgelegte Probe, egal ob ge- oder misslungen, kannst du nur wiederholen, wenn du das Gegenüber besser kennengelernt hast.}}

%KK
\newglossaryentry{zerstörerisch I}
{
    name={Zerstörerisch I},
    description={Proben zum Zerstören oder Durchbrechen von Gegenständen sind um +4 erleichtert.}}

\newglossaryentry{zerstörerisch II}
{
    name={Zerstörerisch II},
    description={Proben zum Zerstören oder Durchbrechen von Gegenständen sind um weitere +4 erleichtert. Erlaubt das Manöver Hammerschlag gegen
Gegenstände. Bei waffenlosen Angriffen gegen Gegenstände verletzt du dich normalerweise nicht.}}

\newglossaryentry{muskelprotz}
{
    name={Muskelprotz},
    description={Einschüchtern-Proben im Kampf sind um +4 und du kannst ohne zusätzliche Erschwernis bis zu 4 Gegner einschüchtern.}}

\newglossaryentry{adrenalinschub}
{
    name={Adrenalinschub},
    description={Du kannst dir eine Probe bei einer körperlichen Tätigkeit (z.B. Schmieden, Laufen, Nahkampfangriff) um +4 erleichtern. Anschließend erleidest du einen Punkt Erschöpfung.}}
        
%KL
\newglossaryentry{scharfsinnig I}
{
    name={Scharfsinnig II},
    description={Proben bei einer Ermittlung oder Recherche sind um +2 erleichtert.}}

\newglossaryentry{scharfsinnig II}
{
    name={Scharfsinnig II},
    description={Proben bei einer Ermittlung oder Recherche sind um weitere +2 erleichtert. Bei Recherchen erhältst du einen zusätzlichen Informationsgrad, wenn der gewertete Würfel eine 12 oder höher zeigt.}}

\newglossaryentry{vorbereitung}
{
    name={Vorbereitung},
    description={Du kannst Proben auf profane Fertigkeiten besonders sorgfältig vorbereiten (sofern Vorbereitung sinnvoll ist). Wenn du die doppelte notwendige Zeit aufwendest, erhältst du +4 Punkte Erleichterung.}}

\newglossaryentry{eingebung}
{
    name={Eingebung},
    description={Pro Abenteuer kannst du den Spielleiter W3 mal um einen Tipp bitten. Diese Tipps sollen nicht das Abenteuer lösen, können dich aber auf Fehler in deinem Plan oder bisher übersehene Aspekte oder Zusammenhänge aufmerksam machen.}}

\newglossaryentry{abgehärtet I}
{
    name={Abgehärtet I},
    description={Proben zur Abwehr von Giften und Krankheiten sind um +4 erleichtert.}}

\newglossaryentry{abgehärtet II}
{
    name={Abgehärtet II},
    description={Proben zur Abwehr von Giften und Krankheiten sind um weitere +4 erleichtert und das Durchhaltevermögen steigt um 2.}}

\newglossaryentry{schnelle heilung}
{
    name={Schelle Heilung},
    description={Du regenerierst 2 Wunden pro durchschlafener Nacht und immer noch 1 Wunde, wenn die Schlafphase unterbrochen wurde.}}
        
\newglossaryentry{unverwüstlich}
{
    name={Unverwüstlich},
    description={\gls{ws} +1. Wenn \glslink{zähigkeit}{Zähigkeitsproben} zum Ignorieren von \gls{kampfunfähigkeit} misslingen, erleidest du keine \gls{erschöpfung}.}}

%MU
\newglossaryentry{willensstark I}
{
    name={Willensstark I},
    description={\gls{mr}+4.}}

\newglossaryentry{willensstark II}
{
    name={Willensstark II},
    description={\gls{mr}+4.}}

\newglossaryentry{geisterpanzer}
{
    name={Geisterpanzer},
    description={Gegen direkten Schaden aus Zaubern wie Fulminictus, Ignisphaero oder Hexengalle kannst du deinen \gls{mu} als \gls{ws} verwenden.}}

\newglossaryentry{unbeugsamkeit}
{
    name={Unbeugsamkeit},
    description={\gls{mr} +\gls{mu}/2. Mit einer \gls{aktion} \gls{konflikt} und einer \gls{konterprobe} (MU,16) kannst du einen auf dir liegenden \gls{zauber} abschütteln. Anschließend erleidest du einen Punkt \gls{erschöpfung}.}}