%gesundheit

\newglossaryentry{erschöpfung}
{
    name={Erschöpfung},
    description={Die Erschöpfung ist eine \gls{einschränkung}, die schneller heilt als \glslink{wunde}{Wunden} (siehe \gls{regeneration}).}}

\newglossaryentry{wunde}
{
    name={Wunde},
    description={Die Wunde ist eine \gls{einschränkung}, die langsamer heilt als \gls{erschöpfung} (siehe \gls{regeneration}).}}

\newglossaryentry{einschränkung}
{
    name={Einschränkung},
    description={\gls{erschöpfung} und \glslink{wunde}{Wunden} sind jeweils Einschränkungen. Spricht man von Einschränkungen im Plural ist die Summe aus Stufen \gls{erschöpfung} und \glslink{wunde}{Wunden} gemeint.}}

\newglossaryentry{regeneration}
{
    name={Regeneration},
    description={Damit die heilende Luft Aventuriens wirkt, müssen zwei Vorauss­etzungen gegeben sein: Erstens muss der Charakter ruhen oder besser noch schlafen, zweitens dürfen keine schädlichen Effekte wie Gifte, Krankheiten oder Durst auf ihn wirken. Unter diesen Umständen regeneriert ein Charakter eine \gls{wunde} pro durchgeschlafener Nacht von mindestens 6 Stunden oder einen Punkt \gls{erschöpfung} pro Stunde. Wenn sich dadurch Lücken in der Statusleiste ergeben, wandern die \glslink{einschränkung}{Einschränkungen} dahinter nach links.}}

\newglossaryentry{kampfunfähigkeit}
{
    name={Kampfunfähigkeit},
    description={Mit fünf \glslink{einschränkung}{Einschränkungen} wird dein Charakter potentiell kampfunfähig. Potentiell stürzt er, kann nicht mehr aufstehen und keine \glslink{aktion}{Aktionen} oder \glslink{reaktion}{Reaktionen} ausführen. Der Spielleiter entscheidet, ob der Charakter bei Bewusstsein ist. Du kannst eine \glslink{zähigkeit}{Zähigkeits-Probe} (12, \gls{I}) ablegen, um weiter kampffähig zu bleiben. Der Wundabzug gilt natürlich auch bei dieser Probe. Misslingt sie, erleidet dein Charakter durch diese Anstrengung noch einen zusätzlichen Punkt \gls{erschöpfung} und wird dennoch kampfunfähig. Die Kampfunfähigkeit endet nach einer Stunde oder sobald eine \gls{einschränkung} geheilt wurde – allerdings nur, wenn keine schädlichen Effekte wie Gifte, Krankheiten oder Durst auf deinen Charakter wirken.}}

\newglossaryentry{wundschmerz}
{
    name={Wundschmerz},
    description={Wenn dein Charakter auf einen Schlag zwei Einschränkungen erleidet, muss er sofort eine KO-Probe (20, I) ablegen. Für jede zusätzliche Wunde steigt die Schwierigkeit um weitere +4. Misslingt die Probe, wird er vom Schmerz überwältigt und ist betäubt. Er kann erst in seiner übernächsten Initiativephase wieder Aktionen und Reaktionen ausführen.}}

\newglossaryentry{wundabzüge}
{
    name={Wundabzüge},
    description={Auf deinem Charakterbogen findest du eine Statusleiste mit 8 Kästchen, in denen du erlittene Einschränkungen notieren kannst. Wunden notierst du dabei mit einem X , Erschöpfung mit einem /. Sobald dein Charakter insgesamt mehr als zwei Einschränkungen erlitten hat, erleidet er für jede weitere Einschränkung einen Abzug von –2 auf alle Proben. Den momentanen Wundabzug kannst du natürlich jederzeit auf der Statusleiste ablesen.}}

\newglossaryentry{blutungen und tod}
{
    name={Blutungen und Tod},
    description={Blutungen drohen dir, wenn du mehr als 4 Wunden durch Waffengewalt oder ähnliche akute Verletzungen erlitten hast. Solltest du dann kampfunfähig werden, oder bereits kampfunfähig sein und eine weitere Wunde durch Waffengewalt erleiden, muss dir eine KO-Probe (12, I) gelingen. Misslingt die Probe, beginnt dein Charakter zu bluten: Du musst nach jeweils WS Initiativephasen eine KO-Probe (12, I) ablegen, bei deren Misslingen du eine weitere Wunde erleidest. Eine Blutung endet bei erfolgreicher erster Hilfe (S. 33). Bei acht Einschränkungen ist auch für den zähesten Charakter die Grenze erreicht. Eine weitere Einschränkung bedeutet unweigerlich den Tod.}}

\newglossaryentry{gesundheit von gegenständen}
{
    name={Gesundheit von Gegenständen},
    description={Für Gegenstände wird in Ilaris dasselbe Gesundheitssystem wie für Lebewesen verwendet, nur einige Begriffe unterscheiden sich. Statt einer \gls{ws} verfügt ein Gegenstand über eine Härte. Ein Seil hat eine Härte von 2, ein stabile Holztür eine Härte von 8 und Stahlgitterstangen eine Härte von 24. Übersteigt der Schaden die Härte, erleidet der Gegenstand eine Beschädigung. Jede Beschädigung nach der zweiten kann die Verwendung des Gegenstands um –2 erschweren. Nach 5 Beschädigungen ist das Objekt unbrauchbar, kann aber noch repariert werden. Erst ein Gegenstand mit 9 Beschädigungen ist unwiderruflich zerstört.}}

\newglossaryentry{heilkunde}
{
    name={Heilkunde},
    description={Ein erfahrener Heilkundiger kann lebensbedrohliche Situationen abwenden. Mit erster Hilfe kann eine Blutung gestoppt werden oder die Auswirkung von Giften und Krankheiten gemildert werden. Dafür benötigst du so viele Aktionen Konzentration, wie die Patientin am Beginn der Heilung Wunden erlitten hat, mindestens aber 4 Initiativephasen. Außerdem musst du über Verbandsmaterial (Wunden) oder passende Heilkräuter (Gifte und Krankheiten) verfügen.\\
Darauf folgen heilungsfördernde Maßnahmen, wodurch der Patient sofort eine Wunde regeneriert. Pro Patient und Tag darf nur eine Probe angelegt werden. Eine solche Probe dauert eine halbe Stunde. Außerdem benötigst du heilende Salben oder ähnliche Hilfsmittel, Verbands­­material und sauberes Wasser. Die Schwierigkeit der Probe beträgt 16 + den Betrag der aktuellen Wundabzüge des Patienten.}}

\newglossaryentry{übernatürliche heilung}
{
    name={Übernatürliche Heilung},
    description={Heiltränke, Balsam und Wundsegen – übernatürliche Heilung kennt viele Wege. Trotzdem wirkt sie immer auf eine von zwei Arten: Entweder der heilende Effekt stärkt und beschleunigt die natürliche Regeneration, wie etwa der Zauber Ruhe Körper. Solche Effekte wirken meist langsam, heilen aber dafür Wunden auf direktem Weg. Oder der Effekt greift von außen in den Körper ein, wie der Balsam oder ein Heiltrank. Ein solcher Effekt bringt eine gewisse Anzahl an \gls{hp}. Für jede Überschreitung der WS verschwindet eine Wunde. Heilpunkte funktionieren also genau umgekehrt wie Schadenspunkte; dadurch ist es schwieriger, ein großes und kräftiges Wesen zu heilen. Allerdings wirken solche Effekte meist wesentlich schneller.}}

\newglossaryentry{schadensquellen}
{
    name={Schadensquellen},
    description={Stürze, Entbehrungen und Gifte sind nur einige Beispiele für die Gefahren, die deinen Charakter erwarten. Viele dieser Schadensquellen wirken nicht sofort, sondern erst nach einer gewissen Verzögerung. Ein Beispiel dafür wäre die Inkubationszeit einer Krankheit. Erst dann beginnt die Wirkungsdauer, in der du sofort und dann in bestimmten Intervallen Schaden erleidest. Verzögerung, Intervall und Schaden sind bei den jeweiligen Schadensquellen angegeben.}}

\newglossaryentry{körperliche anstrengung}
{
    name={Körperliche Anstrengung},
    description={Wie lange dein Charakter eine anstrengende Tätigkeit durchhält, hängt von deiner Konstitution und deiner Behinderung – kurz deinem Durchhaltevermögen (DH) ab. Das Durchhaltevermögen ergibt sich aus KO – 2 x BE, aber beträgt wenigstens 1. Es ist auf deinem Charakterbogen vermerkt.}}

\newglossaryentry{hunger und durst}
{
    name={Hunger und Durst},
    description={Hunger und Durst Nahrungs- oder Wassermangel können innerhalb einiger Tage zum Tod führen. Diese Zeitspannen gelten, wenn dein Charakter überhaupt kein Wasser oder keine Nahrung zur Verfügung hat. Muss er mit der halben benötigten Menge auskommen, verdreifachen sich Verzögerung und Intervall. Die Verzögerung beginnt von vorne, wenn dein Charakter mindestens eine Woche ausreichend Nahrung bzw. Wasser zu sich nimmt.}}

\newglossaryentry{stürze}
{
    name={Stürze},
    description={Stürze richten 1W6 SP pro Schritt Höhe an; bei besonders harten oder weichen Böden 1W6+1 bzw. 1W6–1 SP. Du kannst eine \gls{akrobatik}-Probe (12+Schritt Höhe) ablegen, um die effektive Höhe zu halbieren. Bei einem freiwilligen Sprung senkt eine gelungene Akrobatik-Probe die effektive Höhe sogar auf ein Viertel des Ursprungswertes.}}

\newglossaryentry{hitze und kälte}
{
    name={Hitze und Kälte},
    description={Hitze und Kälte sind unbarmherzige Feinde, die selbst zähe Charaktere in die Knie zwingen können. Die Verzögerung entspricht dabei immer dem Intervall. Schutz gegen Hitze bietet nur eine Rast im Schatten, kühler Wind, oder, nach Spielleiterentscheid, ein mindestens doppelter Wasserverbrauch. All diese Maßnahmen senken die Temperaturstufe um je –1. Warme, trockene Kleidung schützt gegen Kälte und erhöht die Temperaturstufe um +1 (dicke Winterkleidung) bis +2 (fellgefütterte Kleidung). Die Behinderung solcher Kleidung ist genau so hoch wie ihr Kälteschutz. Außerdem erhöht ein Lagerfeuer die Temperatur um +1 Stufe, während sie durch Wind um –1 oder sogar –2 Stufen gesenkt wird.}}

\newglossaryentry{gifte}
{
    name={Gifte},
    description={Wir unterscheiden Gifte nach der Art der Verabreichung in Einnahme- (E), Kontakt- (K) und Waffengifte (W). Letztere wirken, sobald die TP der vergifteten Waffe den Rüstungsschutz übersteigen und sind nur für 1W6 Treffer einsetzbar. Die Stufe eines Giftes gibt seine Stärke an. Beachte, dass hier die Mindeststufe eines Giftes angeführt ist und ein kompetenter Alchemist stärkere Gifte anfertigen kann (die Giftstufe steigt pro Stufe Hohe Qualität um +4).\\
Der Vorteil Resistenz gegen Gifte, eine gelungene erste Hilfe oder eine \gls{ko}-Probe (Giftstufe, I) können die Wirkung eines Giftes mildern: Das Intervall eines Giftes verdoppelt sich bei gleicher Wirkungsdauer, Zusatzwirkungen wie Erschwernisse durch eine Lähmung werden halbiert. Profitierst du sogar von zwei mildernden Effekten (wie eine gelungene KO-Probe und eine gelungene erste Hilfe), wird das Gift aufgehoben. Die mehrfache Vergiftung mit demselben Gift erhöht die Wirkung nur dann, wenn dies ausdrücklich angegeben ist. Normalerweise gilt schlicht die längere Wirkungsdauer sowie die höhere Giftstufe.}}

\newglossaryentry{krankheiten}
{
    name={Krankheiten},
    description={Auch wenn Krankheiten und Seuchen in Aventurien häufig vorkommen, ist ein kranker und damit handlungsunfähiger Charakter eine sehr frustrierende Erfahrung. Wir empfehlen deswegen, Krankheiten nur zu verwenden, wenn sie zentral für ein Abenteuer sind. Krankheiten verwenden dieselben Regeln wie \gls{gifte}, allerdings muss sich der Kranke meist schonen, sonst kann der Spielleiter die Auswirkungen der Krankheit deutlich erhöhen.}}
    
        
        
        
        