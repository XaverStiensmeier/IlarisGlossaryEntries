%Vorteile
\newglossaryentry{vorteil}
{
    name={Vorteil},
    description={Vorteile sind besondere Fähigkeiten, die deinem Charakter im Spiel nützen können. Sie können entweder schon bei der Generierung oder bei späteren Steigerungen gekauft werden, solange du die Voraussetzungen erfüllst (oft ein Attribut in einer bestimmten Höhe). Deswegen ist es auch kein Problem, wenn du zu Beginn noch nicht alle Vorteile kennst. Zu dieser Regel gibt es allerdings auch einige wenige Ausnahmen unter den allgemeinen Vorteilen. Die Gabe der Magie etwa ist angeboren und kann nicht einfach erlernt werden. Genausowenig kann der Bettler durch den Aufwand von EP in den Adelsstand aufsteigen. Unter dem Punkt Nachkauf findest du deswegen bei jedem allgemeinen Vorteil einen Hinweis, ob und unter welchen Bedingungen der Vorteil im späteren Spiel erworben werden kann. Ist der Nachkauf häufig oder sogar üblich, sollte dir der Spielleiter keine zu großen Steine in den Weg legen. Extrem selten bedeutet hingegen, dass der Nachkauf Dämonenpaktierern, Gezeichneten oder anderen herausragenden Gestalten vorbehalten ist. Wenn du einen besonders exklusiven Vorteil erwirbst, kann der Spielleiter verlangen, dass du eine Lehrmeisterin aufsuchst. Dieser kann dich einer Prüfung unterziehen oder Gegengefallen verlangen, sodass der Kauf des Vorteils zu einem besonderen Moment wird oder sogar direkt ins nächste Abenteuer führt. Vorteile werden unterteilt in: Profane Vorteile, Kampfvorteile, Kampfstile, Magische Vorteile, Magische Traditionen, Karmale Vorteile und Karmale Traditionen}}

\newglossaryentry{achaz}
{
    name={Achaz},
    description={Du kannst mit deinem Schwanz waffenlose Angriffe in \gls{reichweite} 1 (siehe auch S. 38) durchführen. Allerdings leidest du unter Kältestarre; für jede \gls{temperaturstufe} unter normal sind alle körperlichen Proben um \glslink{erschwernis}{–4} erschwert. Voraussetzungen: üblicherweise nur von Achaz wählbar; 0 EP. Nachkauf: extrem selten}}

\newglossaryentry{angepasst}
{
    name={Angepasst I/II (Umgebung)},
    description={Durch deine Spezies oder langjährige Erfahrung hast du dich an eine bestimmte Umgebung oder Umweltbedingung gewöhnt. Abzüge durch diese Umgebung, insbesondere im Kampf, sinken für dich um eine/zwei Stufen. Die Kosten für Angepasst legt der Spielleiter fest, wobei er sich an der Häufigkeit der Umgebung orientieren sollte. Zu allgemein gefasste Umgebungen wie \glqq unsicherer Untergrund\grqq{} sollte er nicht zulassen. Beispiele für Angepasst sind:
\begin{description}
\item Dunkelheit: verringert Abzüge durch schlechte Lichtverhältnisse (40 EP pro Stufe)
\item Schnee: verringert Abzüge durch schneebedeckten oder eisigen Untergrund (20 EP pro Stufe)
\item Wasser: verringert Abzüge durch knie- oder hüfttiefes Wasser und unter Wasser (20 EP pro Stufe)
\item Wald: verringert Abzüge durch Wurzeln, Gestrüpp und dichtes Unterholz (40EP pro Stufe)
\end{description}
Voraussetzungen: keine/Angepasst I. Nachkauf: häufig/selten.}}

\newglossaryentry{besonderer besitz}
{
    name={Besonderer Besitz (Gegenstand)},
    description={Ein besonderer Gegenstand wie die 33-fach gefaltete Klinge deines ruhmreichen Großvaters oder ein heilendes Diadem befindet sich in deinem Besitz. Der Spielleiter sollte diesem Gegenstand einen gewissen Schutz gewähren; er wird nicht zufällig von einem Taschendieb gestohlen oder geht durch Pech verloren, ohne dass du ihn wieder zurückerlangen kannst. Umgekehrt darf der Charakter den Gegenstand auch nicht verkaufen. Die Kosten für einen Besonderen Besitz legt der Spielleiter fest, wobei dieser sich am Spielnutzen des Gegenstandes orientieren sollte. Ein Besonderer Besitz für 40 EP bringt einen willkommenen Bonus in manchen Situationen, während ein Besonderer Besitz für 160 EP in vielen Situationen entscheidend ist. Der Wert des Gegenstandes spielt hierbei keine Rolle. Beispiele für besonderen Besitz sind:
\begin{description}
\item Ein Schlachtross, eine verbesserte Waffe (+2 TP) oder verbesserte Rüstung (–1 BE) oder eine Ferrara-Kutsche mit Pferden (40 EP)
\item Eine verbesserte und magische Waffe oder ein aufladbares Schutzartefakt mit einfachem Auslöser (80 EP)
\item Ein verbessertes Enduriumschwert, ein verbesserter Toschkrilpanzer oder ein Reithippogriff (120 EP)
\item Ein bewaffnetes Handelsschiff oder ein semipermanentes Universalschutzamulett mit intelligentem Auslöser (160 EP)
\end{description}
Nachkauf: nicht möglich}}

\newglossaryentry{einkommen}
{
    name={Einkommen I/II/III/IV},
    description={Deine Familie, Organisation oder eine persönliche Mäzenin stellt dir ein monatliches Einkommen von 4/16/64/256 Dukaten zur Verfügung, die du beispielsweise über eine Filiale der Nordlandbank oder ein Ordenshaus beziehen kannst. Fern der Zivilisation kannst du nicht auf dein Einkommen zugreifen. Das Einkommen reicht aus, um die Lebenshaltungskosten eines Angehörigen der Unterschicht/Mittelschicht/Oberschicht/Elite zu decken. Voraussetzungen: keine/Einkommen I/Einkommen II/ Einkommen III; 20 EP pro Stufe. Nachkauf: häufig/häufig/häufig/häufig. Anmerkung: Wenn Geld in eurem Spiel nicht so wichtig sein soll, könnt ihr den Vorteil Einkommen kostenlos an jeden vergeben.}}

\newglossaryentry{eisenaffine aura}
{
    name={Eisenaffine Aura},
    description={Malusse durch den Bann des Eisens sinken um 8 Punkte. Voraussetzungen: Zauberer I, 40 EP. Nachkauf: selten.}}

\newglossaryentry{gefahreninstinkt}
{
    name={Gefahreninstinkt},
    description={Mit der Gabe Gefahreninstinkt kannst du mit dem Talent Wachsamkeit auch Gefahren wahrnehmen, die mit gewöhnlichen Sinnen nicht wahrzunehmen sind. Dazu gehören etwa magische Fallen oder eine bald losbrechende Lawine. Ist die Gefahr auch mit gewöhnlichen Sinnen wahrnehmbar erhältst du eine Erleichterung von +4. Voraussetzungen: 100 EP. Nachkauf: extrem selten.}}

\newglossaryentry{geweiht}
{
    name={Geweiht I/II/III/IV},
    description={Du verfügst über 8/16/24/32 Karmapunkte und kannst karmale Traditionen erlernen. Voraussetzungen: keine/Geweiht I/Geweiht II/Geweiht III; 40 EP pro Stufe. Nachkauf: üblich/üblich/üblich/üblich.}}

\newglossaryentry{glück}
{
    name={Glück I/II},
    description={Deine maximalen Schicksalspunkte steigen auf 5/6. Voraussetzungen: keine/Glück I; jede Stufe 40 EP. Nachkauf: häufig/häufig.}}

\newglossaryentry{kreis der verdammnis}
{
    name={Kreis der Verdammnis I-VII},
    description={Der Vorteil hat vier Auswirkungen. Erstens erhältst du beim Erwerb der ersten Stufe 400 EP, für jede weitere Stufe 200 EP. Zweitens werden deine Gunstpunkte (S. 93) sofort aufgefüllt. Drittens erhältst du für jede Stufe eine verdorbene Eigenheit. Viertens entwickelst du ein Dämonenmal und kannst über die Liturgie Seelenprüfung als Paktierer erkannt werden. Auf geweihtem/heiligen Boden musst du jede Minute eine Willensstärke-Probe (20/28) ablegen, um nicht durch starke Schmerzen aufzufallen und eine Wunde zu erleiden. Voraussetzungen: 0 EP pro Stufe. Nachkauf: für alle Stufen üblich.}}

\newglossaryentry{magieabweisend}
{
    name={Magieabweisend},
    description={Zauber wirken auf dich deutlich schwächer. Du ignorierst bei allen Zaubern eine Stufe der spontanen Modifikation Mächtige Magie. Zauber ohne Mächtige Magie haben auf dich keine Wirkung. Voraussetzung: 40 EP. Nachkauf: extrem selten.}}

\newglossaryentry{magiegespür}
{
    name={Magiegespür},
    description={In der Nähe astraler Kräfte überfällt dich ein Frösteln, du hörst sphärische Klänge oder ein Farbschleier legt sich für dich über die Umgebung. Mit dem \gls{talent} \glslink{wahrnehmung}{sinnenschärfe} kannst du \glslink{intensitätsanalyse}{Intensitätsanalysen} von magischen Gegenständen durchführen. Nach dem aktiven Einsatz der Gabe erleidest du einen Punkt Erschöpfung. Voraussetzungen: 60 EP. Nachkauf: extrem selten.}}

\newglossaryentry{natürliche rüstung}
{
    name={Natürliche Rüstung},
    description={Du verfügst über dichtes Fell oder zähe Schuppenhaut, wo­durch dein RS um 1 steigt. Die BE verändert sich dadurch nicht. Voraussetzungen: üblicherweise nur von Achaz, Orks oder ähnlichen Spezies wählbar; 80 EP. Nachkauf: extrem selten.}}

\newglossaryentry{paktierer}
{
    name={Paktierer I/II/III/IV},
    description={Du verfügst über 8/16/24/32 Gunstpunkte und kannst die dämonische Tradition deines Erzdämons erlernen. Voraussetzungen: Kreis der Verdammnis I/Paktierer I/Paktierer II/Paktierer III; 40 EP pro Stufe. Nachkauf: üblich/üblich/üblich/üblich.}}

\newglossaryentry{privilegien}
{
    name={Privilegien (Privilegierte Gruppe)},
    description={Unzählige Privilegien können das Leben ihrer Träger erleichtern, weswegen wir auch nicht jedes einzelne Privileg in Werte kleiden möchten. Die Kosten eines Privilegs legt der Spielleiter fest, wobei er sich an der Spielrelevanz und dem Geltungsbereich der Privilegien orientieren sollte. Zum Beispiel bringt ein eigenes Wappen keine spielrelevanten Vorteile – ganz im Gegensatz zu der Erlaubnis, in Städten Waffen zu tragen. Doch auch diese Erlaubnis verliert an Wert, wenn sie nur in Albenhus oder am Markttag gilt. Beispiele für Privilegien sind:
\begin{description}
\item Adel: Du darfst ein \glqq von\grqq{} im Namen tragen, an Turnieren teilnehmen und jederzeit angemessen bewaffnet und gerüstet sein. Vor Gericht kannst du nur von höheren Adligen gerichtet werden und du darfst bei der Abwesenheit eines Richters über Leibeigene richten, solange dabei kein Blut fließt. Diese Privilegien gelten in ähnlicher Art und Weise im Mittel- und Horasreich, dem Bornland und Aranien (40 EP).
\item Krieger/Schwertgeselle/Rondrageweihter: Du darfst an Turnieren teilnehmen und in Städten Waffen tragen. Diese Privilegien gelten im Mittel- und Horasreich, sowie eingeschränkt in fast ganz Aventurien (20 EP).
\item Gildenmagier: Du darfst Geld für magische Dienstleistungen verlangen und hast erleichterten Zugang zu Bibliotheken und Lehrmeistern der eigenen Gilde. Von weltlichen Gerichten kannst du nur in deiner Anwesenheit verurteilt werden oder du wirst vor ein Gildengericht gestellt. Im Gegenzug musst du dich den Bedingungen des Codex Albyricus unterwerfen, also stets als Gildenmagier erkenntlich sein. Auch die meisten Waffen und fast alle Rüstungen mit einem RS von mehr als 1 sind verboten. Diese Privilegien gelten im Mittel- und Horasreich, dem Bornland, im Kalifat und in Aranien, sowie eingeschränkt in den tulamidischen Stadtstaaten und in Südaventurien (benötigt Zauberer, Gildenmagische Rep. I, 20 EP).
\end{description}
Nachkauf: selten.}}

\newglossaryentry{prophezeien}
{
    name={Prophezeien},
    description={Du kannst das Talent Willenskraft nutzen, um mit Spielkarten, Würfeln, Astrologie, Drogen oder prophetischen Träumen einen vagen und meist mehrdeutigen Blick in die Zukunft zu werfen. Nach dem Einsatz der Gabe erleidest du einen Punkt Erschöpfung. Voraussetzungen: 40 EP. Nachkauf: selten.}}

\newglossaryentry{resistenz/immunität gegen gifte}
{
    name={Resistenz/Immunität gegen Gifte},
    description={Resistenz mildert die Auswirkungen von Giften, mit Immunität sind Gifte gegen dich wirkungslos. Voraussetzungen: keine/Resistenz gegen Gifte; 40 EP pro Stufe. Nachkauf: selten/extrem selten.}}

\newglossaryentry{resistenz/immunität gegen krankheiten}
{
    name={Resistenz/Immunität gegen Krankheiten},
    description={Resistenz mildert die Auswirkungen von Krankheiten, mit Immunität erkrankst du niemals. Voraussetzungen: keine/Resistenz gegen Krankheiten; 20 EP pro Stufe. Nachkauf: selten/extrem selten.}}

\newglossaryentry{resistenz gegen hitze/kälte}
{
    name={Resistenz gegen Hitze/Kälte},
    description={Resistenz gegen Hitze/Kälte Verschiebt die Temperaturstufe bei hohen/niedrigen Temperaturen um eine Stufe in Richtung normal. Voraussetzungen: jeder Vorteil 40 EP. Nachkauf: selten.}}

\newglossaryentry{tierempathie}
{
    name={Tierempathie},
    description={Mit der Gabe Tierempathie kannst du ein passendes Talent der Fertigkeit Überleben verwenden, um die Gedanken von Tieren zu verstehen. Zusätzlich kannst du ihnen sogar einfache Botschaften zukommen lassen. Nach dem aktiven Einsatz der Gabe erleidest du einen Punkt Erschöpfung. Voraussetzungen: 60 EP für alle Tiere, 20-40 EP für eine bestimmte Gruppe von Tieren. Nachkauf: extrem selten.}}

\newglossaryentry{verbindungen}
{
    name={Verbindungen (Organisation/Person)},
    description={Du hast einen guten Draht zu einer bestimmten Organisation oder Person, die dir üblicherweise freundlich gegenübersteht. Außerdem gelten passende Verbindungen als Werkzeuge für Recherchen und viele andere Tätigkeiten. Verbindungen können mehrmals gewählt werden, um gute Kontakte zu mehreren Gruppen darzustellen. Die Kosten einer Verbindung legt der Spielleiter fest, wobei er sich an der Macht und dem Einflussbereich der Verbindung orientieren sollte. Beispiele für Verbindungen sind:
\begin{description}
\item Örtlicher Baron: Starker Einfluss in einer kleinen Region (20 EP)
\item Badilakaner: Geringer Einfluss in weiten Teilen des zwölfgöttergläubigen Aventuriens (20 EP)
\item Efferdbrüder: Ansehnlicher Einfluss in Hafenstädten des zwölfgöttergläubigen Aventuriens (40 EP)
\item Haus Gareth: Immenser Einfluss im Mittelreich, weitreichende Beziehungen in ganz Aventurien (80 EP)
\end{description}
Nachkauf: üblich bis selten.
}}

\newglossaryentry{zauberer}
{
    name={Zauberer I/II/III/IV},
    description={Du verfügst über 8/16/24/32 Astralpunkte und kannst magische Traditionen erlernen. Dämonen, Elementare oder Hellsichtsmagier können dich magisch wahrnehmen. Zauberer I entspricht dabei einem einfachen Magiedilettanten, Zauberer II einer wenig begabten Alchemistin. Die meisten ausgebildeten Zauberer verfügen dagegen über Zauberer III oder IV. Voraussetzungen: keine/Zauberer I/Zauberer II/Zauberer III; 40 EP pro Stufe. Nachkauf: extrem selten/üblich/üblich/üblich.}}

\newglossaryentry{zwergennase}
{
    name={Zwergennase},
    description={Mit Zwergennase besitzt du einen übernatürlich guten Riecher für Verstecke, Geheimgänge und mechanische Fallen. Du kannst sie mit dem Talent Wachsamkeit wahrnehmen, auch wenn das mit gewöhnlichen Sinnen unmöglich wäre. Ist das Objekt auch mit gewöhnlichen Sinnen wahrnehmbar, erhältst du eine Erleichterung von +4. Voraussetzungen: 60 EP. Nachkauf: extrem selten.}}
        
    