\newglossaryentry{rüstung}
{
    name={Rüstung},
description={Der Rüstungsschutz erhöht die \gls{ws*} entscheidend, wodurch dein Charakter seltener Wunden und Wundschmerz erleidet. Dafür schränkt Rüstung die Beweglichkeit ein, was durch die \gls{be} dargestellt wird. Normalerweise ist die Behinderung gleich hoch wie der Rüstungsschutz. Jeder Punkt Behinderung erschwert Kampfwürfe um –1, senkt die \gls{gs} um –1 (bis minimal 1) und das Durchhaltevermögen um –2 (bis minimal 1, S. 34). Proben auf die Fertigkeit \gls{athletik} und \gls{ge} sind nach Spielleiterentscheid um –2 (Reiten in schwierigem Gelände) bis –16 (Schwimmen in schwerer Rüstung) erschwert. Die Behinderung kann durch die Vorteile \gls{rüstungsgewöhnung} und besonders hochwertige Herstellung gesenkt werden. Eine unpassende Rüstung erhöht die Behinderung um +1.}}

\newglossaryentry{trefferzonen}
{
    name={Optional: Trefferzonen},
    description={Diese Optionalregeln eignen sich für deine Spielrunde, wenn ihr Rüstungen und Wundschmerz detaillierter darstellen möchtet. Mit ihnen hat dein Charakter 6 Trefferzonen: \gls{kopf}, \gls{brust}, \gls{bauch}, \gls{schwertarm}, \gls{schildarm} und \gls{beine}. Die \gls{rüstung} wird weiter individualisiert, indem du 6 x RS Punkte auf die 6 Zonen verteilst. Dieser Zonen-Rüstungsschutz (\gls{zrs}) ergeben mit der \gls{ws} die 6 Zonen-Wundschwellen (\gls{zws}*). Selbstverständlich sollte diese Aufteilung dem gesunden Menschenverstand folgen, also ein Helm nicht den Bauch schützen.\\
Mit dem Manöver \gls{gezielter schlag} kannst du eine Trefferzone aussuchen, sonst wird die Trefferzone mit 1W6 zufällig bestimmt. Die TP des Angriffes werden dann mit der jeweiligen ZWS* verglichen. Die WS* verwendest du nur noch für Flächenschaden, zum Beispiel durch den Feuerodem eines Drachen. Die Regeln für Wundabzüge ändern sich mit den Trefferzonen nicht. Die Auswirkungen des Wundschmerzes unterscheiden sich aber je nach Trefferzone: Ein heftiger Kopftreffer hat wie in den Basisregeln eine betäubende Wirkung, während dich ein Treffer an den Beinen auf die Bretter schickt. Die Probe wird je nach Zone auf ein unterschiedliches Attribut abgelegt, die Schwierigkeit ist gleich der der Basisregeln (20, I).}}

\newglossaryentry{kopf}
{
    name={Trefferzone: Kopf},
    description={Bei einer 6: \gls{mu}(20, I). Betäubt: entspricht dem \gls{wundschmerz} in Basisregeln.}}

\newglossaryentry{brust}
{
    name={Trefferzone: Brust},
    description={Bei einer 5: \gls{ko}(20, I). Organtreffer: Das Opfer erleidet eine Zusatzwunde.}}

\newglossaryentry{bauch}
{
    name={Trefferzone: Bauch},
    description={Bei einer 4: \gls{ko}(20, I). Organtreffer: Das Opfer erleidet eine Zusatzwunde.}}

\newglossaryentry{schwertarm}
{
    name={Trefferzone: Schwertarm},
    description={Bei einer 3: \gls{kk}(20, I). Entwaffnet: Die mit dieser Hand geführte Waffe fällt zu Boden.}}

\newglossaryentry{schildarm}
{
    name={Trefferzone: Schildarm},
    description={Bei einer 2: \gls{kk}(20, I). Entwaffnet: Die mit dieser Hand geführte Waffe fällt zu Boden.}}

\newglossaryentry{beine}
{
    name={Trefferzone: Beine},
    description={Bei einer 1: \gls{ge}(20, I). Sturz: Das Opfer stürzt und liegt am Boden.}}