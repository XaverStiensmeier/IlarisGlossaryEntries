\newglossaryentry{initiativephase}
{
    name=Initiativephase,
    description={In deiner Initiativephase kannst du handeln. Eine Minute entspricht 16 Initiativephasen einer Person.}
}

\newglossaryentry{schwierigkeit}
{
	name={Schwierigkeit},
	description={Wenn dein \gls{erfolgswert} größer gleich der Schwierigkeit ist, gelingt die Probe.}
}

\newglossaryentry{attributprobe}
{
    name={Attributsproben},
    description={Der \gls{probenwert} beträgt das doppelte des \glslink{attribut}{Attributs}.}
}
\newglossaryentry{fertigkeitsprobe}
{
    name={Fertigkeitsprobe},
    description={Der \gls{probenwert} beträgt die Summe aus \gls{basiswert} und \gls{fertigkeitswert}.}
}
\newglossaryentry{basiswert}
{
    name={Basiswert},
    description={Der Basiswert ist der Durchschnitt der drei \glslink{attribut}{Attribute}, die bei jeder \gls{fertigkeit} angegeben sind.}}

\newglossaryentry{erschwernis}
{
    name={Erschwernis},
    description={Verringert den \gls{erfolgswert} (bspw. -2).}}
    
\newglossaryentry{erleichterung}
{
    name={Erleichterung},
    description={Erhöht den \gls{erfolgswert} (bspw. +2).}}
\newglossaryentry{vier stufen}
{
    name={Vier Stufen},
    description={\glslink{erschwernis}{Erschwernisse} bzw. \glslink{erleichterung}{Erleichterungen} betragen immer 2/4/8/16.}}
\newglossaryentry{freiwillig erschweren}
{
    name={Freiwillig erschweren},
    description={Das freiwillige Erschweren gibt dir verschiedene Boni (siehe z.B. \gls{mächtige magie}).}}
\newglossaryentry{hohe qualität}
{
    name={Hohe Qualität},
    description={Mit Hohe Qualität kannst du schärfere Schwerter schmieden, tödlichere Gifte brauen oder deine Gegner stärker \gls{einschüchtern}. Jede Verbesserung erschwert deine Probe um –4. Die genauen Auswirkungen von Hohe Qualität findest du bei der jeweiligen Handlung.}}
\newglossaryentry{zufällig}
{
    name={Zufällig},
    description={Proben, in denen Glück eine große Rolle spielt, werden mit \gls{1w20} statt mit \gls{3w20} abgelegt.}}
\newglossaryentry{umgebung}
{
    name={Umgebung},
    description={Proben, in denen die Umgebung eine große Rolle spielt, werden mit \gls{1w20} statt mit \gls{3w20} abgelegt.}}
\newglossaryentry{triumph}
{
    name={Triumph},
    description={Wenn deine Probe gelingt und der gewertete Würfel eine 20 zeigt, hast du einen Triumph erzielt.Und deinem Charakter gelingt ein außergewöhnlicher, ja herausragender, Erfolg.}}
\newglossaryentry{patzer}
{
    name={Patzer},
    description={Wenn deine Probe misslingt und der Würfel eine 1 zeigt, bedeutet das einen Patzer – dein Charakter hat sich in eine unangenehme oder sogar lebensbedrohliche Situation gebracht.}}
\newglossaryentry{zusammenarbeit}
{
    name={Zusammenarbeit},
    description={Bei manchen Aufgaben ist es sinnvoll, wenn mehrere Charaktere zusammenarbeiten. Der Spielleiter entscheidet,
wie viele Charaktere helfend eingreifen können und wie groß ihr \gls{einfluss} ist. Helfer können eine um 4 Punkte leichtere Probe ablegen, um die entscheidende Probe um +2 (kleiner Einfluss), +4 (großer Einfluss) oder selten sogar +8 (enormer Einfluss) pro Helfer zu erleichtern. Misslingt der Helferin die Probe, ist die entscheidende Probe aber um –2 erschwert.}}
\newglossaryentry{einfluss}
{
    name={Einfluss},
    description={Einfluss kann klein (+2), groß (+4) oder selten sogar enorm (+8) sein. Einfluss ist besonders für \gls{zusammenarbeit} relevant.}}
\newglossaryentry{gruppenprobe}
{
    name={Gruppenprobe},
    description={Dabei würfelt ein Spieler eine Probe, deren Ergebnis für die ganze Gruppe bindend ist.}}

\newglossaryentry{offene probe}
{
    name={Offene Probe},
    description={Dabei würfelt der Spieler gegen eine vom Spielleiter offen bestimmte \gls{schwierigkeit}.}}

\newglossaryentry{vergleichende probe}
{
    name={Vergleichende Probe},
    description={Vergleichende Proben sind ein zentraler Bestandteil von Ilaris und kommen in jedem Konflikt vor, egal ob dieser mit Waffen, Zauberformeln oder Worten ausgetragen wird. Beide Beteiligten würfeln eine Probe auf die geforderte Fertigkeit und der höhere Erfolgswert setzt sich durch. Bei einem Gleichstand gewinnt der Beteiligte mit dem höheren Probenwert, bei gleichem Probenwert gewinnt der Spieler. Vergleichende Proben können auf drei Arten abgelegt werden: \gls{beide seiten würfeln}, \gls{die aktive seite würfelt} und \gls{der spieler würfelt aktiv}.}}
        
\newglossaryentry{ausgedehnte probe}
{
    name={Ausgedehnte Probe},
    description={In einer ausgedehnten Probe bestimmt der Spielleiter den \gls{dg}. Um erfolgreich zu sein, müssen dir insgesamt so viele Einzelproben gelingen, wie der \gls{dg} beträgt – und zwar, bevor dir dieselbe Anzahl an Einzelproben
misslungen ist. (Eine gewöhnliche Probe entspricht also einem \gls{dg} von 1)}
}

\newglossaryentry{detailgrad}
{
    name={Detailgrad},
    description={Bestimmt wie viele Proben bei einer \glslink{ausgedehnte probe}{ausgedehnten Probe} gewürfelt werden. Normalerweise liegt der \gls{dg} bei 2 bis 4.}}

\newglossaryentry{eigenheit}
{
    name={Eigenheit},
    description={Eigenheiten beschreiben die Stärken und Schwächen deines Charakters, seine Herkunft, Ausbildung und Weltanschauung. Außerdem lassen sie sich mit \glslink{schip}{Schips} \glslink{eigenheiten einsetzen}{einsetzen} oder \glslink{eigenheiten ausnutzen}{ausnutzen}, um \glslink{schip}{Schips} zu generieren.}}

\newglossaryentry{status}
{
    name={Status},
    description={Der Status bestimmt, ob dein Charakter in seinem bisherigen Leben Wachteleier gespeist hat oder gerade der menschen­verachtenden Sklavenarbeit in einer Mine entkommen ist: Elite, Oberschicht, Mittelschicht,Unterschicht und Abschaum. Außerdem bestimmt dein Status deine \gls{lebenserhaltungskosten}.}}

\newglossaryentry{erfahrungspunkte}
{
    name={Erfahrungspunkte},
    description={Erfahrungspunkte sind die Währung, mit der du die Werte deines Charakters weiterentwickelst, sodass dein Charakter immer größere Herausforderungen bestehen kann.}}

\newglossaryentry{attributsteigerungskosten}
{
    name={Attributsteigerungskosten},
    description={\glslink{attribut}{Attribute} werden Punkt für Punkt gesteigert. Jeder Punkt kostet Zielwert x 16 \gls{ep}.}}

\newglossaryentry{fertigkeitswertsteigerungskosten}
{
    name={Fertigkeitswertsteigerungskosten},
    description={Fertigkeits­werte werden Punkt für Punkt gesteigert. Jeder Punkt kostet Zielwert x Steigerungsfaktor \gls{ep}. Der Steigerungsfaktor hängt von der \gls{fertigkeit} ab und liegt zwischen 2 und 4.}}

\newglossaryentry{talent}
{
    name={Talent},
    description={Jeder \gls{fertigkeit} sind mehrere Talente zugeordnet, die für einen Teilbereich dieser \gls{fertigkeit} stehen. Ohne das passende Talent darfst du zwar Proben ablegen, dabei aber nur den halben \gls{fertigkeitswert} einsetzen.}}

\newglossaryentry{talentkosten}
{
    name={Talentkosten},
    description={Ein \gls{talent} kostet dich 20 x \gls{steigerungsfaktor} \gls{ep}. Selten anwendbare \glslink{talent}{Talente} sind verbilligt und kosten nur die Hälfte. Da sich jeder Charakter mit den Gebräuchen seiner eigenen Kultur auskennt erhält er das entsprechende \gls{talent} gratis.}}

\newglossaryentry{freie fertigkeit}
{
    name={Freie Fertigkeit},
    description={Freie Fertigkeiten sind verschiedene seltene Handwerks­künste, aber auch Sprachen und Schriften. Freie Fertigkeiten sind in drei Stufen geteilt. Mit der ersten Stufe bist du unerfahren in dieser Freien Fertigkeit, mit der zweiten erfahren und mit der dritten ein Meister dieses Faches. Du kannst diese Stufen I/II/III für nur 4/8/16 EP erwerben, wobei die vorhergehende Stufe vorausgesetzt wird. Selbstverständlich spricht jeder Charakter seine eigene Muttersprache.}}

\newglossaryentry{energie}
{
    name={Energie},
    description={Über \glslink{vorteil}{Vorteile} werden übernatürliche Energien zur Verfügung gestellt. Beispielsweise müssen Zauberer den \gls{vorteil} \gls{zauberer} kaufen, der ihnen \gls{asp} verleiht.}}

\newglossaryentry{traditionsvorteil}
{
    name={Traditionsvorteil},
    description={Die Tradition bestimmt, welche \glslink{übernatürliches talent}{Zauber/Liturgien} du sprechen kannst und verstärkt alle mit ihr eingesetzten \glslink{übernatürliches talent}{Zauber/Liturgien}. Jede Stufe setzt die vorherigen Stufen voraus.}}

\newglossaryentry{übernatürliche fertigkeit}
{
    name={Übernatürliche Fertigkeit},
    description={Kategorie des übernatürlichen Wirkens (bspw. Einfluss, Antimagie oder Seefahrt, Heiliges Handwerk).}}

\newglossaryentry{übernatürliches talent}
{
    name={Übernatürliches Talent},
    description={Sammelbegriff für Liturgien, Zauber und Anrufungen (bspw. Böser Blick, Herr über das Tierreich oder Segensreiches Wasser, Handwerkssegen).}}

\newglossaryentry{energiesteigerungskosten}
{
    name={Energiesteigerungskosten},
    description={Jeder Punkt kostet Zielwert \gls{ep}.}}

\newglossaryentry{schicksalspunkt}
{
    name={Schicksalspunkt},
    description={Schicksalspunkte bieten verschiedene Boni: \gls{glückliche fügung}, \gls{nur ein kratzer} und \gls{von der schippe springen}. Außerdem wird ihr Einsatz zusammen mit passenden \glslink{eigenheit}{Eigenheiten} noch wirkungsvoller.}}

\newglossaryentry{glückliche fügung}
{
    name={Glückliche Fügung},
    description={Du kannst einen \gls{schip} einsetzen, der dir einen zusätzlichen Würfel für die nächste Probe verleiht. Statt mit drei Würfeln würfelst du mit vier und der zweithöchste Wert zählt. Statt mit einem Würfel würfelst du mit zwei und der höchste Wert zählt.}}


\newglossaryentry{nur ein kratzer}
{
    name={Nur ein Kratzer},
    description={Unmittelbar nachdem du Wunden erlitten hast, kannst du einen \gls{schip} einsetzen. Dadurch werden die gerade erlittenen Wunden halbiert.}}


\newglossaryentry{von der schippe springen}
{
    name={Von der Schippe springen},
    description={Du kannst zwei \glslink{schip}{Schips} aufwenden, um den Tod deines Charakters abzuwenden. Dazu musst du deinem Spielleiter einen plausiblen Vorschlag machen, wie dein Charakter die tödliche Situation überlebt. Von der Schippe springen lässt deinen Charakter übrigens nicht ungeschoren davonkommen. Eher überlebt er einen eigentlich tödlichen Sturz schwer verletzt oder wird von den Feinden für tot gehalten und ausgeplündert.}}
                

\newglossaryentry{startschicksalspunkt}
{
    name={Startschicksalspunkt},
    description={Du startest mit sehr sehr 2(sehr reich)/3(reich)/4(normal)/5(arm)/6(sehr arm) \glslink{schip}{Schips}. Du kannst \gls{schip}{Schips} über dem Maximum (normalerweise 4) nicht zurückgewinnen. Dies wirkt sich auch auf dein \gls{startkapital} aus.}}
        

\newglossaryentry{startkapital}
{
    name={Startkapital},
    description={Du startest mit 256(sehr reich)/128(reich)/32(normal)/16(arm)/4(sehr arm) Dukaten. Dies wirkt sich auch auf deine \glslink{startschicksalspunkt}{Startschicksalspunkte} aus.}}

\newglossaryentry{eigenheiten einsetzen}
{
    name={Eigenheiten einsetzen},
    description={Du kannst \glslink{eigenheit}{Eigenheiten} einsetzen, um deinem Charakter einen \gls{vorteil} zu verschaffen. Passt eine deiner \glslink{eigenheit}{Eigenheiten} zu deinem Vorhaben, kannst du \glslink{schip}{Schips} effektiver nutzen: Erstens erhältst du bei einer glücklichen Fügung sogar zwei zusätzliche Würfel. Zweitens kannst du für einen \gls{schip} einen Probenwurf wiederholen. Der Spielleiter hat bei all diesen Varianten das Recht auf ein Veto.}}

\newglossaryentry{eigenheiten ausnutzen}
{
    name={Eigenheiten ausnutzen},
    description={\glslink{eigenheit}{Eigenheiten} stellen außerdem die Schwächen und Marotten deines Charakters dar und können ihm das Leben schwer machen. Als Gegenleistung erhältst du einen \gls{schip}, außer du verfügst schon über die maximale Zahl von \glslink{schip}{Schips}. Dieses Ausnutzen von \glslink{eigenheit}{Eigenheiten} kann auf drei Arten geschehen: \gls{vom spielleiter ausgehend}, \gls{von dir ausgehend}, und \gls{aus dem spiel heraus}.}}

\newglossaryentry{vom spielleiter ausgehend}
{
    name={Vom Spielleiter ausgehend},
    description={Der Spielleiter bietet dir an, eine deiner \glslink{eigenheit}{Eigenheiten} auszunutzen. Wenn du das zulässt, kann er eine Handlung deines Charakters scheitern lassen oder ihn auf eine andere Art und Weise in Schwierigkeiten bringen. Dafür bekommst du einen \gls{schip}. Du kannst das Ausnutzen einer \gls{eigenheit} auch ablehnen, wenn du dafür einen deiner \glslink{schip}{Schips} abgibst. Besitzt du keinen \gls{schip}, kannst du nicht ablehnen.}}

\newglossaryentry{von dir ausgehend}
{
    name={Von dir ausgehend},
    description={Du schlägst dem Spielleiter eine Handlung vor, die deiner Meinung nach einen \gls{schip} wert ist und zu deinen \glslink{eigenheit}{Eigenheiten} passt. Nimmt der Spielleiter an, erhältst du einen \gls{schip}. Hier gibt es kein Risiko, Eigeninitiative wird also belohnt!}}

\newglossaryentry{aus dem spiel heraus}
{
    name={Aus dem Spiel heraus},
    description={Manchmal bringst du dich durch das Ausspielen deiner \glslink{eigenheit}{Eigenheiten} ganz von selbst in haarige Situationen, ohne dass irgendjemand an \glslink{schip}{Schips} gedacht hätte. In diesem Fall sollte dir der Spielleiter nachträglich einen \gls{schip} zugestehen.}}

\newglossaryentry{einsatz erhöhen}
{
    name={Einsatz erhöhen},
    description={Wenn du in einer wirklich entscheidenden Situation das \glslink{eigenheiten ausnutzen}{Ausnutzen} ablehnst, kann der Spielleiter auch den Einsatz erhöhen. Statt einem \gls{schicksalspunkt} erhältst du nun zwei, wenn du das Ausnutzen der \gls{eigenheit} zulässt.}}

\newglossaryentry{lebenserhaltungskosten}
{
    name={Lebenserhaltungskosten},
    description={Diese geben den ungefähren Betrag an, den dein Charakter pro Monat für Unterkunft, Verpflegung, Kleidung, Unterhaltung und alle übrigen Ausgaben des täglichen Lebens ausgibt, wenn er nicht auf Selbstversorgung zurückgreift oder auf fremde Kosten lebt. Ein höherer \gls{status} bedeutet höhere Lebenserhaltungskosten. Der \gls{vorteil} \gls{einkommen} kann dabei helfen auch hohe Lebenserhalungskosten zu bestreiten.}}

\newglossaryentry{attribut}
{
    name={Attribut},
    description={Die insgesamt acht Attribute (\gls{ch}, \gls{ff}, \gls{ge}, \gls{in}, \gls{kk}, \gls{kl}, \gls{ko} und \gls{mu}) stellen die geistige und körperliche Grundlage deines Charakters dar. Manchmal legst du \glslink{attributprobe}{Attributproben} ab. Darüber hinaus bestimmen sie deine \glslink{abgeleiteter wert}{abgeleiteten Werte} (s.u.), beeinflussen den \gls{bw} für \glslink{fertigkeitsprobe}{Fertigkeitsproben} und werden für viele \glslink{vorteil}{Vorteile} vorausgesetzt.}}

\newglossaryentry{charisma}
{
    name={Charisma},
    description={ist die natürliche Ausstrahlung deines Charakters auf seine Umgebung. Es steht für Führungsqualitäten, Selbstbewusstsein, Überzeugungskraft und ein gewinnendes Wesen. Alle gesellschaftlichen Fertigkeiten hängen von deinem Charisma ab und auch Elementaristen benötigen ein hohes Charisma.}}

\newglossaryentry{fingerfertigkeit}
{
    name={Fingerfertigkeit},
    description={ist die manuelle Geschicklichkeit deines Charakters. Er verfügt über eine gute Hand-Augen-Koordination und kann feine Bewegungen schnell und fehlerfrei ausführen. Handwerker, Fernkämpfer und Hersteller von Artefakten sollten deswegen nicht auf eine hohe Fingerfertigkeit verzichten.}}

\newglossaryentry{gewandtheit}
{
    name={Gewandtheit},
    description={ist die Beweglichkeit und Gelenkigkeit deines Charakters. Gewandte Charaktere bewegen sich geschmeidig und können ihre Bewegungen gut abschätzen. Das fördert deine körperlichen und kämpferischen Fertigkeiten und erhöht deine \gls{gs}.}}

\newglossaryentry{intuition}
{
    name={Intuition},
    description={ermöglicht es deinem Charakter, Personen und Sachverhalte schnell richtig einzuschätzen. Sie steht auch für Wahrnehmung und Einfühlungsvermögen deines Charakters. Intuition beeinflusst eine breite Palette an Fertigkeiten und bestimmt deine \gls{ini} im Kampf.}}

\newglossaryentry{körperkraft}
{
    name={Körperkraft},
    description={ist ein Maß für die Stärke deines Charakters. Kräftige Charaktere können schwere Lasten heben und ihre Angriffe richten durch den höheren \gls{schadensbonus} mehr Schaden an. Dadurch ist eine hohe Körperkraft gerade für kämpferische Charaktere hilfreich.}}

\newglossaryentry{klugheit}
{
    name={Klugheit},
    description={Klugheit ist das logische Denkvermögen deines Charakters und seine Fähigkeit, komplizierte Zusammenhänge zu erkennen und zu analysieren. Kluge Charaktere verfügen auch über ein gutes Allgemeinwissen. Eine hohe Klugheit ist bei vielen gesellschaftlichen und allen Wissensfertigkeiten unverzichtbar.}}

\newglossaryentry{konstitution}
{
    name={Konstitution},
    description={beschreibt die Widerstandsfähigkeit deines Charakters gegen äußere Einflüsse wie Strapazen, Gifte, Krankheiten und Verwundungen. Die Konstitution beeinflusst nur wenige \glslink{fertigkeit}{Fertigkeiten}, bestimmt aber deine \gls{wundschwelle}, den wohl wichtigsten \glslink{abgeleiteter wert}{abgeleiteten Wert}.}}

\newglossaryentry{mut}
{
    name={Mut},
    description={Mut (MU): Ein mutiger Charakter bewahrt in kritischen Situationen einen kühlen Kopf und schreckt nicht vor Gefahren zurück, was gerade im Nahkampf unerlässlich ist. Zusätzlich stärkt Mut den Widerstand gegen magische Beeinflussungen, indem er deine \gls{magieresistenz} erhöht. Auch Dämonologen benötigen einen hohen Mut.}}

\newglossaryentry{abgeleiteter wert}
{
    name={Abgeleiteter Wert},
    description={Die abgeleiteten Werte berechnen sich aus deinen \glslink{attribut}{Attributen}.
Du kannst sie nicht direkt steigern, sondern musst entweder die \glslink{attribut}{Attribute} erhöhen oder spezielle \glslink{vorteil}{Vorteile} erwerben. Die abgeleiteten Werte sind \gls{ws}, \gls{mr}, \gls{gs}, \gls{ini} und \gls{schadensbonus}.}}

\newglossaryentry{wundschwelle}
{
    name={Wundschwelle},
    description={bestimmt, wie gut dein Charakter Schaden widerstehen kann. Schaden bis zu deiner Wundschwelle ist zwar schmerzhaft, aber noch nicht wirklich gefährlich. Erst Schadensmengen über deiner Wundschwelle können deinen Charakter beeinträchtigen oder sogar töten. Die Wundschwelle beträgt 4 + 1 für je 4 volle Punkte \gls{ko}. Mit dem Vorteil \gls{unverwüstlich} kannst du die \gls{ws} um +1 erhöhen.}}

\newglossaryentry{magieresistenz}
{
    name={Magieresistenz},
    description={ist die Widerstandsfähigkeit gegen Zauberei. Viele Zauber wirken nur auf deinen Charakter, wenn sie seine Magieresistenz in einer vergleichenden Probe überwinden. Die Magieresistenz beträgt 4 + 1 für je 4 volle Punkte \gls{mu}. Die Vorteile \gls{willensstark I} und \glslink{willensstark II}{II} und \gls{unbeugsamkeit} steigern die \gls{mr} weiter.}}

\newglossaryentry{geschwindigkeit}
{
    name={Geschwindigkeit},
    description={stellt die Schnelligkeit und Beweglichkeit deines Charakters dar. Sie bestimmt, wie weit er sich im Kampf bewegen kann und hilft auch bei Verfolgungsjagden zu Fuß. Die Geschwindigkeit beträgt 4 + 1 für je 4 volle Punkte \gls{ge}. Eine noch höhere \gls{gs} verleihen ihm \gls{flink I} und \glslink{flink II}{II}.}}

\newglossaryentry{initiative}
{
    name={Initiative},
    description={steht für Reaktionsgeschwindigkeit und Übersicht im Kampf. Kämpfer mit hoher Initiative können zu Beginn eines Kampfes schneller handeln und so den Erstschlag führen. Die Initiative entspricht der \gls{in}. Der \gls{vorteil} \gls{kampfreflexe} erhöht die \gls{ini} zusätzlich.}}

\newglossaryentry{schadensbonus}
{
    name={Schadensbonus},
    description={erhöht den Waffenschaden bei allen Nahkampfangriffen. Kopflastige Waffen profitieren sogar doppelt vom Schadensbonus. Für je 4 volle Punkte \gls{kk} erhältst du +1 Schadensbonus.}}

\newglossaryentry{probenwert}
{
    name={Probenwert},
    description={Der Probenwert für Fertigkeiten besteht aus zwei Teilen, dem \gls{basiswert} und dem \gls{fertigkeitswert}.}}

\newglossaryentry{erfolgswert}
{
    name={Erfolgswert},
    description={Der Erfolgswert ist die Summe des \gls{probenwurf} und des \gls{probenwert} zuzüglich der Modifikatoren. Er wird mit der \gls{schwierigkeit} verglichen, um den Erfolg der Probe zu bestimmen.}}

\newglossaryentry{astralpunkte}
{
    name={Astralpunkte},
    description={Astralpunkte sind eine Maßeinheit, die angibt wie viel Astralkraft eine Person hat. Sie wird für Zauber ausgegeben.}}

\newglossaryentry{probenwurf}
{
    name={Probenwurf},
    description={1. Würfle mit drei zwanzigseitigen Würfeln (\gls{3w20}). Der
Würfel mit dem mittleren Ergebnis gilt. 2. Addiere den \gls{probenwert} sowie
mögliche \glslink{erschwernis}{Erschwernisse} oder \glslink{erleichterung}{Erleichterungen}. Das
Ergebnis ist der \gls{erfolgswert}. 3. Ist der \gls{erfolgswert} höher als oder gleich hoch wie die
\gls{schwierigkeit}, gelingt die Probe.}}

\newglossaryentry{fertigkeitswert}
{
    name={Fertigkeitswert},
    description={Den Fertigkeitswert kannst du im Gegensatz zum \gls{basiswert} direkt steigern, maximal bis auf das höchste am Basiswert beteiligte \gls{attribut} +2.}}

\newglossaryentry{konterprobe}
{
    name={Konterprobe(Talent oder Attribut)},
    description={Manchen Zaubern kannst du auch mit profanen \gls{fertigkeit}{Fertigkeiten} widerstehen, zum Beispiel wenn du dich mit deiner \gls{willenskraft} gegen die Wirkung des Friedenslieds stemmst. In diesem Fall kannst du eine Konterprobe ablegen. Die Auswirkungen und die \gls{schwierigkeit} einer Konterprobe findest du beim jeweiligen\gls{übernatürliches talent}{Magie/Liturgie}, die \gls{schwierigkeit} einer Konterprobe steigt aber in jedem Fall um +4 pro Mächtige \gls{übernatürliches talent}{Magie/Liturgie}. Das gilt auch, wenn \gls{mächtige magie} keine anderen Auswirkungen hat und deswegen beim \gls{zauber} gar nicht angeführt ist.}}

%XP
\newglossaryentry{steigerungsfaktor}
{
    name={Steigerungsfaktor},
    description={Der Steigerungsfaktor ist ein Faktor für die Steigerung des \gls{fertigkeitswert} (siehe \gls{fertigkeitswertsteigerungskosten}).}}

\newglossaryentry{verbilligt}
{
    name={verbilligt},
    description={Selten anwendbare Talente sind verbilligt und kosten
nur die Hälfte.}}
        

\newglossaryentry{beide seiten würfeln}
{
    name={Beide Seiten würfeln},
    description={Beide Seiten der \glslink{vergleichende probe}{Vergleichenden Probe} würfeln: Dies ist die langsamste Variante, bietet aber die größte Breite an unterschiedlichen Ergebnissen. Wir empfehlen diese Variante in Konflikten, in denen nur wenige Proben gewürfelt werden – etwa in einer \gls{verfolgungsjagd}.}}

\newglossaryentry{die aktive seite würfelt}
{
    name={Die aktive Seite würfelt},
    description={In dieser Variante wird für den passiveren Teilnehmer am Konflikt ein Würfelwurf von 10 angenommen. Das beschleunigt das Spiel, aber nimmt dir manchmal die Möglichkeit, dich aktiv zu verteidigen. Deswegen empfehlen wir diese Variante für den Spielleiter, wenn er seine Spieler nicht warnen möchte – etwa wenn sie in einen Hinterhalt laufen oder sie unbemerkt mit einem Bannbaladin belegt werden.}}

\newglossaryentry{der spieler würfelt aktiv}
{
    name={Der Spieler würfelt aktiv},
    description={Der Spieler würfelt seine Probe immer aktiv, während der Spielleiter für seine Nicht­s­­pieler­charaktere einen Wurf von 10 annimmt. Dadurch wird das Spiel beschleunigt und der Spiel­leiter entlastet. Wir empfehlen diese Variante für Szenen, in denen häufig gewürfelt wird – wie in Kämpfen.}}


% sonstiges profanes

\newglossaryentry{verfolgungsjagd}
{
    name={Verfolgungsjagd},
    description={Verfolgungsjagden beginnen, sobald die Verfolger bemerkt werden. Der Spielleiter legt den \gls{dg} fest, dann beschreibst du deine Handlungen und es werden Proben (\gls{I}) auf das \gls{talent} abgelegt, das deiner \gls{fortbewegungsart} entspricht. Dabei wird die \gls{gs} deines Fortbewegungsmittels zum \gls{pw} addiert. Gewinnt der Flüchtende den Konflikt, kann er seinen Verfolgern entkommen. Bei einem Sieg der Verfolger haben diese ihr Ziel eingeholt und befinden sich in (Nah-)Kampfreichweite.}}

\newglossaryentry{fortbewegungsart}
{
    name={Fortbewegungsart},
    description={Unter anderem für \glslink{verfolgungsjagd}{Verfolgungsjagden} relevant. Zu Fuß: Laufen, Im Wasser: Schwimmen, Auf einem Reittier: Reiten, Mit einem Schiff: \glslink{freie fertigkeit}{Freie Fertigkeiten} (z.B. Seefahrt). Natürlich sind andere Fortbewegungsarten denkbar.}}
        
    
% talente

\newglossaryentry{zähigkeit}
{
    name={Zähigkeit},
    description={}}

\newglossaryentry{willenskraft}
{
    name={Willenskraft},
    description={}}
        
        
% magie

\newglossaryentry{zauber}
{
    name={Zauber},
    description={Zauber sind eine Untergruppe der \glslink{übernatürliche fertigkeit}{Überntürlichen Fertigkeiten} (siehe \gls{zauber wirken}).}}

\newglossaryentry{zauber wirken}
{
    name={Zauber wirken},
    description={}}
        

\newglossaryentry{magie}
{
    name={Magie},
    description={Aventurien ist von Magie durchzogen, doch nur Charaktere mit dem Vorteil \gls{zauberer} können die allgegenwärtige Astralenergie aufnehmen und mit diesen \glslink{asp}{Vorrat} ihre Zauber wirken. Diese Zauber kannst du mit \glslink{spontane modifikation}{spontanen Modifikationen} verstärken, beschleunigen oder auf mehrere Ziele ausweiten. Die Unterschiede zwischen Zauberern werden durch die \gls{magische tradition} verdeutlicht – sie bestimmt die Stärken und Schwächen deiner Zauberei und welche \gls{zauber} du überhaupt erlernen kannst.}}

\newglossaryentry{spontane modifikation}
{
    name={Spontane Modifikation},
    description={Spontane Modifikationen helfen dir, deine \gls{zauber} an die aktuelle Situation anzupassen. Normalerweise sind für einen \gls{zauber} alle spontanen Modifikationen möglich – aber nicht unbedingt sinnvoll. Genauso wie \gls{manöver} sind spontane Modifikationen stets miteinander kombinierbar, außerdem können sie mehrmals eingesetzt werden. Spontane Modifikationen werden in \glslink{basismodifikation}{Basismodifikationen} und \glslink{aufbauende modifikation}{Aufbauende Modifikationen} unterteilt.}}

\newglossaryentry{aufbauende modifikation}
{
    name={Aufbauende Modifikation},
    description={}}

\newglossaryentry{basismodifikation}
{
    name={Basismodifikation},
    description={}}

\newglossaryentry{magische tradition}
{
    name={Magische Tradition},
    description={Jede Art von Zauberwirkern gibt ihre einzigartige Tradition an ihre Schüler weiter. Dieses Wissen meist eifersüchtig gehütet, wodurch nur wenige aufgeschlossene Zauberkundige jemals andere Tradition als ihre eigene über die Grundlagen hinaus erlernen. Die erste Stufe der Tradition ist die Voraussetzung, um \gls{zauber} der jeweiligen Tradition erlernen zu können. Du kannst keinen \gls{zauber} ohne eine Tradition wirken und keine zwei Traditionen gleichzeitig nutzen. Das Einstimmen auf eine andere Tradition benötigt eine \gls{aktion} \gls{bereit machen}. Mit den weiteren Stufen der Tradition können alle mit dieser Tradition eingesetzten \gls{zauber} weiter verbessert werden. Die vierte Stufe der Tradition kann nur bei wenigen Meistern der Zauberei gelernt werden, zu denen du fortan gehörst. Du darfst 8 Punkte verteilen: Für 1 Punkt kann ein \gls{zauber} um \glslink{erleichterung}{+2} erleichtert werden (maximal \glslink{erleichterung}{+4}). Für 2 Punkte kannst du eine \gls{spontane modifikation} außer \gls{mächtige magie} um \glslink{erleichterung}{+1} erleichtern (maximal \glslink{erleichterung}{+1}). Spezielle Vorteile wie eine geringere Patzerchance oder länger nachbrennende Feuerzauber sind ebenfalls möglich. Solche Verbesserungen (und ihre Kosten) sollten mit dem Spielleiter und der Gruppe abgesprochen werden.}}
        
            
    

%aktion
\newglossaryentry{aktion}
{
    name={Aktion},
    description={In seiner \gls{initiativephase} kann dein Charakter eine \glslink{volle aktion}{volle} oder bis zu zwei verschiedene \glslink{einfache aktion}{einfache Aktionen} ausführen – wenn du jedoch zwei einfache Aktionen nutzt, sind alle Proben in diesen Aktionen um \glslink{erschwernis}{–4} erschwert. Proben außerhalb dieser Aktionen, zum Beispiel bei \glslink{reaktion}{Reaktionen}, sind von diesem \glslink{erschwernis}{Malus} nicht betroffen. Aktionen sind: \gls{konflikt}, \gls{volle offensive}, \gls{volle defensive}, \gls{bereit machen}, \gls{bewegung}, \gls{konzentration} und \gls{verzögern}.}}

\newglossaryentry{einfache aktion}
{
    name={Einfache Aktion},
    description={Siehe \gls{aktion}. Einfache Aktionen sind: \gls{konflikt}, \gls{bereit machen} und \gls{bewegung} (mit dem \gls{vorteil} \gls{defensiver kampfstil} auch \gls{volle defensive}).}}

\newglossaryentry{volle aktion}
{
    name={Volle Aktion},
    description={Siehe \gls{aktion}. Volle Aktionen sind: \gls{volle offensive}, \gls{volle defensive}, \gls{konzentration} und \gls{verzögern}.}}

\newglossaryentry{freie aktion}
{
    name={Freie Aktion},
    description={}}
        
\newglossaryentry{konflikt}
{
    name={Konflikt (einfach)},
    description={Du kannst deinen Gegner angreifen, vorbereitete \gls{zauber} auf deine Feinde schleudern oder dein Gegenüber \gls{einschüchtern}. Fast jede \gls{aktion}, die eine \gls{vergleichende probe} beinhaltet, ist ein Konflikt.}}

\newglossaryentry{volle offensive}
{
    name={Volle Offensive (voll)},
    description={Du führst einen tollkühnen Angriff aus. Alle Nahkampfangriffe in deiner Aktion sind um \glslink{erleichterung}{+4} erleichtert, alle Verteidigungen bis zu deiner nächsten \gls{initiativephase} um \glslink{erschwernis}{–8} erschwert.}}

\newglossaryentry{volle defensive}
{
    name={Volle Defensive (voll)},
    description={Du konzentrierst dich voll auf deine Verteidigung. Alle Verteidigungen bis zu deiner nächsten \gls{initiativephase} sind um \glslink{erleichterung}{+4} erleichtert. Mit dem \gls{vorteil} \glslink{defensiver kampfstil} wird volle Defensive zu einer \glslink{einfache aktion}{Einfachen Aktion}.}}

\newglossaryentry{bereit machen}
{
    name={Bereit machen (einfach)},
     description={Du ziehst eine Waffe (1 Aktion), kramst einen Heiltrankhervor (je nach Aufbewahrungsort 2-4 Aktionen) oder führst andere Handlungen aus, die nicht deine volle Aufmerksamkeit benötigen.}}

\newglossaryentry{bewegung}
{
    name={Bewegung (einfach)},
    description={Du läufst, reitest oder schwingst an einem Seil. In einer normalen Kampfsituation kannst du so GS Schritt zurücklegen. Auf unsicherem Untergrund sinkt dieser Wert auf die Hälfte, in liegender oder kniender Position auf ein Viertel. Unter folgenden Bedingungen kannst du dich aber auch weiter bewegen: Geradeaus vorwärts doppelt so weit; geradeaus vorwärts und zusätzlich ohne Gepäck, Rüstung und sperrige Waffen viermal so weit.}}

\newglossaryentry{konzentration}
{
    name={Konzentration (voll)},
    description={Du führst eine Handlung aus, die deine volle Konzentration erfordert. Dazu gehört das Vorbereiten eines \glslink{zauber}{Zaubers} oder Fernkampfangriffes sowie das Entschärfen einer Falle. Du kannst bis zu deiner nächsten \glslink{initiativephase}{Initiativephase} keine \glslink{freie aktion}{Freien Aktionen} oder \glslink{reaktion}{Reaktionen} ausführen und musst bei Störungen eine \glslink{willenskraft}{Willenskraft-Probe} (16, \gls{I}) ablegen. Erleidest du Schaden, steigt die \gls{schwierigkeit} der Probe um \glslink{erleichterung}{+4} Punkte pro soeben erlittener \gls{wunde}. Bei Misslingen verfallen alle bereits aufgewendeten \glslink{aktion}{Aktionen} Konzentration.}}

\newglossaryentry{verzögern}
{
    name={Verzögern (voll)},
    description={Du wartest ab und handelst erst, wenn ein bestimmtes Ereignis eintritt. Dazu bestimmst du das Ereignis und die genaue Handlung, die du ausführen möchtest. Diese Handlung kann unmittelbar vor oder nach dem Ereignis stattfinden, muss aber eine einfache \gls{aktion} sein (Proben in dieser Aktion sind um \glslink{erschwernis}{–4} erschwert, weil auch hier zwei \glslink{aktion}{Aktionen} genutzt werden). Beispiele wären: \textit{Wenn der Ork auf mich zukommt, halte ich den Abstand}, oder \textit{Durchbricht der Dämon den Schutzkreis, attackiere ich ihn mit einem \gls{wuchtschlag} \glslink{erschwernis}{–4}}. Sollte das Ereignis bis zu deiner nächsten \gls{initiativephase} nicht eintreten, verfällt die \gls{aktion}.}}

\newglossaryentry{reaktion}
{
    name={Reaktion},
    description={}}


%modifikationen
\newglossaryentry{mächtige magie}
{
    name={Mächtige Magie},
    description={Basismodifikation (-4): Verstärkt die Wirkung des Zaubers (zu den genauen Auswirkungen siehe die Beschreibung des jeweiligen Zaubers). Zusätzlich steigt die Schwierigkeit von \glslink{konterprobe}{Konterproben} gegen den Zauber um +4 Punkte.}}
    
    
% Kampf

\newglossaryentry{wuchtschlag}
{
    name={Wuchtschlag},
    description={}}

\newglossaryentry{defensiver kampfstil}
{
    name={Defensiver Kampfstil},
    description={}}

\newglossaryentry{manöver}
{
    name={Manöver},
    description={}}

\newglossaryentry{einschüchtern}
{
    name={Einschüchtern},
    description={}}

\newglossaryentry{reichweite}
{
    name={Reichweite},
    description={}}

\newglossaryentry{temperaturstufe}
{
    name={Temperaturstufe},
    description={}}

\newglossaryentry{kampfreflexe}
{
    name={Kampfreflexe},
    description={}}

\newglossaryentry{gezielter schlag}
{
    name={Gezielter Schlag},
    description={}}

\newglossaryentry{zonenwundschwelle}
{
    name={Zonen-Wundschwelle},
    description={Siehe \gls{trefferzonen}}}

\newglossaryentry{intensitätsanalyse}
{
    name={Intensitätsanalyse},
    description={}}

\newglossaryentry{rüstungsgewöhnung}
{
    name={Rüstungsgewöhnung},
    description={}}
        
        
        
        
        
        
        
        
        
        
        
    
        
        
        
        
    

%acronyms
\newacronym{pw}{PW}{\gls{probenwert}}
\newacronym{ew}{EW}{\gls{erfolgswert}}
\newacronym{3w20}{3w20}{3 20-seitige Würfel}
\newacronym{1w20}{1w20}{1 20-seitiger Würfel}
\newacronym{I}{I}{1w20 Probe}
\newacronym{dg}{DG}{\gls{detailgrad}}
\newacronym{ep}{EP}{\gls{erfahrungspunkte}}
\newacronym{ch}{CH}{\gls{charisma}}
\newacronym{ff}{FF}{\gls{fingerfertigkeit}}
\newacronym{ge}{GE}{\gls{gewandtheit}}
\newacronym{in}{IN}{\gls{intuition}}
\newacronym{kk}{KK}{\gls{körperkraft}}
\newacronym{kl}{KL}{\gls{klugheit}}
\newacronym{ko}{KO}{\gls{konstitution}}
\newacronym{mu}{MU}{\gls{mut}}
\newacronym{ws}{WS}{\gls{wundschwelle}}
\newacronym{mr}{MR}{\gls{magieresistenz}}
\newacronym{gs}{GS}{\gls{geschwindigkeit}}
\newacronym{ini}{INI}{\gls{initiative}}
\newacronym{asp}{AsP}{\gls{astralpunkte}}
\newacronym{bw}{BW}{\gls{basiswert}}
\newacronym{fw}{FW}{\gls{fertigkeitswert}}
\newacronym{schip}{Schip}{\gls{schicksalspunkt}}
\newacronym{zrs}{Zonen-Rüstungsschutz}{Siehe \gls{trefferzonen}}
\newacronym{zws}{Zonen-Wundschwelle}{Siehe \gls{trefferzonen}}
\newacronym{hp}{Heilpunkte}{Siehe \gls{übernatürliche heilung}}
\newacronym{be}{Behinderung}{Siehe \gls{rüstung}}
%\newglossaryentry{#1}
%{
%    name=#1,
%    description={#2}
%}