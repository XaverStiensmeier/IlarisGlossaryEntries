%Vorteile
\newglossaryentry{vorteil}
{
    name={Vorteil},
    description={Vorteile sind besondere Fähigkeiten, die deinem Charakter im Spiel nützen können. Sie können entweder schon bei der Generierung oder bei späteren Steigerungen gekauft werden, solange du die Voraussetzungen erfüllst (oft ein Attribut in einer bestimmten Höhe). Deswegen ist es auch kein Problem, wenn du zu Beginn noch nicht alle Vorteile kennst. Zu dieser Regel gibt es allerdings auch einige wenige Ausnahmen unter den allgemeinen Vorteilen. Die Gabe der Magie etwa ist angeboren und kann nicht einfach erlernt werden. Genausowenig kann der Bettler durch den Aufwand von EP in den Adelsstand aufsteigen. Unter dem Punkt Nachkauf findest du deswegen bei jedem allgemeinen Vorteil einen Hinweis, ob und unter welchen Bedingungen der Vorteil im späteren Spiel erworben werden kann. Ist der Nachkauf häufig oder sogar üblich, sollte dir der Spielleiter keine zu großen Steine in den Weg legen. Extrem selten bedeutet hingegen, dass der Nachkauf Dämonenpaktierern, Gezeichneten oder anderen herausragenden Gestalten vorbehalten ist. Wenn du einen besonders exklusiven Vorteil erwirbst, kann der Spielleiter verlangen, dass du eine Lehrmeisterin aufsuchst. Dieser kann dich einer Prüfung unterziehen oder Gegengefallen verlangen, sodass der Kauf des Vorteils zu einem besonderen Moment wird oder sogar direkt ins nächste Abenteuer führt. Vorteile werden unterteilt in: Profane Vorteile, Kampfvorteile, Kampfstile, Magische Vorteile, Magische Traditionen, Karmale Vorteile und Karmale Traditionen}}

\newglossaryentry{achaz}
{
    name={Achaz},
    description={Du kannst mit deinem Schwanz waffenlose Angriffe in \gls{reichweite} 1 (siehe auch S. 38) durchführen. Allerdings leidest du unter Kältestarre; für jede \gls{temperaturstufe} unter normal sind alle körperlichen Proben um \glslink{erschwernis}{–4} erschwert. Voraussetzungen: üblicherweise nur von Achaz wählbar; 0 EP. Nachkauf: extrem selten}}

\newglossaryentry{angepasst}
{
    name={Angepasst I/II (Umgebung)},
    description={Durch deine Spezies oder langjährige Erfahrung hast du dich an eine bestimmte Umgebung oder Umweltbedingung gewöhnt. Abzüge durch diese Umgebung, insbesondere im Kampf, sinken für dich um eine/zwei Stufen. Die Kosten für Angepasst legt der Spielleiter fest, wobei er sich an der Häufigkeit der Umgebung orientieren sollte. Zu allgemein gefasste Umgebungen wie \glqq unsicherer Untergrund\grqq{} sollte er nicht zulassen. Beispiele für Angepasst sind:
\begin{description}
\item Dunkelheit: verringert Abzüge durch schlechte Lichtverhältnisse (40 EP pro Stufe)
\item Schnee: verringert Abzüge durch schneebedeckten oder eisigen Untergrund (20 EP pro Stufe)
\item Wasser: verringert Abzüge durch knie- oder hüfttiefes Wasser und unter Wasser (20 EP pro Stufe)
\item Wald: verringert Abzüge durch Wurzeln, Gestrüpp und dichtes Unterholz (40EP pro Stufe)
\end{description}
Voraussetzungen: keine/Angepasst I. Nachkauf: häufig/selten.}}

\newglossaryentry{besonderer besitz}
{
    name={Besonderer Besitz (Gegenstand)},
    description={Ein besonderer Gegenstand wie die 33-fach gefaltete Klinge deines ruhmreichen Großvaters oder ein heilendes Diadem befindet sich in deinem Besitz. Der Spielleiter sollte diesem Gegenstand einen gewissen Schutz gewähren; er wird nicht zufällig von einem Taschendieb gestohlen oder geht durch Pech verloren, ohne dass du ihn wieder zurückerlangen kannst. Umgekehrt darf der Charakter den Gegenstand auch nicht verkaufen. Die Kosten für einen Besonderen Besitz legt der Spielleiter fest, wobei dieser sich am Spielnutzen des Gegenstandes orientieren sollte. Ein Besonderer Besitz für 40 EP bringt einen willkommenen Bonus in manchen Situationen, während ein Besonderer Besitz für 160 EP in vielen Situationen entscheidend ist. Der Wert des Gegenstandes spielt hierbei keine Rolle. Beispiele für besonderen Besitz sind:
\begin{description}
\item Ein Schlachtross, eine verbesserte Waffe (+2 TP) oder verbesserte Rüstung (–1 BE) oder eine Ferrara-Kutsche mit Pferden (40 EP)
\item Eine verbesserte und magische Waffe oder ein aufladbares Schutzartefakt mit einfachem Auslöser (80 EP)
\item Ein verbessertes Enduriumschwert, ein verbesserter Toschkrilpanzer oder ein Reithippogriff (120 EP)
\item Ein bewaffnetes Handelsschiff oder ein semipermanentes Universalschutzamulett mit intelligentem Auslöser (160 EP)
\end{description}
Nachkauf: nicht möglich}}

\newglossaryentry{einkommen}
{
    name={Einkommen I/II/III/IV},
    description={Deine Familie, Organisation oder eine persönliche Mäzenin stellt dir ein monatliches Einkommen von 4/16/64/256 Dukaten zur Verfügung, die du beispielsweise über eine Filiale der Nordlandbank oder ein Ordenshaus beziehen kannst. Fern der Zivilisation kannst du nicht auf dein Einkommen zugreifen. Das Einkommen reicht aus, um die Lebenshaltungskosten eines Angehörigen der Unterschicht/Mittelschicht/Oberschicht/Elite zu decken. Voraussetzungen: keine/Einkommen I/Einkommen II/ Einkommen III; 20 EP pro Stufe. Nachkauf: häufig/häufig/häufig/häufig. Anmerkung: Wenn Geld in eurem Spiel nicht so wichtig sein soll, könnt ihr den Vorteil Einkommen kostenlos an jeden vergeben.}}

\newglossaryentry{eisenaffine aura}
{
    name={Eisenaffine Aura},
    description={Malusse durch den Bann des Eisens sinken um 8 Punkte. Voraussetzungen: Zauberer I, 40 EP. Nachkauf: selten.}}

\newglossaryentry{gefahreninstinkt}
{
    name={Gefahreninstinkt},
    description={Mit der Gabe Gefahreninstinkt kannst du mit dem Talent Wachsamkeit auch Gefahren wahrnehmen, die mit gewöhnlichen Sinnen nicht wahrzunehmen sind. Dazu gehören etwa magische Fallen oder eine bald losbrechende Lawine. Ist die Gefahr auch mit gewöhnlichen Sinnen wahrnehmbar erhältst du eine Erleichterung von +4. Voraussetzungen: 100 EP. Nachkauf: extrem selten.}}

\newglossaryentry{geweiht}
{
    name={Geweiht I/II/III/IV},
    description={Du verfügst über 8/16/24/32 Karmapunkte und kannst karmale Traditionen erlernen. Voraussetzungen: keine/Geweiht I/Geweiht II/Geweiht III; 40 EP pro Stufe. Nachkauf: üblich/üblich/üblich/üblich.}}

\newglossaryentry{glück}
{
    name={Glück I/II},
    description={Deine maximalen Schicksalspunkte steigen auf 5/6. Voraussetzungen: keine/Glück I; jede Stufe 40 EP. Nachkauf: häufig/häufig.}}

\newglossaryentry{kreis der verdammnis}
{
    name={Kreis der Verdammnis I-VII},
    description={Der Vorteil hat vier Auswirkungen. Erstens erhältst du beim Erwerb der ersten Stufe 400 EP, für jede weitere Stufe 200 EP. Zweitens werden deine Gunstpunkte (S. 93) sofort aufgefüllt. Drittens erhältst du für jede Stufe eine verdorbene Eigenheit. Viertens entwickelst du ein Dämonenmal und kannst über die Liturgie Seelenprüfung als Paktierer erkannt werden. Auf geweihtem/heiligen Boden musst du jede Minute eine Willensstärke-Probe (20/28) ablegen, um nicht durch starke Schmerzen aufzufallen und eine Wunde zu erleiden. Voraussetzungen: 0 EP pro Stufe. Nachkauf: für alle Stufen üblich.}}

\newglossaryentry{magieabweisend}
{
    name={Magieabweisend},
    description={Zauber wirken auf dich deutlich schwächer. Du ignorierst bei allen Zaubern eine Stufe der spontanen Modifikation Mächtige Magie. Zauber ohne Mächtige Magie haben auf dich keine Wirkung. Voraussetzung: 40 EP. Nachkauf: extrem selten.}}

\newglossaryentry{magiegespür}
{
    name={Magiegespür},
    description={In der Nähe astraler Kräfte überfällt dich ein Frösteln, du hörst sphärische Klänge oder ein Farbschleier legt sich für dich über die Umgebung. Mit dem \gls{talent} \glslink{wahrnehmung}{sinnenschärfe} kannst du \glslink{intensitätsanalyse}{Intensitätsanalysen} von magischen Gegenständen durchführen. Nach dem aktiven Einsatz der Gabe erleidest du einen Punkt Erschöpfung. Voraussetzungen: 60 EP. Nachkauf: extrem selten.}}

\newglossaryentry{natürliche rüstung}
{
    name={Natürliche Rüstung},
    description={Du verfügst über dichtes Fell oder zähe Schuppenhaut, wo­durch dein RS um 1 steigt. Die BE verändert sich dadurch nicht. Voraussetzungen: üblicherweise nur von Achaz, Orks oder ähnlichen Spezies wählbar; 80 EP. Nachkauf: extrem selten.}}

\newglossaryentry{paktierer}
{
    name={Paktierer I/II/III/IV},
    description={Du verfügst über 8/16/24/32 Gunstpunkte und kannst die dämonische Tradition deines Erzdämons erlernen. Voraussetzungen: Kreis der Verdammnis I/Paktierer I/Paktierer II/Paktierer III; 40 EP pro Stufe. Nachkauf: üblich/üblich/üblich/üblich.}}

\newglossaryentry{privilegien}
{
    name={Privilegien (Privilegierte Gruppe)},
    description={Unzählige Privilegien können das Leben ihrer Träger erleichtern, weswegen wir auch nicht jedes einzelne Privileg in Werte kleiden möchten. Die Kosten eines Privilegs legt der Spielleiter fest, wobei er sich an der Spielrelevanz und dem Geltungsbereich der Privilegien orientieren sollte. Zum Beispiel bringt ein eigenes Wappen keine spielrelevanten Vorteile – ganz im Gegensatz zu der Erlaubnis, in Städten Waffen zu tragen. Doch auch diese Erlaubnis verliert an Wert, wenn sie nur in Albenhus oder am Markttag gilt. Beispiele für Privilegien sind:
\begin{description}
\item Adel: Du darfst ein \glqq von\grqq{} im Namen tragen, an Turnieren teilnehmen und jederzeit angemessen bewaffnet und gerüstet sein. Vor Gericht kannst du nur von höheren Adligen gerichtet werden und du darfst bei der Abwesenheit eines Richters über Leibeigene richten, solange dabei kein Blut fließt. Diese Privilegien gelten in ähnlicher Art und Weise im Mittel- und Horasreich, dem Bornland und Aranien (40 EP).
\item Krieger/Schwertgeselle/Rondrageweihter: Du darfst an Turnieren teilnehmen und in Städten Waffen tragen. Diese Privilegien gelten im Mittel- und Horasreich, sowie eingeschränkt in fast ganz Aventurien (20 EP).
\item Gildenmagier: Du darfst Geld für magische Dienstleistungen verlangen und hast erleichterten Zugang zu Bibliotheken und Lehrmeistern der eigenen Gilde. Von weltlichen Gerichten kannst du nur in deiner Anwesenheit verurteilt werden oder du wirst vor ein Gildengericht gestellt. Im Gegenzug musst du dich den Bedingungen des Codex Albyricus unterwerfen, also stets als Gildenmagier erkenntlich sein. Auch die meisten Waffen und fast alle Rüstungen mit einem RS von mehr als 1 sind verboten. Diese Privilegien gelten im Mittel- und Horasreich, dem Bornland, im Kalifat und in Aranien, sowie eingeschränkt in den tulamidischen Stadtstaaten und in Südaventurien (benötigt Zauberer, Gildenmagische Rep. I, 20 EP).
\end{description}
Nachkauf: selten.}}

\newglossaryentry{prophezeien}
{
    name={Prophezeien},
    description={Du kannst das Talent Willenskraft nutzen, um mit Spielkarten, Würfeln, Astrologie, Drogen oder prophetischen Träumen einen vagen und meist mehrdeutigen Blick in die Zukunft zu werfen. Nach dem Einsatz der Gabe erleidest du einen Punkt Erschöpfung. Voraussetzungen: 40 EP. Nachkauf: selten.}}

\newglossaryentry{resistenz/immunität gegen gifte}
{
    name={Resistenz/Immunität gegen Gifte},
    description={Resistenz mildert die Auswirkungen von Giften, mit Immunität sind Gifte gegen dich wirkungslos. Voraussetzungen: keine/Resistenz gegen Gifte; 40 EP pro Stufe. Nachkauf: selten/extrem selten.}}

\newglossaryentry{resistenz/immunität gegen krankheiten}
{
    name={Resistenz/Immunität gegen Krankheiten},
    description={Resistenz mildert die Auswirkungen von Krankheiten, mit Immunität erkrankst du niemals. Voraussetzungen: keine/Resistenz gegen Krankheiten; 20 EP pro Stufe. Nachkauf: selten/extrem selten.}}

\newglossaryentry{resistenz gegen hitze/kälte}
{
    name={Resistenz gegen Hitze/Kälte},
    description={Resistenz gegen Hitze/Kälte Verschiebt die Temperaturstufe bei hohen/niedrigen Temperaturen um eine Stufe in Richtung normal. Voraussetzungen: jeder Vorteil 40 EP. Nachkauf: selten.}}

\newglossaryentry{tierempathie}
{
    name={Tierempathie},
    description={Mit der Gabe Tierempathie kannst du ein passendes Talent der Fertigkeit Überleben verwenden, um die Gedanken von Tieren zu verstehen. Zusätzlich kannst du ihnen sogar einfache Botschaften zukommen lassen. Nach dem aktiven Einsatz der Gabe erleidest du einen Punkt Erschöpfung. Voraussetzungen: 60 EP für alle Tiere, 20-40 EP für eine bestimmte Gruppe von Tieren. Nachkauf: extrem selten.}}

\newglossaryentry{verbindungen}
{
    name={Verbindungen (Organisation/Person)},
    description={Du hast einen guten Draht zu einer bestimmten Organisation oder Person, die dir üblicherweise freundlich gegenübersteht. Außerdem gelten passende Verbindungen als Werkzeuge für Recherchen und viele andere Tätigkeiten. Verbindungen können mehrmals gewählt werden, um gute Kontakte zu mehreren Gruppen darzustellen. Die Kosten einer Verbindung legt der Spielleiter fest, wobei er sich an der Macht und dem Einflussbereich der Verbindung orientieren sollte. Beispiele für Verbindungen sind:
\begin{description}
\item Örtlicher Baron: Starker Einfluss in einer kleinen Region (20 EP)
\item Badilakaner: Geringer Einfluss in weiten Teilen des zwölfgöttergläubigen Aventuriens (20 EP)
\item Efferdbrüder: Ansehnlicher Einfluss in Hafenstädten des zwölfgöttergläubigen Aventuriens (40 EP)
\item Haus Gareth: Immenser Einfluss im Mittelreich, weitreichende Beziehungen in ganz Aventurien (80 EP)
\end{description}
Nachkauf: üblich bis selten.
}}

\newglossaryentry{zauberer}
{
    name={Zauberer I/II/III/IV},
    description={Du verfügst über 8/16/24/32 Astralpunkte und kannst magische Traditionen erlernen. Dämonen, Elementare oder Hellsichtsmagier können dich magisch wahrnehmen. Zauberer I entspricht dabei einem einfachen Magiedilettanten, Zauberer II einer wenig begabten Alchemistin. Die meisten ausgebildeten Zauberer verfügen dagegen über Zauberer III oder IV. Voraussetzungen: keine/Zauberer I/Zauberer II/Zauberer III; 40 EP pro Stufe. Nachkauf: extrem selten/üblich/üblich/üblich.}}

\newglossaryentry{zwergennase}
{
    name={Zwergennase},
    description={Mit Zwergennase besitzt du einen übernatürlich guten Riecher für Verstecke, Geheimgänge und mechanische Fallen. Du kannst sie mit dem Talent Wachsamkeit wahrnehmen, auch wenn das mit gewöhnlichen Sinnen unmöglich wäre. Ist das Objekt auch mit gewöhnlichen Sinnen wahrnehmbar, erhältst du eine Erleichterung von +4. Voraussetzungen: 60 EP. Nachkauf: extrem selten.}}
        
    \newglossaryentry{initiativephase}
{
    name=Initiativephase,
    description={In deiner Initiativephase kannst du handeln. Eine Minute entspricht 16 Initiativephasen einer Person.}
}

\newglossaryentry{schwierigkeit}
{
	name={Schwierigkeit},
	description={Wenn dein \gls{erfolgswert} größer gleich der Schwierigkeit ist, gelingt die Probe.}
}

\newglossaryentry{attributprobe}
{
    name={Attributsproben},
    description={Der \gls{probenwert} beträgt das doppelte des \glslink{attribut}{Attributs}.}
}
\newglossaryentry{fertigkeitsprobe}
{
    name={Fertigkeitsprobe},
    description={Der \gls{probenwert} beträgt die Summe aus \gls{basiswert} und \gls{fertigkeitswert}.}
}
\newglossaryentry{basiswert}
{
    name={Basiswert},
    description={Der Basiswert ist der Durchschnitt der drei \glslink{attribut}{Attribute}, die bei jeder \gls{fertigkeit} angegeben sind.}}

\newglossaryentry{erschwernis}
{
    name={Erschwernis},
    description={Verringert den \gls{erfolgswert} (bspw. -2).}}
    
\newglossaryentry{erleichterung}
{
    name={Erleichterung},
    description={Erhöht den \gls{erfolgswert} (bspw. +2).}}
\newglossaryentry{vier stufen}
{
    name={Vier Stufen},
    description={\glslink{erschwernis}{Erschwernisse} bzw. \glslink{erleichterung}{Erleichterungen} betragen immer 2/4/8/16.}}
\newglossaryentry{freiwillig erschweren}
{
    name={Freiwillig erschweren},
    description={Das freiwillige Erschweren gibt dir verschiedene Boni (siehe z.B. \gls{mächtige magie}).}}
\newglossaryentry{hohe qualität}
{
    name={Hohe Qualität},
    description={Mit Hohe Qualität kannst du schärfere Schwerter schmieden, tödlichere Gifte brauen oder deine Gegner stärker \gls{einschüchtern}. Jede Verbesserung erschwert deine Probe um –4. Die genauen Auswirkungen von Hohe Qualität findest du bei der jeweiligen Handlung.}}
\newglossaryentry{zufällig}
{
    name={Zufällig},
    description={Proben, in denen Glück eine große Rolle spielt, werden mit \gls{1w20} statt mit \gls{3w20} abgelegt.}}
\newglossaryentry{umgebung}
{
    name={Umgebung},
    description={Proben, in denen die Umgebung eine große Rolle spielt, werden mit \gls{1w20} statt mit \gls{3w20} abgelegt.}}
\newglossaryentry{triumph}
{
    name={Triumph},
    description={Wenn deine Probe gelingt und der gewertete Würfel eine 20 zeigt, hast du einen Triumph erzielt.Und deinem Charakter gelingt ein außergewöhnlicher, ja herausragender, Erfolg.}}
\newglossaryentry{patzer}
{
    name={Patzer},
    description={Wenn deine Probe misslingt und der Würfel eine 1 zeigt, bedeutet das einen Patzer – dein Charakter hat sich in eine unangenehme oder sogar lebensbedrohliche Situation gebracht.}}
\newglossaryentry{zusammenarbeit}
{
    name={Zusammenarbeit},
    description={Bei manchen Aufgaben ist es sinnvoll, wenn mehrere Charaktere zusammenarbeiten. Der Spielleiter entscheidet,
wie viele Charaktere helfend eingreifen können und wie groß ihr \gls{einfluss} ist. Helfer können eine um 4 Punkte leichtere Probe ablegen, um die entscheidende Probe um +2 (kleiner Einfluss), +4 (großer Einfluss) oder selten sogar +8 (enormer Einfluss) pro Helfer zu erleichtern. Misslingt der Helferin die Probe, ist die entscheidende Probe aber um –2 erschwert.}}
\newglossaryentry{einfluss}
{
    name={Einfluss},
    description={Einfluss kann klein (+2), groß (+4) oder selten sogar enorm (+8) sein. Einfluss ist besonders für \gls{zusammenarbeit} relevant.}}
\newglossaryentry{gruppenprobe}
{
    name={Gruppenprobe},
    description={Dabei würfelt ein Spieler eine Probe, deren Ergebnis für die ganze Gruppe bindend ist.}}

\newglossaryentry{offene probe}
{
    name={Offene Probe},
    description={Dabei würfelt der Spieler gegen eine vom Spielleiter offen bestimmte \gls{schwierigkeit}.}}

\newglossaryentry{vergleichende probe}
{
    name={Vergleichende Probe},
    description={Vergleichende Proben sind ein zentraler Bestandteil von Ilaris und kommen in jedem Konflikt vor, egal ob dieser mit Waffen, Zauberformeln oder Worten ausgetragen wird. Beide Beteiligten würfeln eine Probe auf die geforderte Fertigkeit und der höhere Erfolgswert setzt sich durch. Bei einem Gleichstand gewinnt der Beteiligte mit dem höheren Probenwert, bei gleichem Probenwert gewinnt der Spieler. Vergleichende Proben können auf drei Arten abgelegt werden: \gls{beide seiten würfeln}, \gls{die aktive seite würfelt} und \gls{der spieler würfelt aktiv}.}}
        
\newglossaryentry{ausgedehnte probe}
{
    name={Ausgedehnte Probe},
    description={In einer ausgedehnten Probe bestimmt der Spielleiter den \gls{dg}. Um erfolgreich zu sein, müssen dir insgesamt so viele Einzelproben gelingen, wie der \gls{dg} beträgt – und zwar, bevor dir dieselbe Anzahl an Einzelproben
misslungen ist. (Eine gewöhnliche Probe entspricht also einem \gls{dg} von 1)}
}

\newglossaryentry{detailgrad}
{
    name={Detailgrad},
    description={Bestimmt wie viele Proben bei einer \glslink{ausgedehnte probe}{ausgedehnten Probe} gewürfelt werden. Normalerweise liegt der \gls{dg} bei 2 bis 4.}}

\newglossaryentry{eigenheit}
{
    name={Eigenheit},
    description={Eigenheiten beschreiben die Stärken und Schwächen deines Charakters, seine Herkunft, Ausbildung und Weltanschauung. Außerdem lassen sie sich mit \glslink{schip}{Schips} \glslink{eigenheiten einsetzen}{einsetzen} oder \glslink{eigenheiten ausnutzen}{ausnutzen}, um \glslink{schip}{Schips} zu generieren.}}

\newglossaryentry{status}
{
    name={Status},
    description={Der Status bestimmt, ob dein Charakter in seinem bisherigen Leben Wachteleier gespeist hat oder gerade der menschen­verachtenden Sklavenarbeit in einer Mine entkommen ist: Elite, Oberschicht, Mittelschicht,Unterschicht und Abschaum. Außerdem bestimmt dein Status deine \gls{lebenserhaltungskosten}.}}

\newglossaryentry{erfahrungspunkte}
{
    name={Erfahrungspunkte},
    description={Erfahrungspunkte sind die Währung, mit der du die Werte deines Charakters weiterentwickelst, sodass dein Charakter immer größere Herausforderungen bestehen kann.}}

\newglossaryentry{attributsteigerungskosten}
{
    name={Attributsteigerungskosten},
    description={\glslink{attribut}{Attribute} werden Punkt für Punkt gesteigert. Jeder Punkt kostet Zielwert x 16 \gls{ep}.}}

\newglossaryentry{fertigkeitswertsteigerungskosten}
{
    name={Fertigkeitswertsteigerungskosten},
    description={Fertigkeits­werte werden Punkt für Punkt gesteigert. Jeder Punkt kostet Zielwert x Steigerungsfaktor \gls{ep}. Der Steigerungsfaktor hängt von der \gls{fertigkeit} ab und liegt zwischen 2 und 4.}}

\newglossaryentry{talent}
{
    name={Talent},
    description={Jeder \gls{fertigkeit} sind mehrere Talente zugeordnet, die für einen Teilbereich dieser \gls{fertigkeit} stehen. Ohne das passende Talent darfst du zwar Proben ablegen, dabei aber nur den halben \gls{fertigkeitswert} einsetzen.}}

\newglossaryentry{talentkosten}
{
    name={Talentkosten},
    description={Ein \gls{talent} kostet dich 20 x \gls{steigerungsfaktor} \gls{ep}. Selten anwendbare \glslink{talent}{Talente} sind verbilligt und kosten nur die Hälfte. Da sich jeder Charakter mit den Gebräuchen seiner eigenen Kultur auskennt erhält er das entsprechende \gls{talent} gratis.}}

\newglossaryentry{freie fertigkeit}
{
    name={Freie Fertigkeit},
    description={Freie Fertigkeiten sind verschiedene seltene Handwerks­künste, aber auch Sprachen und Schriften. Freie Fertigkeiten sind in drei Stufen geteilt. Mit der ersten Stufe bist du unerfahren in dieser Freien Fertigkeit, mit der zweiten erfahren und mit der dritten ein Meister dieses Faches. Du kannst diese Stufen I/II/III für nur 4/8/16 EP erwerben, wobei die vorhergehende Stufe vorausgesetzt wird. Selbstverständlich spricht jeder Charakter seine eigene Muttersprache.}}

\newglossaryentry{energie}
{
    name={Energie},
    description={Über \glslink{vorteil}{Vorteile} werden übernatürliche Energien zur Verfügung gestellt. Beispielsweise müssen Zauberer den \gls{vorteil} \gls{zauberer} kaufen, der ihnen \gls{asp} verleiht.}}

\newglossaryentry{traditionsvorteil}
{
    name={Traditionsvorteil},
    description={Die Tradition bestimmt, welche \glslink{übernatürliches talent}{Zauber/Liturgien} du sprechen kannst und verstärkt alle mit ihr eingesetzten \glslink{übernatürliches talent}{Zauber/Liturgien}. Jede Stufe setzt die vorherigen Stufen voraus.}}

\newglossaryentry{übernatürliche fertigkeit}
{
    name={Übernatürliche Fertigkeit},
    description={Kategorie des übernatürlichen Wirkens (bspw. Einfluss, Antimagie oder Seefahrt, Heiliges Handwerk).}}

\newglossaryentry{übernatürliches talent}
{
    name={Übernatürliches Talent},
    description={Sammelbegriff für Liturgien, Zauber und Anrufungen (bspw. Böser Blick, Herr über das Tierreich oder Segensreiches Wasser, Handwerkssegen).}}

\newglossaryentry{energiesteigerungskosten}
{
    name={Energiesteigerungskosten},
    description={Jeder Punkt kostet Zielwert \gls{ep}.}}

\newglossaryentry{schicksalspunkt}
{
    name={Schicksalspunkt},
    description={Schicksalspunkte bieten verschiedene Boni: \gls{glückliche fügung}, \gls{nur ein kratzer} und \gls{von der schippe springen}. Außerdem wird ihr Einsatz zusammen mit passenden \glslink{eigenheit}{Eigenheiten} noch wirkungsvoller.}}

\newglossaryentry{glückliche fügung}
{
    name={Glückliche Fügung},
    description={Du kannst einen \gls{schip} einsetzen, der dir einen zusätzlichen Würfel für die nächste Probe verleiht. Statt mit drei Würfeln würfelst du mit vier und der zweithöchste Wert zählt. Statt mit einem Würfel würfelst du mit zwei und der höchste Wert zählt.}}


\newglossaryentry{nur ein kratzer}
{
    name={Nur ein Kratzer},
    description={Unmittelbar nachdem du Wunden erlitten hast, kannst du einen \gls{schip} einsetzen. Dadurch werden die gerade erlittenen Wunden halbiert.}}


\newglossaryentry{von der schippe springen}
{
    name={Von der Schippe springen},
    description={Du kannst zwei \glslink{schip}{Schips} aufwenden, um den Tod deines Charakters abzuwenden. Dazu musst du deinem Spielleiter einen plausiblen Vorschlag machen, wie dein Charakter die tödliche Situation überlebt. Von der Schippe springen lässt deinen Charakter übrigens nicht ungeschoren davonkommen. Eher überlebt er einen eigentlich tödlichen Sturz schwer verletzt oder wird von den Feinden für tot gehalten und ausgeplündert.}}
                

\newglossaryentry{startschicksalspunkt}
{
    name={Startschicksalspunkt},
    description={Du startest mit sehr sehr 2(sehr reich)/3(reich)/4(normal)/5(arm)/6(sehr arm) \glslink{schip}{Schips}. Du kannst \gls{schip}{Schips} über dem Maximum (normalerweise 4) nicht zurückgewinnen. Dies wirkt sich auch auf dein \gls{startkapital} aus.}}
        

\newglossaryentry{startkapital}
{
    name={Startkapital},
    description={Du startest mit 256(sehr reich)/128(reich)/32(normal)/16(arm)/4(sehr arm) Dukaten. Dies wirkt sich auch auf deine \glslink{startschicksalspunkt}{Startschicksalspunkte} aus.}}

\newglossaryentry{eigenheiten einsetzen}
{
    name={Eigenheiten einsetzen},
    description={Du kannst \glslink{eigenheit}{Eigenheiten} einsetzen, um deinem Charakter einen \gls{vorteil} zu verschaffen. Passt eine deiner \glslink{eigenheit}{Eigenheiten} zu deinem Vorhaben, kannst du \glslink{schip}{Schips} effektiver nutzen: Erstens erhältst du bei einer glücklichen Fügung sogar zwei zusätzliche Würfel. Zweitens kannst du für einen \gls{schip} einen Probenwurf wiederholen. Der Spielleiter hat bei all diesen Varianten das Recht auf ein Veto.}}

\newglossaryentry{eigenheiten ausnutzen}
{
    name={Eigenheiten ausnutzen},
    description={\glslink{eigenheit}{Eigenheiten} stellen außerdem die Schwächen und Marotten deines Charakters dar und können ihm das Leben schwer machen. Als Gegenleistung erhältst du einen \gls{schip}, außer du verfügst schon über die maximale Zahl von \glslink{schip}{Schips}. Dieses Ausnutzen von \glslink{eigenheit}{Eigenheiten} kann auf drei Arten geschehen: \gls{vom spielleiter ausgehend}, \gls{von dir ausgehend}, und \gls{aus dem spiel heraus}.}}

\newglossaryentry{vom spielleiter ausgehend}
{
    name={Vom Spielleiter ausgehend},
    description={Der Spielleiter bietet dir an, eine deiner \glslink{eigenheit}{Eigenheiten} auszunutzen. Wenn du das zulässt, kann er eine Handlung deines Charakters scheitern lassen oder ihn auf eine andere Art und Weise in Schwierigkeiten bringen. Dafür bekommst du einen \gls{schip}. Du kannst das Ausnutzen einer \gls{eigenheit} auch ablehnen, wenn du dafür einen deiner \glslink{schip}{Schips} abgibst. Besitzt du keinen \gls{schip}, kannst du nicht ablehnen.}}

\newglossaryentry{von dir ausgehend}
{
    name={Von dir ausgehend},
    description={Du schlägst dem Spielleiter eine Handlung vor, die deiner Meinung nach einen \gls{schip} wert ist und zu deinen \glslink{eigenheit}{Eigenheiten} passt. Nimmt der Spielleiter an, erhältst du einen \gls{schip}. Hier gibt es kein Risiko, Eigeninitiative wird also belohnt!}}

\newglossaryentry{aus dem spiel heraus}
{
    name={Aus dem Spiel heraus},
    description={Manchmal bringst du dich durch das Ausspielen deiner \glslink{eigenheit}{Eigenheiten} ganz von selbst in haarige Situationen, ohne dass irgendjemand an \glslink{schip}{Schips} gedacht hätte. In diesem Fall sollte dir der Spielleiter nachträglich einen \gls{schip} zugestehen.}}

\newglossaryentry{einsatz erhöhen}
{
    name={Einsatz erhöhen},
    description={Wenn du in einer wirklich entscheidenden Situation das \glslink{eigenheiten ausnutzen}{Ausnutzen} ablehnst, kann der Spielleiter auch den Einsatz erhöhen. Statt einem \gls{schicksalspunkt} erhältst du nun zwei, wenn du das Ausnutzen der \gls{eigenheit} zulässt.}}

\newglossaryentry{lebenserhaltungskosten}
{
    name={Lebenserhaltungskosten},
    description={Diese geben den ungefähren Betrag an, den dein Charakter pro Monat für Unterkunft, Verpflegung, Kleidung, Unterhaltung und alle übrigen Ausgaben des täglichen Lebens ausgibt, wenn er nicht auf Selbstversorgung zurückgreift oder auf fremde Kosten lebt. Ein höherer \gls{status} bedeutet höhere Lebenserhaltungskosten. Der \gls{vorteil} \gls{einkommen} kann dabei helfen auch hohe Lebenserhalungskosten zu bestreiten.}}

\newglossaryentry{attribut}
{
    name={Attribut},
    description={Die insgesamt acht Attribute (\gls{ch}, \gls{ff}, \gls{ge}, \gls{in}, \gls{kk}, \gls{kl}, \gls{ko} und \gls{mu}) stellen die geistige und körperliche Grundlage deines Charakters dar. Manchmal legst du \glslink{attributprobe}{Attributproben} ab. Darüber hinaus bestimmen sie deine \glslink{abgeleiteter wert}{abgeleiteten Werte} (s.u.), beeinflussen den \gls{bw} für \glslink{fertigkeitsprobe}{Fertigkeitsproben} und werden für viele \glslink{vorteil}{Vorteile} vorausgesetzt.}}

\newglossaryentry{charisma}
{
    name={Charisma},
    description={ist die natürliche Ausstrahlung deines Charakters auf seine Umgebung. Es steht für Führungsqualitäten, Selbstbewusstsein, Überzeugungskraft und ein gewinnendes Wesen. Alle gesellschaftlichen Fertigkeiten hängen von deinem Charisma ab und auch Elementaristen benötigen ein hohes Charisma.}}

\newglossaryentry{fingerfertigkeit}
{
    name={Fingerfertigkeit},
    description={ist die manuelle Geschicklichkeit deines Charakters. Er verfügt über eine gute Hand-Augen-Koordination und kann feine Bewegungen schnell und fehlerfrei ausführen. Handwerker, Fernkämpfer und Hersteller von Artefakten sollten deswegen nicht auf eine hohe Fingerfertigkeit verzichten.}}

\newglossaryentry{gewandtheit}
{
    name={Gewandtheit},
    description={ist die Beweglichkeit und Gelenkigkeit deines Charakters. Gewandte Charaktere bewegen sich geschmeidig und können ihre Bewegungen gut abschätzen. Das fördert deine körperlichen und kämpferischen Fertigkeiten und erhöht deine \gls{gs}.}}

\newglossaryentry{intuition}
{
    name={Intuition},
    description={ermöglicht es deinem Charakter, Personen und Sachverhalte schnell richtig einzuschätzen. Sie steht auch für Wahrnehmung und Einfühlungsvermögen deines Charakters. Intuition beeinflusst eine breite Palette an Fertigkeiten und bestimmt deine \gls{ini} im Kampf.}}

\newglossaryentry{körperkraft}
{
    name={Körperkraft},
    description={ist ein Maß für die Stärke deines Charakters. Kräftige Charaktere können schwere Lasten heben und ihre Angriffe richten durch den höheren \gls{schadensbonus} mehr Schaden an. Dadurch ist eine hohe Körperkraft gerade für kämpferische Charaktere hilfreich.}}

\newglossaryentry{klugheit}
{
    name={Klugheit},
    description={Klugheit ist das logische Denkvermögen deines Charakters und seine Fähigkeit, komplizierte Zusammenhänge zu erkennen und zu analysieren. Kluge Charaktere verfügen auch über ein gutes Allgemeinwissen. Eine hohe Klugheit ist bei vielen gesellschaftlichen und allen Wissensfertigkeiten unverzichtbar.}}

\newglossaryentry{konstitution}
{
    name={Konstitution},
    description={beschreibt die Widerstandsfähigkeit deines Charakters gegen äußere Einflüsse wie Strapazen, Gifte, Krankheiten und Verwundungen. Die Konstitution beeinflusst nur wenige \glslink{fertigkeit}{Fertigkeiten}, bestimmt aber deine \gls{wundschwelle}, den wohl wichtigsten \glslink{abgeleiteter wert}{abgeleiteten Wert}.}}

\newglossaryentry{mut}
{
    name={Mut},
    description={Mut (MU): Ein mutiger Charakter bewahrt in kritischen Situationen einen kühlen Kopf und schreckt nicht vor Gefahren zurück, was gerade im Nahkampf unerlässlich ist. Zusätzlich stärkt Mut den Widerstand gegen magische Beeinflussungen, indem er deine \gls{magieresistenz} erhöht. Auch Dämonologen benötigen einen hohen Mut.}}

\newglossaryentry{abgeleiteter wert}
{
    name={Abgeleiteter Wert},
    description={Die abgeleiteten Werte berechnen sich aus deinen \glslink{attribut}{Attributen}.
Du kannst sie nicht direkt steigern, sondern musst entweder die \glslink{attribut}{Attribute} erhöhen oder spezielle \glslink{vorteil}{Vorteile} erwerben. Die abgeleiteten Werte sind \gls{ws}, \gls{mr}, \gls{gs}, \gls{ini} und \gls{schadensbonus}.}}

\newglossaryentry{wundschwelle}
{
    name={Wundschwelle},
    description={bestimmt, wie gut dein Charakter Schaden widerstehen kann. Schaden bis zu deiner Wundschwelle ist zwar schmerzhaft, aber noch nicht wirklich gefährlich. Erst Schadensmengen über deiner Wundschwelle können deinen Charakter beeinträchtigen oder sogar töten. Die Wundschwelle beträgt 4 + 1 für je 4 volle Punkte \gls{ko}. Mit dem Vorteil \gls{unverwüstlich} kannst du die \gls{ws} um +1 erhöhen.}}

\newglossaryentry{magieresistenz}
{
    name={Magieresistenz},
    description={ist die Widerstandsfähigkeit gegen Zauberei. Viele Zauber wirken nur auf deinen Charakter, wenn sie seine Magieresistenz in einer vergleichenden Probe überwinden. Die Magieresistenz beträgt 4 + 1 für je 4 volle Punkte \gls{mu}. Die Vorteile \gls{willensstark I} und \glslink{willensstark II}{II} und \gls{unbeugsamkeit} steigern die \gls{mr} weiter.}}

\newglossaryentry{geschwindigkeit}
{
    name={Geschwindigkeit},
    description={stellt die Schnelligkeit und Beweglichkeit deines Charakters dar. Sie bestimmt, wie weit er sich im Kampf bewegen kann und hilft auch bei Verfolgungsjagden zu Fuß. Die Geschwindigkeit beträgt 4 + 1 für je 4 volle Punkte \gls{ge}. Eine noch höhere \gls{gs} verleihen ihm \gls{flink I} und \glslink{flink II}{II}.}}

\newglossaryentry{initiative}
{
    name={Initiative},
    description={steht für Reaktionsgeschwindigkeit und Übersicht im Kampf. Kämpfer mit hoher Initiative können zu Beginn eines Kampfes schneller handeln und so den Erstschlag führen. Die Initiative entspricht der \gls{in}. Der \gls{vorteil} \gls{kampfreflexe} erhöht die \gls{ini} zusätzlich.}}

\newglossaryentry{schadensbonus}
{
    name={Schadensbonus},
    description={erhöht den Waffenschaden bei allen Nahkampfangriffen. Kopflastige Waffen profitieren sogar doppelt vom Schadensbonus. Für je 4 volle Punkte \gls{kk} erhältst du +1 Schadensbonus.}}

\newglossaryentry{probenwert}
{
    name={Probenwert},
    description={Der Probenwert für Fertigkeiten besteht aus zwei Teilen, dem \gls{basiswert} und dem \gls{fertigkeitswert}.}}

\newglossaryentry{erfolgswert}
{
    name={Erfolgswert},
    description={Der Erfolgswert ist die Summe des \gls{probenwurf} und des \gls{probenwert} zuzüglich der Modifikatoren. Er wird mit der \gls{schwierigkeit} verglichen, um den Erfolg der Probe zu bestimmen.}}

\newglossaryentry{astralpunkte}
{
    name={Astralpunkte},
    description={Astralpunkte sind eine Maßeinheit, die angibt wie viel Astralkraft eine Person hat. Sie wird für Zauber ausgegeben.}}

\newglossaryentry{probenwurf}
{
    name={Probenwurf},
    description={1. Würfle mit drei zwanzigseitigen Würfeln (\gls{3w20}). Der
Würfel mit dem mittleren Ergebnis gilt. 2. Addiere den \gls{probenwert} sowie
mögliche \glslink{erschwernis}{Erschwernisse} oder \glslink{erleichterung}{Erleichterungen}. Das
Ergebnis ist der \gls{erfolgswert}. 3. Ist der \gls{erfolgswert} höher als oder gleich hoch wie die
\gls{schwierigkeit}, gelingt die Probe.}}

\newglossaryentry{fertigkeitswert}
{
    name={Fertigkeitswert},
    description={Den Fertigkeitswert kannst du im Gegensatz zum \gls{basiswert} direkt steigern, maximal bis auf das höchste am Basiswert beteiligte \gls{attribut} +2.}}

\newglossaryentry{konterprobe}
{
    name={Konterprobe(Talent oder Attribut)},
    description={Manchen Zaubern kannst du auch mit profanen \gls{fertigkeit}{Fertigkeiten} widerstehen, zum Beispiel wenn du dich mit deiner \gls{willenskraft} gegen die Wirkung des Friedenslieds stemmst. In diesem Fall kannst du eine Konterprobe ablegen. Die Auswirkungen und die \gls{schwierigkeit} einer Konterprobe findest du beim jeweiligen\gls{übernatürliches talent}{Magie/Liturgie}, die \gls{schwierigkeit} einer Konterprobe steigt aber in jedem Fall um +4 pro Mächtige \gls{übernatürliches talent}{Magie/Liturgie}. Das gilt auch, wenn \gls{mächtige magie} keine anderen Auswirkungen hat und deswegen beim \gls{zauber} gar nicht angeführt ist.}}

%XP
\newglossaryentry{steigerungsfaktor}
{
    name={Steigerungsfaktor},
    description={Der Steigerungsfaktor ist ein Faktor für die Steigerung des \gls{fertigkeitswert} (siehe \gls{fertigkeitswertsteigerungskosten}).}}

\newglossaryentry{verbilligt}
{
    name={verbilligt},
    description={Selten anwendbare Talente sind verbilligt und kosten
nur die Hälfte.}}
        

\newglossaryentry{beide seiten würfeln}
{
    name={Beide Seiten würfeln},
    description={Beide Seiten der \glslink{vergleichende probe}{Vergleichenden Probe} würfeln: Dies ist die langsamste Variante, bietet aber die größte Breite an unterschiedlichen Ergebnissen. Wir empfehlen diese Variante in Konflikten, in denen nur wenige Proben gewürfelt werden – etwa in einer \gls{verfolgungsjagd}.}}

\newglossaryentry{die aktive seite würfelt}
{
    name={Die aktive Seite würfelt},
    description={In dieser Variante wird für den passiveren Teilnehmer am Konflikt ein Würfelwurf von 10 angenommen. Das beschleunigt das Spiel, aber nimmt dir manchmal die Möglichkeit, dich aktiv zu verteidigen. Deswegen empfehlen wir diese Variante für den Spielleiter, wenn er seine Spieler nicht warnen möchte – etwa wenn sie in einen Hinterhalt laufen oder sie unbemerkt mit einem Bannbaladin belegt werden.}}

\newglossaryentry{der spieler würfelt aktiv}
{
    name={Der Spieler würfelt aktiv},
    description={Der Spieler würfelt seine Probe immer aktiv, während der Spielleiter für seine Nicht­s­­pieler­charaktere einen Wurf von 10 annimmt. Dadurch wird das Spiel beschleunigt und der Spiel­leiter entlastet. Wir empfehlen diese Variante für Szenen, in denen häufig gewürfelt wird – wie in Kämpfen.}}


% sonstiges profanes

\newglossaryentry{verfolgungsjagd}
{
    name={Verfolgungsjagd},
    description={Verfolgungsjagden beginnen, sobald die Verfolger bemerkt werden. Der Spielleiter legt den \gls{dg} fest, dann beschreibst du deine Handlungen und es werden Proben (\gls{I}) auf das \gls{talent} abgelegt, das deiner \gls{fortbewegungsart} entspricht. Dabei wird die \gls{gs} deines Fortbewegungsmittels zum \gls{pw} addiert. Gewinnt der Flüchtende den Konflikt, kann er seinen Verfolgern entkommen. Bei einem Sieg der Verfolger haben diese ihr Ziel eingeholt und befinden sich in (Nah-)Kampfreichweite.}}

\newglossaryentry{fortbewegungsart}
{
    name={Fortbewegungsart},
    description={Unter anderem für \glslink{verfolgungsjagd}{Verfolgungsjagden} relevant. Zu Fuß: Laufen, Im Wasser: Schwimmen, Auf einem Reittier: Reiten, Mit einem Schiff: \glslink{freie fertigkeit}{Freie Fertigkeiten} (z.B. Seefahrt). Natürlich sind andere Fortbewegungsarten denkbar.}}
        
    
% talente

\newglossaryentry{zähigkeit}
{
    name={Zähigkeit},
    description={}}

\newglossaryentry{willenskraft}
{
    name={Willenskraft},
    description={}}
        
        
% magie

\newglossaryentry{zauber}
{
    name={Zauber},
    description={Zauber sind eine Untergruppe der \glslink{übernatürliche fertigkeit}{Überntürlichen Fertigkeiten} (siehe \gls{zauber wirken}).}}

\newglossaryentry{zauber wirken}
{
    name={Zauber wirken},
    description={}}
        

\newglossaryentry{magie}
{
    name={Magie},
    description={Aventurien ist von Magie durchzogen, doch nur Charaktere mit dem Vorteil \gls{zauberer} können die allgegenwärtige Astralenergie aufnehmen und mit diesen \glslink{asp}{Vorrat} ihre Zauber wirken. Diese Zauber kannst du mit \glslink{spontane modifikation}{spontanen Modifikationen} verstärken, beschleunigen oder auf mehrere Ziele ausweiten. Die Unterschiede zwischen Zauberern werden durch die \gls{magische tradition} verdeutlicht – sie bestimmt die Stärken und Schwächen deiner Zauberei und welche \gls{zauber} du überhaupt erlernen kannst.}}

\newglossaryentry{spontane modifikation}
{
    name={Spontane Modifikation},
    description={Spontane Modifikationen helfen dir, deine \gls{zauber} an die aktuelle Situation anzupassen. Normalerweise sind für einen \gls{zauber} alle spontanen Modifikationen möglich – aber nicht unbedingt sinnvoll. Genauso wie \gls{manöver} sind spontane Modifikationen stets miteinander kombinierbar, außerdem können sie mehrmals eingesetzt werden. Spontane Modifikationen werden in \glslink{basismodifikation}{Basismodifikationen} und \glslink{aufbauende modifikation}{Aufbauende Modifikationen} unterteilt.}}

\newglossaryentry{aufbauende modifikation}
{
    name={Aufbauende Modifikation},
    description={}}

\newglossaryentry{basismodifikation}
{
    name={Basismodifikation},
    description={}}

\newglossaryentry{magische tradition}
{
    name={Magische Tradition},
    description={Jede Art von Zauberwirkern gibt ihre einzigartige Tradition an ihre Schüler weiter. Dieses Wissen meist eifersüchtig gehütet, wodurch nur wenige aufgeschlossene Zauberkundige jemals andere Tradition als ihre eigene über die Grundlagen hinaus erlernen. Die erste Stufe der Tradition ist die Voraussetzung, um \gls{zauber} der jeweiligen Tradition erlernen zu können. Du kannst keinen \gls{zauber} ohne eine Tradition wirken und keine zwei Traditionen gleichzeitig nutzen. Das Einstimmen auf eine andere Tradition benötigt eine \gls{aktion} \gls{bereit machen}. Mit den weiteren Stufen der Tradition können alle mit dieser Tradition eingesetzten \gls{zauber} weiter verbessert werden. Die vierte Stufe der Tradition kann nur bei wenigen Meistern der Zauberei gelernt werden, zu denen du fortan gehörst. Du darfst 8 Punkte verteilen: Für 1 Punkt kann ein \gls{zauber} um \glslink{erleichterung}{+2} erleichtert werden (maximal \glslink{erleichterung}{+4}). Für 2 Punkte kannst du eine \gls{spontane modifikation} außer \gls{mächtige magie} um \glslink{erleichterung}{+1} erleichtern (maximal \glslink{erleichterung}{+1}). Spezielle Vorteile wie eine geringere Patzerchance oder länger nachbrennende Feuerzauber sind ebenfalls möglich. Solche Verbesserungen (und ihre Kosten) sollten mit dem Spielleiter und der Gruppe abgesprochen werden.}}
        
            
    

%aktion
\newglossaryentry{aktion}
{
    name={Aktion},
    description={In seiner \gls{initiativephase} kann dein Charakter eine \glslink{volle aktion}{volle} oder bis zu zwei verschiedene \glslink{einfache aktion}{einfache Aktionen} ausführen – wenn du jedoch zwei einfache Aktionen nutzt, sind alle Proben in diesen Aktionen um \glslink{erschwernis}{–4} erschwert. Proben außerhalb dieser Aktionen, zum Beispiel bei \glslink{reaktion}{Reaktionen}, sind von diesem \glslink{erschwernis}{Malus} nicht betroffen. Aktionen sind: \gls{konflikt}, \gls{volle offensive}, \gls{volle defensive}, \gls{bereit machen}, \gls{bewegung}, \gls{konzentration} und \gls{verzögern}.}}

\newglossaryentry{einfache aktion}
{
    name={Einfache Aktion},
    description={Siehe \gls{aktion}. Einfache Aktionen sind: \gls{konflikt}, \gls{bereit machen} und \gls{bewegung} (mit dem \gls{vorteil} \gls{defensiver kampfstil} auch \gls{volle defensive}).}}

\newglossaryentry{volle aktion}
{
    name={Volle Aktion},
    description={Siehe \gls{aktion}. Volle Aktionen sind: \gls{volle offensive}, \gls{volle defensive}, \gls{konzentration} und \gls{verzögern}.}}

\newglossaryentry{freie aktion}
{
    name={Freie Aktion},
    description={}}
        
\newglossaryentry{konflikt}
{
    name={Konflikt (einfach)},
    description={Du kannst deinen Gegner angreifen, vorbereitete \gls{zauber} auf deine Feinde schleudern oder dein Gegenüber \gls{einschüchtern}. Fast jede \gls{aktion}, die eine \gls{vergleichende probe} beinhaltet, ist ein Konflikt.}}

\newglossaryentry{volle offensive}
{
    name={Volle Offensive (voll)},
    description={Du führst einen tollkühnen Angriff aus. Alle Nahkampfangriffe in deiner Aktion sind um \glslink{erleichterung}{+4} erleichtert, alle Verteidigungen bis zu deiner nächsten \gls{initiativephase} um \glslink{erschwernis}{–8} erschwert.}}

\newglossaryentry{volle defensive}
{
    name={Volle Defensive (voll)},
    description={Du konzentrierst dich voll auf deine Verteidigung. Alle Verteidigungen bis zu deiner nächsten \gls{initiativephase} sind um \glslink{erleichterung}{+4} erleichtert. Mit dem \gls{vorteil} \glslink{defensiver kampfstil} wird volle Defensive zu einer \glslink{einfache aktion}{Einfachen Aktion}.}}

\newglossaryentry{bereit machen}
{
    name={Bereit machen (einfach)},
     description={Du ziehst eine Waffe (1 Aktion), kramst einen Heiltrankhervor (je nach Aufbewahrungsort 2-4 Aktionen) oder führst andere Handlungen aus, die nicht deine volle Aufmerksamkeit benötigen.}}

\newglossaryentry{bewegung}
{
    name={Bewegung (einfach)},
    description={Du läufst, reitest oder schwingst an einem Seil. In einer normalen Kampfsituation kannst du so GS Schritt zurücklegen. Auf unsicherem Untergrund sinkt dieser Wert auf die Hälfte, in liegender oder kniender Position auf ein Viertel. Unter folgenden Bedingungen kannst du dich aber auch weiter bewegen: Geradeaus vorwärts doppelt so weit; geradeaus vorwärts und zusätzlich ohne Gepäck, Rüstung und sperrige Waffen viermal so weit.}}

\newglossaryentry{konzentration}
{
    name={Konzentration (voll)},
    description={Du führst eine Handlung aus, die deine volle Konzentration erfordert. Dazu gehört das Vorbereiten eines \glslink{zauber}{Zaubers} oder Fernkampfangriffes sowie das Entschärfen einer Falle. Du kannst bis zu deiner nächsten \glslink{initiativephase}{Initiativephase} keine \glslink{freie aktion}{Freien Aktionen} oder \glslink{reaktion}{Reaktionen} ausführen und musst bei Störungen eine \glslink{willenskraft}{Willenskraft-Probe} (16, \gls{I}) ablegen. Erleidest du Schaden, steigt die \gls{schwierigkeit} der Probe um \glslink{erleichterung}{+4} Punkte pro soeben erlittener \gls{wunde}. Bei Misslingen verfallen alle bereits aufgewendeten \glslink{aktion}{Aktionen} Konzentration.}}

\newglossaryentry{verzögern}
{
    name={Verzögern (voll)},
    description={Du wartest ab und handelst erst, wenn ein bestimmtes Ereignis eintritt. Dazu bestimmst du das Ereignis und die genaue Handlung, die du ausführen möchtest. Diese Handlung kann unmittelbar vor oder nach dem Ereignis stattfinden, muss aber eine einfache \gls{aktion} sein (Proben in dieser Aktion sind um \glslink{erschwernis}{–4} erschwert, weil auch hier zwei \glslink{aktion}{Aktionen} genutzt werden). Beispiele wären: \textit{Wenn der Ork auf mich zukommt, halte ich den Abstand}, oder \textit{Durchbricht der Dämon den Schutzkreis, attackiere ich ihn mit einem \gls{wuchtschlag} \glslink{erschwernis}{–4}}. Sollte das Ereignis bis zu deiner nächsten \gls{initiativephase} nicht eintreten, verfällt die \gls{aktion}.}}

\newglossaryentry{reaktion}
{
    name={Reaktion},
    description={}}


%modifikationen
\newglossaryentry{mächtige magie}
{
    name={Mächtige Magie},
    description={Basismodifikation (-4): Verstärkt die Wirkung des Zaubers (zu den genauen Auswirkungen siehe die Beschreibung des jeweiligen Zaubers). Zusätzlich steigt die Schwierigkeit von \glslink{konterprobe}{Konterproben} gegen den Zauber um +4 Punkte.}}
    
    
% Kampf

\newglossaryentry{wuchtschlag}
{
    name={Wuchtschlag},
    description={}}

\newglossaryentry{defensiver kampfstil}
{
    name={Defensiver Kampfstil},
    description={}}

\newglossaryentry{manöver}
{
    name={Manöver},
    description={}}

\newglossaryentry{einschüchtern}
{
    name={Einschüchtern},
    description={}}

\newglossaryentry{reichweite}
{
    name={Reichweite},
    description={}}

\newglossaryentry{temperaturstufe}
{
    name={Temperaturstufe},
    description={}}

\newglossaryentry{kampfreflexe}
{
    name={Kampfreflexe},
    description={}}

\newglossaryentry{gezielter schlag}
{
    name={Gezielter Schlag},
    description={}}

\newglossaryentry{zonenwundschwelle}
{
    name={Zonen-Wundschwelle},
    description={Siehe \gls{trefferzonen}}}
        
        
        
        
        
        
        
        
    
        
        
        
        
    

%acronyms
\newacronym{pw}{PW}{\gls{probenwert}}
\newacronym{ew}{EW}{\gls{erfolgswert}}
\newacronym{3w20}{3w20}{3 20-seitige Würfel}
\newacronym{1w20}{1w20}{1 20-seitiger Würfel}
\newacronym{I}{I}{1w20 Probe}
\newacronym{dg}{DG}{\gls{detailgrad}}
\newacronym{ep}{EP}{\gls{erfahrungspunkte}}
\newacronym{ch}{CH}{\gls{charisma}}
\newacronym{ff}{FF}{\gls{fingerfertigkeit}}
\newacronym{ge}{GE}{\gls{gewandtheit}}
\newacronym{in}{IN}{\gls{intuition}}
\newacronym{kk}{KK}{\gls{körperkraft}}
\newacronym{kl}{KL}{\gls{klugheit}}
\newacronym{ko}{KO}{\gls{konstitution}}
\newacronym{mu}{MU}{\gls{mut}}
\newacronym{ws}{WS}{\gls{wundschwelle}}
\newacronym{mr}{MR}{\gls{magieresistenz}}
\newacronym{gs}{GS}{\gls{geschwindigkeit}}
\newacronym{ini}{INI}{\gls{initiative}}
\newacronym{asp}{AsP}{\gls{astralpunkte}}
\newacronym{bw}{BW}{\gls{basiswert}}
\newacronym{fw}{FW}{\gls{fertigkeitswert}}
\newacronym{schip}{Schip}{\gls{schicksalspunkt}}
\newacronym{zrs}{Zonen-Rüstungsschutz}{Siehe \gls{trefferzonen}}
\newacronym{zws}{Zonen-Wundschwelle}{Siehe \gls{trefferzonen}}
%\newglossaryentry{#1}
%{
%    name=#1,
%    description={#2}
%}%Fertigkeit
\newglossaryentry{fertigkeit}
{
    name={Fertigkeit},
    description={Fertigkeiten setzen sich aus verwandten Bereichen zusammen, von denen dein Charakter grundsätzlich etwas versteht, aber die er nicht alle gleich gut beherrscht – sogenannten \gls{talent}{Talenten}.}}

\newglossaryentry{nahkampffertigkeit}
{
    name={Nahkampffertigkeit},
    description={Erste Nahkampferfahrung muss sich dein Charakter mühsam erarbeiten. Sobald er jedoch ein Gefühl für die Dynamik eines Kampfes besitzt, wird er auch andere Arten des Kampfes schnell begreifen. Deswegen steigerst du Nahkampffertigkeiten normalerweise nach \gls{steigerungsfaktor} 4. Besitzt du aber vor dem Steigern Nahkampffertigkeiten mit höheren \glslink{fertigkeitswert}{Fertigkeitswerten}, gilt ein Faktor von 2. Für Talente gilt immer \gls{steigerungsfaktor} 2. Durch diese Regel ändert sich übrigens nichts an den Gesamtkosten, egal in welcher Reihenfolge du steigerst.}}

\newglossaryentry{handgemenge}
{
    name={Handgemenge (GE/KK/MU, 4/2)},
    description={Auf Handgemenge würfelst du bei allen waffenlosen Nahkampftechniken, im Umgang mit Handgemengewaffen wie Messern, Dolchen, Schlagstöcken oder Kettenstäben und bei der Verwendung von Schilden. Talente: Handgemengewaffen, Unbewaffnet, Schilde.}}

\newglossaryentry{hiebwaffen}
{
    name={Hiebwaffen (GE/KK/MU, 4/2)},
    description={Mit der Fertigkeit Hiebwaffen führst du alle Arten von stumpfen und scharfen Hiebwaffen, wie Keulen, Äxte und Hämmer. Talente: Einhandhiebwaffen, Zweihandhiebwaffen.}}

\newglossaryentry{klingenwaffen}
{
    name={Klingenwaffen (GE/KK/MU, 4/2)},
    description={Im Kampf mit allen Varianten von Schwertern, Säbeln und Fechtwaffen werden Proben auf die Fertigkeit Klingenwaffen abgelegt. Dolche gehören nicht zu den Klingenwaffen – sie werden mit der Fertigkeit \gls{handgemenge} genutzt. Talente: Einhandklingenwaffen, Zweihandklingenwaffen.}}

\newglossaryentry{stangenwaffen}
{
    name={Stangenwaffen (GE/KK/MU, 4/2)},
    description={Alle Waffen mit einem langen Schaft gelten als Stangenwaffen. Zu den klassischen Stangenwaffen zählen Speere, Hellebarden und der Schnitter. Zusätzlich werden Kampfstäbe und die Lanzen berittener Kämpfer mit dieser Fertigkeit genutzt. Talente: Infanteriewaffen und Speere, Lanzenreiten (\gls{verbilligt})}}

\newglossaryentry{fernkampffertigkeiten}
{
    name={Fernkampffertigkeiten},
    description={}}

\newglossaryentry{schusswaffen}
{
    name={Schusswaffen (FF/IN/KK, 3)},
    description={Mit der Fertigkeit Schusswaffen bedienst du alle Waffen, mit denen Geschosse auf den Gegner abgefeuert werden. Dazu gehören so verbreitete Schusswaffen wie der Bogen oder die Armbrust, aber auch die Schleuder oder das Blasrohr. Talente: Armbrüste, Blasrohre (verbilligt), Bögen}}

\newglossaryentry{wurfwaffen}
{
    name={Wurfwaffen (FF/IN/KK, 2)},
    description={Als Wurfwaffen gelten sämtliche geworfenen Geschosse wie Diskusse oder Wurfspeere. Zusätzlich umfassen sie kurze Wurfwaffen, zu denen Dolche, Wurfsterne und improvisierte Geschosse wie einen Stein oder eine Flasche gehören. Talente: Diskusse, kurze Wurfwaffen, Schleudern (\gls{verbilligt}), Wurfspeere.}}

\newglossaryentry{profane fertigkeiten}
{
    name={Profane Fertigkeiten},
    description={Profane Fertigkeiten können von jedem Charakter genutzt werden.}}

\newglossaryentry{athletik}
{
    name={Athletik (GE/KK/KO, 3)},
    description={Athletik umfasst alle Aktivitäten, bei denen der Charakter seinen ganzen Körper kurz- oder längerfristig koordiniert einsetzen muss. Talente: Laufen, Klettern, Schwimmen, Reiten, Akrobatik.
    \begin{description}
\item Laufen kommt bei Verfolgungsjagden zu Fuß zum Einsatz, wenn du einen Verbrecher stellen oder einem Raubtier entkommen möchtest.
\item Mit Klettern überwindest du alle Arten von Hindernissen.
\item Schwimmen erlaubt eine schnellere Fortbewegung im Wasser und längere Tauchgänge.
\item Reiten ist die Fähigkeit, ein Pferd, ein Kamel oder einen Hippogriff zu kontrollieren und als schnelles Reisemittel oder im Kampf einzusetzen.
\item Akrobatik wird für gewagte Kunststücke und Balanceakte verwendet und kann auch den Fallschaden reduzieren.
\end{description}}}

\newglossaryentry{heimlichkeit}
{
    name={Heimlichkeit (GE/IN/MU, 2)},
    description={Heimlichkeit ermöglicht deinem Charakter, ungehört und ungesehen zu bleiben. Die Probe wird häufig vergleichend gegen die Wachsamkeit deines Kontrahenten abgelegt. Talente: Pirschen, Untertauchen.
\begin{description}
\item Pirschen ermöglicht das Schleichen, Verstecken und Lauern in der freien Natur. Du pirschst dich auf der Jagd an einen wilden Hirsch heran, legst einen Hinterhalt an einer Reichsstraße oder beobachtest ungesehen das Lager einer Orkbande.
\item Untertauchen erlaubt dir das Schleichen, Verstecken und Beschatten in der Zivilisation. Du verschwindest in der Menschenmenge am Marktplatz, steigst ungehört in die Villa eines Ratsherren ein oder beschattest unauffällig einen Hehler.
\end{description}}}

\newglossaryentry{selbstbeherrschung}
{
    name={Selbstbeherrschung (KO/MU/MU, 3)},
    description={Selbstbeherrschung erlaubt dir, widrige Umstände, körperliche Schmerzen oder Ablenkungen zu ignorieren und dich auf das einzig Wichtige zu konzentrieren. Talente: Willenskraft, Zähigkeit.
\begin{description}
\item Willenskraft erlaubt dir, Beeinflussungen und Versuchungen zu widerstehen und dich nicht ablenken zu lassen. Du kannst dich im Chaos des Gefechts auf einen Zauber konzentrieren oder einen Heiligen Befehl abschütteln. 
\item Zähigkeit hilft dir, Schmerzen und Strapazen wegzustecken. So kannst du trotz zahlreicher Wunden noch handlungsfähig bleiben und einen Kampf zu euren Gunsten wenden oder eine ganze Nacht hindurch wachen.
\end{description}}}

\newglossaryentry{wahrnehmung}
{
    name={Wahrnehmung (IN/IN/KL, 4)},
    description={Wahrnehmung ist die Fähigkeit, selbst kleinste Sinneseindrücke zu erfassen und richtig zu interpretieren. Talente: Menschenkenntnis, Sinnenschärfe, Wachsamkeit.
\begin{description}
\item Menschenkenntnis ist deine Fähigkeit, die Absichten deines Gegenübers zu durchschauen. Damit enttarnst du Lügen und falsche Annäherungsversuche.
\item Sinnenschärfe ist die aktive Verwendung deiner Sinne, um einen Kollaborateur in einem geschäftigen Wirtshaus zu belauschen, die Flagge eines nahenden Schiffes zu erkennen oder die Nadel im Heuhaufen zu finden.\item Wachsamkeit fasst den passiven Einsatz deiner Sinne zusammen. Mit ihr entdeckst du Hinterhalte, bemerkst eine Unstimmigkeit an einem Tatort oder eine verborgene Notiz am Wegesrand.
\end{description}}}

\newglossaryentry{autorität}
{
    name={Autorität (CH/KL/MU, 3)},
    description={Autorität erlaubt es deinem Charakter, sich Respekt zu verschaffen und seinen Willen durchzusetzen. Talente: Anführen, Einschüchtern, Rhetorik.
\begin{description}
\item Mit Anführen leitest und motivierst du Untergebene. Du kannst mit Anführen Löscharbeiten oder einen Trupp von Kämpfern koordinieren, um ihnen so Vorteile zu verschaffen (S. 46). Gerade in großen Schlachten entscheidet der Heerführer über Sieg oder Niederlage.
\item Einschüchtern jagt dem Gegenüber Angst ein und bringt ihn so zu einer gewünschten Handlung. Die Probe wird oft vergleichend gegen den MU oder die Menschenkenntnis des Gegenübers abgelegt.
\item Rhetorik beinhaltet zahlreiche Fähigkeiten und Kniffe, um die eigenen Argumente wirkungsvoll einzusetzen und die des Gegenübers zu entkräften. Du kannst Rhetorik nur einsetzen, wenn dein Charakter von seinem Standpunkt überzeugt ist – dreiste Lügen fallen unter Überreden.
\end{description}}}

\newglossaryentry{beeinflussung}
{
    name={Beeinflussung (CH/CH/IN, 3)},
    description={Hierunter fallen sämtliche Techniken, um das Gegenüber zu einer gewünschten Handlung zu bewegen. Dabei werden Betören und Überreden oft als vergleichende Proben gegen die Menschenkenntnis des Opfers gewürfelt. Talente: Betören, Überreden.
\begin{description}
\item Beim Betören nutzt du deine persönliche Ausstrahlung, um zu bekommen, was du willst. Das reicht von Kleinigkeiten wie einem Freibier bis hin zum Verrat geheimer Staatsinformationen.
\item Überreden bedeutet den geschickten Einsatz von Übertreibungen, Unwahrheiten oder Lügen, um das Gegenüber zumindest kurzfristig zu beeinflussen. Mit Überreden feilschst du am Marktplatz, überlistest eine Stadtwache oder infiltrierst ein Borbaradianerkloster ein.
\end{description}}}

\newglossaryentry{gebräuche}
{
    name={Gebräuche (CH/IN/KL, 2)},
    description={Gebräuche stellt das Wissen um fremde und vertraute Kulturen sowie die dort herrschenden gesellschaftlichen Normen dar. Wie heißt der lokale Markgraf und wie sprichst du ihn an? In welchem Viertel findest du eine vertrauenswürdige Herberge, verschwiegene Schläger oder die neuesten Gerüchte? Wie funktioniert das Kamelspiel und auf welches Getränk könntest du deinen Gegner einladen? Herrscht in der nächsten Stadt ein Waffenverbot und wie streng wird es ausgelegt? Gebräuche beantwortet all diese Fragen und verhindert, dass du unangenehm auffällst. Talente: nach Kultur; Mittelreich, Horasreich, Tulamidenlande, Südaventurien, Bornland (verbilligt), Thorwal (verbilligt), Maraskan (verbilligt), Elfen (verbilligt), Zwerge (verbilligt) und weitere. Das Talent der eigenen Heimat ist kostenlos.}}

\newglossaryentry{derekunde}
{
    name={Derekunde (FF/IN/KL, 2},
    description={Darunter fällt das theoretische Wissen über Natur, Tiere und Pflanzen sowie die Geographie. Talente: Geographie, Pflanzenkunde, Tierkunde.
\begin{description}
\item Geographie befasst sich nicht nur mit Reiserouten und fernen Ländern, sondern auch mit dem Sternenhimmel, der Kartographie und Navigation. Ein Geograph kann Schatzkarten lesen und anfertigen, die Position eines Schiffes bestimmen und Sternenkonstellationen deuten.
\item Pflanzenkundige erforschen die vielen nützlichen, gefährlichen oder wundersamen Gewächse Aventuriens. Sie sammeln die Kräuter für eine Heilsalbe, kennen die Gefahr von Jagdgras und finden essbare Früchte oder Wurzeln.
\item Tierkunde bedeutet die Kenntnis der aventurischen Fauna. Dein Charakter kann einen Schleimfleck als Spur einer Riesenamöbe erkennen, einen gereizten Bären beruhigen und kennt die Schwachstelle eines Tatzelwurms.
\end{description}}}

\newglossaryentry{mythenkunde}
{
    name={Mythenkunde (IN/KL/KL, 2)},
    description={Mythenkunde ist das Wissen über die Götter und ihre Diener, sowohl die eigene Religion, als auch – weniger genau – fremde Religionen betreffend. Zusätzlich gehört dazu die Kenntnis der Geschichte sowie bekannter Sagen und Legenden, was meist schwer voneinander zu trennen ist. Talente: Geschichten und Legenden, Götter und Kulte.
\begin{description}
\item Geschichten und Legenden ist das Wissen um alte Überlieferungen. Damit kannst du das Alter von Grabmälern und Artefakten bestimmen oder aus einer Sage die Vorlieben und Schwachpunkte eines Riesen ableiten.
\item Götter und Kulte befasst sich mit dem Wesen der Götter, ihrer Schöpfung und ihren sterblichen Dienern. Dein Charakter kennt den Aufbau und die Ziele der Kirchen und weiß, wie man ihren Angehörigen gegenübertritt. Genauso kann er eine alte Kultstätte einer Gottheit oder vielleicht einem Erzdämonen zuordnen.
\end{description}}}

\newglossaryentry{magiekunde}
{
    name={Magiekunde (KL/KL/MU, 2)},
    description={Die astrale Kraft durchzieht die ganze Welt und wird seit Jahrtausenden von Zauberern genutzt. Magiekunde ist das theoretische Wissen über diese Kraft und ihre Verwendung. Talente: Dämonenkunde, Elementarkunde, Magietheorie, Zauberpraxis.
\begin{description}
\item Dämonenkunde beschäftigt sich mit den Erzdämonen, ihren Dienern und dämonischen Zaubern. Du kannst Dämonen benennen und kennst ihre Schwachstellen.
\item Elementarkunde befasst sich mit den sechs Elementen, der Elementarbeschwörung und den elementaren Zaubern.
\item Magietheorie hilft dir bei der Einschätzung und der Analyse von magischen Phänomenen.
\item Zauberpraxis umfasst die Kenntnis verbreiteter Zauber und Rituale, ihrer Wirkung und möglicher Gegenmaßnahmen.
\end{description}}}

\newglossaryentry{überleben}
{
    name={Überleben (GE/FF/KO, 3)},
    description={Hierunter fallen alle Fähigkeiten, die zum Überleben in der Wildnis erforderlich sind. Du kannst dich orientieren oder einen Lagerplatz, Nahrung und Wasser finden. Außerdem bist du mit typischen Gefahren der Wildnis vertraut und kannst tierische oder menschliche Fährten deuten und ihnen folgen. Dieses Wissen ist rein praktisch; exotische Tiere, Monster oder seltene Heilpflanzen fallen nicht in diesen Bereich. Talente: nach Region; Hoher Norden (nördlich der Salamandersteine), Nordaventurien (nördlich von Steineichenwald und Ysilsee), Mittelaventurien (nördlich von Eisenwald, Phecanowald und Raschtulswall), Südaventurien (nördlich von Drôl und Thalusien), Tiefer Süden (südlich davon) sowie Maraskan (verbilligt), Wüste (verbilligt), Gebirge (verbilligt) und Meer (verbilligt). Die Gebiete schließen sich gegenseitig aus, Maraskan gehört zum Beispiel nicht zum Süden.}}

\newglossaryentry{verschlagenheit}
{
    name={Verschlagenheit (FF/IN/MU, 2)},
    description={Verschlagenheit ist das Handwerk von zwielichtigen Gestalten, die Diebstahl, Falschspiel und Einbruch nicht scheuen. Talente: Falschspielen (verbilligt), Schlösser knacken, Stehlen.
\begin{description}
\item Falschspieler helfen dem Glück etwas nach, zum Beispiel beim Hütchenspiel oder beim Boltan. Proben auf Falschspielen werden oft vergleichend gegen die Wachsamkeit der Mitspieler abgelegt.
\item Schlösser knacken öffnet mit Hilfe eines Dietrichs oder einer Haarnadel Schatzkisten und Tresorräume. Außerdem kann dein Charakter Fallen entschärfen.
\item Mit Stehlen kannst du zur richtigen Zeit am richtigen Ort sein, um die richtige Person unauffällig um ihren Besitz zu erleichtern. Außerdem können Diebe den Wert ihrer Beute einschätzen und Kontakt zu einem Hehler aufnehmen.
\end{description}}}

\newglossaryentry{Alchemie (FF/FF/KL, 3)}
{
    name={Alchemie (FF/FF/KL, 3)},
    description={Alchemie ist die Wissenschaft von der Umwandlung der Stoffe. Mit ihr kannst du nützliche Tränke herstellen oder unbekannte Gebräue analysieren; für beides ist aber zumindest ein einfacher Alchemiekoffer notwendig. Talente: Analyse (verbilligt), Magische Elixiere, Profane Alchemika.
\begin{description}
\item Mit einer alchemistischen Analyse kannst du feststellen, ob eine verstaubte Phiole einen Heiltrank oder ein tödliches Gift enthält.
\item Magische Elixiere beinhaltet Heiltränke, Stärkungsmittel und Verwandlungselixiere, kurz die hohe Kunst der Alchemie.
\item Profane Alchemika ist das weit bodenständigere Handwerk der Meuchler, Waldläufer und Kräuterfrauen, mit dem du Gifte, Heilsalben oder Rauchbomben herstellen kannst.
\end{description}}}

\newglossaryentry{heilkunde}
{
    name={Heilkunde (CH/FF/KL, 3)},
    description={Ein Heilkundiger bekämpft tödliche Blutungen, heimtückische Gifte sowie langwierige Krankheiten und hilft seinen Gefährten so, wieder auf die Beine zu kommen. Talente: Gifte und Krankheiten, Wundheilung
\begin{description}
\item Mit dem Talent Gifte und Krankheiten stoppst du eine Tulmadron-Vergiftung, erkennst die ersten Anzeichen für Zorganpocken und kannst einen Ghulbiss behandeln.
\item Wundheilung ermöglicht es dir, im Kampf entstandene Blutungen zu stoppen und die Heilung von Verletzungen zu fördern. Zusätzlich kannst du durch die Untersuchung von Wunden Rückschlüsse auf die Tatwaffe und den Tathergang ziehen.
\end{description}}}

\newglossaryentry{handwerk}
{
    name={Handwerk (FF/FF/KK)},
    description={Mit Handwerk schmiedest du Waffen und Rüstungen, fertigst filigrane Mechaniken und stabile Truhen. Talente: Holzbearbeitung (verbilligt), Mechanik (verbilligt), Schmieden.
\begin{description}
\item Mit Holzbearbeitung fertigst du Schilde, Bögen oder ein improvisiertes Floß, reparierst eine Zugbrücke, schnitzt aus einem Knochen eine Pfeilspitze oder flickst ein Leck in einem Schiff.
\item Mit Mechanik stellst du Apparaturen wie Armbrüste, Flaschenzüge oder Südweiser her und wartest sie.
\item Schmieden ist die Herstellung und Reparatur von Schwertern, Äxten oder Plattenpanzern.
\end{description}}}
%gesundheit

\newglossaryentry{erschöpfung}
{
    name={Erschöpfung},
    description={Die Erschöpfung ist eine \gls{einschränkung}, die schneller heilt als \glslink{wunde}{Wunden} (siehe \gls{regeneration}).}}

\newglossaryentry{wunde}
{
    name={Wunde},
    description={Die Wunde ist eine \gls{einschränkung}, die langsamer heilt als \gls{erschöpfung} (siehe \gls{regeneration}).}}

\newglossaryentry{einschränkung}
{
    name={Einschränkung},
    description={\gls{erschöpfung} und \glslink{wunde}{Wunden} sind jeweils Einschränkungen. Spricht man von Einschränkungen im Plural ist die Summe aus Stufen \gls{erschöpfung} und \glslink{wunde}{Wunden} gemeint.}}

\newglossaryentry{regeneration}
{
    name={Regeneration},
    description={Damit die heilende Luft Aventuriens wirkt, müssen zwei Vorauss­etzungen gegeben sein: Erstens muss der Charakter ruhen oder besser noch schlafen, zweitens dürfen keine schädlichen Effekte wie Gifte, Krankheiten oder Durst auf ihn wirken. Unter diesen Umständen regeneriert ein Charakter eine \gls{wunde} pro durchgeschlafener Nacht von mindestens 6 Stunden oder einen Punkt \gls{erschöpfung} pro Stunde. Wenn sich dadurch Lücken in der Statusleiste ergeben, wandern die \glslink{einschränkung}{Einschränkungen} dahinter nach links.}}

\newglossaryentry{kampfunfähigkeit}
{
    name={Kampfunfähigkeit},
    description={Mit fünf \glslink{einschränkung}{Einschränkungen} wird dein Charakter potentiell kampfunfähig. Potentiell stürzt er, kann nicht mehr aufstehen und keine \glslink{aktion}{Aktionen} oder \glslink{reaktion}{Reaktionen} ausführen. Der Spielleiter entscheidet, ob der Charakter bei Bewusstsein ist. Du kannst eine \glslink{zähigkeit}{Zähigkeits-Probe} (12, \gls{I}) ablegen, um weiter kampffähig zu bleiben. Der Wundabzug gilt natürlich auch bei dieser Probe. Misslingt sie, erleidet dein Charakter durch diese Anstrengung noch einen zusätzlichen Punkt \gls{erschöpfung} und wird dennoch kampfunfähig. Die Kampfunfähigkeit endet nach einer Stunde oder sobald eine \gls{einschränkung} geheilt wurde – allerdings nur, wenn keine schädlichen Effekte wie Gifte, Krankheiten oder Durst auf deinen Charakter wirken.}}

\newglossaryentry{wundschmerz}
{
    name={Wundschmerz},
    description={Wenn dein Charakter auf einen Schlag zwei Einschränkungen erleidet, muss er sofort eine KO-Probe (20, I) ablegen. Für jede zusätzliche Wunde steigt die Schwierigkeit um weitere +4. Misslingt die Probe, wird er vom Schmerz überwältigt und ist betäubt. Er kann erst in seiner übernächsten Initiativephase wieder Aktionen und Reaktionen ausführen.}}

\newglossaryentry{wundabzüge}
{
    name={Wundabzüge},
    description={Auf deinem Charakterbogen findest du eine Statusleiste mit 8 Kästchen, in denen du erlittene Einschränkungen notieren kannst. Wunden notierst du dabei mit einem X , Erschöpfung mit einem /. Sobald dein Charakter insgesamt mehr als zwei Einschränkungen erlitten hat, erleidet er für jede weitere Einschränkung einen Abzug von –2 auf alle Proben. Den momentanen Wundabzug kannst du natürlich jederzeit auf der Statusleiste ablesen.}}

\newglossaryentry{blutungen und tod}
{
    name={Blutungen und Tod},
    description={Blutungen drohen dir, wenn du mehr als 4 Wunden durch Waffengewalt oder ähnliche akute Verletzungen erlitten hast. Solltest du dann kampfunfähig werden, oder bereits kampfunfähig sein und eine weitere Wunde durch Waffengewalt erleiden, muss dir eine KO-Probe (12, I) gelingen. Misslingt die Probe, beginnt dein Charakter zu bluten: Du musst nach jeweils WS Initiativephasen eine KO-Probe (12, I) ablegen, bei deren Misslingen du eine weitere Wunde erleidest. Eine Blutung endet bei erfolgreicher erster Hilfe (S. 33). Bei acht Einschränkungen ist auch für den zähesten Charakter die Grenze erreicht. Eine weitere Einschränkung bedeutet unweigerlich den Tod.}}

\newglossaryentry{gesundheit von gegenständen}
{
    name={Gesundheit von Gegenständen},
    description={Für Gegenstände wird in Ilaris dasselbe Gesundheitssystem wie für Lebewesen verwendet, nur einige Begriffe unterscheiden sich. Statt einer \gls{ws} verfügt ein Gegenstand über eine Härte. Ein Seil hat eine Härte von 2, ein stabile Holztür eine Härte von 8 und Stahlgitterstangen eine Härte von 24. Übersteigt der Schaden die Härte, erleidet der Gegenstand eine Beschädigung. Jede Beschädigung nach der zweiten kann die Verwendung des Gegenstands um –2 erschweren. Nach 5 Beschädigungen ist das Objekt unbrauchbar, kann aber noch repariert werden. Erst ein Gegenstand mit 9 Beschädigungen ist unwiderruflich zerstört.}}

\newglossaryentry{heilung}
{
    name={Heilung},
    description={Ein erfahrener Heilkundiger kann lebensbedrohliche Situationen abwenden. Mit erster Hilfe kann eine Blutung gestoppt werden oder die Auswirkung von Giften und Krankheiten gemildert werden. Dafür benötigst du so viele Aktionen Konzentration, wie die Patientin am Beginn der Heilung Wunden erlitten hat, mindestens aber 4 Initiativephasen. Außerdem musst du über Verbandsmaterial (Wunden) oder passende Heilkräuter (Gifte und Krankheiten) verfügen.\\
Darauf folgen heilungsfördernde Maßnahmen, wodurch der Patient sofort eine Wunde regeneriert. Pro Patient und Tag darf nur eine Probe angelegt werden. Eine solche Probe dauert eine halbe Stunde. Außerdem benötigst du heilende Salben oder ähnliche Hilfsmittel, Verbands­­material und sauberes Wasser. Die Schwierigkeit der Probe beträgt 16 + den Betrag der aktuellen Wundabzüge des Patienten.}}

\newglossaryentry{übernatürliche heilung}
{
    name={Übernatürliche Heilung},
    description={Heiltränke, Balsam und Wundsegen – übernatürliche Heilung kennt viele Wege. Trotzdem wirkt sie immer auf eine von zwei Arten: Entweder der heilende Effekt stärkt und beschleunigt die natürliche Regeneration, wie etwa der Zauber Ruhe Körper. Solche Effekte wirken meist langsam, heilen aber dafür Wunden auf direktem Weg. Oder der Effekt greift von außen in den Körper ein, wie der Balsam oder ein Heiltrank. Ein solcher Effekt bringt eine gewisse Anzahl an \gls{hp}. Für jede Überschreitung der WS verschwindet eine Wunde. Heilpunkte funktionieren also genau umgekehrt wie Schadenspunkte; dadurch ist es schwieriger, ein großes und kräftiges Wesen zu heilen. Allerdings wirken solche Effekte meist wesentlich schneller.}}

\newglossaryentry{schadensquellen}
{
    name={Schadensquellen},
    description={Stürze, Entbehrungen und Gifte sind nur einige Beispiele für die Gefahren, die deinen Charakter erwarten. Viele dieser Schadensquellen wirken nicht sofort, sondern erst nach einer gewissen Verzögerung. Ein Beispiel dafür wäre die Inkubationszeit einer Krankheit. Erst dann beginnt die Wirkungsdauer, in der du sofort und dann in bestimmten Intervallen Schaden erleidest. Verzögerung, Intervall und Schaden sind bei den jeweiligen Schadensquellen angegeben.}}

\newglossaryentry{körperliche anstrengung}
{
    name={Körperliche Anstrengung},
    description={Wie lange dein Charakter eine anstrengende Tätigkeit durchhält, hängt von deiner Konstitution und deiner Behinderung – kurz deinem Durchhaltevermögen (DH) ab. Das Durchhaltevermögen ergibt sich aus KO – 2 x BE, aber beträgt wenigstens 1. Es ist auf deinem Charakterbogen vermerkt.}}

\newglossaryentry{hunger und durst}
{
    name={Hunger und Durst},
    description={Hunger und Durst Nahrungs- oder Wassermangel können innerhalb einiger Tage zum Tod führen. Diese Zeitspannen gelten, wenn dein Charakter überhaupt kein Wasser oder keine Nahrung zur Verfügung hat. Muss er mit der halben benötigten Menge auskommen, verdreifachen sich Verzögerung und Intervall. Die Verzögerung beginnt von vorne, wenn dein Charakter mindestens eine Woche ausreichend Nahrung bzw. Wasser zu sich nimmt.}}

\newglossaryentry{stürze}
{
    name={Stürze},
    description={Stürze richten 1W6 SP pro Schritt Höhe an; bei besonders harten oder weichen Böden 1W6+1 bzw. 1W6–1 SP. Du kannst eine \gls{akrobatik}-Probe (12+Schritt Höhe) ablegen, um die effektive Höhe zu halbieren. Bei einem freiwilligen Sprung senkt eine gelungene Akrobatik-Probe die effektive Höhe sogar auf ein Viertel des Ursprungswertes.}}

\newglossaryentry{hitze und kälte}
{
    name={Hitze und Kälte},
    description={Hitze und Kälte sind unbarmherzige Feinde, die selbst zähe Charaktere in die Knie zwingen können. Die Verzögerung entspricht dabei immer dem Intervall. Schutz gegen Hitze bietet nur eine Rast im Schatten, kühler Wind, oder, nach Spielleiterentscheid, ein mindestens doppelter Wasserverbrauch. All diese Maßnahmen senken die Temperaturstufe um je –1. Warme, trockene Kleidung schützt gegen Kälte und erhöht die Temperaturstufe um +1 (dicke Winterkleidung) bis +2 (fellgefütterte Kleidung). Die Behinderung solcher Kleidung ist genau so hoch wie ihr Kälteschutz. Außerdem erhöht ein Lagerfeuer die Temperatur um +1 Stufe, während sie durch Wind um –1 oder sogar –2 Stufen gesenkt wird.}}

\newglossaryentry{gifte}
{
    name={Gifte},
    description={Wir unterscheiden Gifte nach der Art der Verabreichung in Einnahme- (E), Kontakt- (K) und Waffengifte (W). Letztere wirken, sobald die TP der vergifteten Waffe den Rüstungsschutz übersteigen und sind nur für 1W6 Treffer einsetzbar. Die Stufe eines Giftes gibt seine Stärke an. Beachte, dass hier die Mindeststufe eines Giftes angeführt ist und ein kompetenter Alchemist stärkere Gifte anfertigen kann (die Giftstufe steigt pro Stufe Hohe Qualität um +4).\\
Der Vorteil Resistenz gegen Gifte, eine gelungene erste Hilfe oder eine \gls{ko}-Probe (Giftstufe, I) können die Wirkung eines Giftes mildern: Das Intervall eines Giftes verdoppelt sich bei gleicher Wirkungsdauer, Zusatzwirkungen wie Erschwernisse durch eine Lähmung werden halbiert. Profitierst du sogar von zwei mildernden Effekten (wie eine gelungene KO-Probe und eine gelungene erste Hilfe), wird das Gift aufgehoben. Die mehrfache Vergiftung mit demselben Gift erhöht die Wirkung nur dann, wenn dies ausdrücklich angegeben ist. Normalerweise gilt schlicht die längere Wirkungsdauer sowie die höhere Giftstufe.}}

\newglossaryentry{krankheiten}
{
    name={Krankheiten},
    description={Auch wenn Krankheiten und Seuchen in Aventurien häufig vorkommen, ist ein kranker und damit handlungsunfähiger Charakter eine sehr frustrierende Erfahrung. Wir empfehlen deswegen, Krankheiten nur zu verwenden, wenn sie zentral für ein Abenteuer sind. Krankheiten verwenden dieselben Regeln wie \gls{gifte}, allerdings muss sich der Kranke meist schonen, sonst kann der Spielleiter die Auswirkungen der Krankheit deutlich erhöhen.}}
    
        
        
        
        %CH
\newglossaryentry{eindrucksvoll I}
{
    name={Eindrucksvoll I},
    description={In Rededuellen sind Proben auf Betören und Einschüchtern um 2 erleichtert.}}

\newglossaryentry{eindrucksvoll II}
{
    name={Eindrucksvoll II},
    description={In Rededuellen sind Proben auf Betören und Einschüchtern um +2 erleichtert. Bei Proben auf Betören, Einschüchtern und Gebräuche gelten Patzer als gewöhnlich misslungen, außer sie entstehen durch eine ungewohnte Umgebung.}}

\newglossaryentry{soziale anpassungsfähigkeit}
{
    name={Soziale Anpassungsfähigkeit},
    description={Du bist immun gegen alle Auswirkungen durch ungewohnte Umgebung.}}

\newglossaryentry{starke aura}
{
    name={Starke Aura},
    description={Während eines Rededuells kannst du eine CH-Probe gegen die Willenskraft deines Gegenübers ablegen. Wenn die Probe gelingt, kannst du eine Probe in diesem Rededuell wiederholen.}}

%FF

\newglossaryentry{routiniert I}
{
    name={Routiniert I},
    description={Die Dauer zur Fertigung handwerklicher Produkte, zur Wundversorgung und zum Schlösserknacken verkürzt sich um ein Viertel des unmod. Werts.}}

\newglossaryentry{routiniert II}
{
    name={Routiniert II},
    description={Die Dauer zur Fertigung handwerklicher Produkte, zur Wundversorgung und zum Schlösser knacken verkürzt sich um ein weiteres Viertel des unmodifizierten Werts. Bei Proben auf Alchemie, Handwerk, Heilkunde, und Verschlagenheit gelten Patzer als gewöhnlich misslungen.}}

\newglossaryentry{improvisation}
{
    name={Improvisation},
    description={Unzureichendes Werkzeug oder minderwertige
Verbrauchsmaterialien gelten um eine Stufe höher.}}

\newglossaryentry{meisterwerk}
{
    name={Meisterwerk},
    description={Mit hervorragenden Materialien und dem doppelten Zeitaufwand kannst du ein Meisterwerk schaffen. Dieses erhält zusätzlich 2x die Modifikation hohe Qualität und verbesserte Eigenschaften nach Meisterentscheid. Solche Werke sind extrem selten und Gegenstand von Sagen und Legenden. Der Vorteil kann nur mit Fertigkeiten mit einem unmod. Probenwert von mindestens 16 angewandt werden.}}

%GE
\newglossaryentry{flink I}
{
    name={Flink I},
    description={\gls{gs}+1. Bei \gls{athletik}{Athletikproben} gelten Patzer als gewöhnlich misslungen.}}

\newglossaryentry{flink II}
{
    name={Flink II},
    description={\gls{gs}+1.}}

\newglossaryentry{katzenhaft}
{
    name={Katzenhaft},
    description={Erschwernisse durch unsicheren Untergrund sinken um eine Stufe und gegen den Schaden aus Stürzen, Zusammenstößen usw. kannst du deine \gls{ge} als \gls{ws} verwenden.}}

\newglossaryentry{körperbeherrschung}
{
    name={Körperbeherrschung},
    description={Du kannst Nah- oder Fernkampfangriffen, elementaren Schadenszaubern oder ähnlichen Schadensquellen mit einer \gls{ge}-Gegenprobe entgehen. Der Einsatz von Körperbeherrschung wird angesagt, nachdem übliche Verteidigungsmöglichkeiten (wie eine Verteidigung) versagt haben, aber bevor der Schaden bestimmt wird. Anschließend erleidest du einen Punkt \gls{erschöpfung}.}}

%IN
\newglossaryentry{vorausschauend I}
{
    name={vorausschauend I},
    description={In Rededuellen sind Proben auf Rhetorik und Überreden um +2 erleichtert.}}

\newglossaryentry{vorausschauend II}
{
    name={Vorausschauend II},
    description={In Rededuellen sind Proben auf Rhetorik und Überreden um +2 erleichtert. Bei Proben auf Überreden, Rhetorik und Menschenkenntnis gelten Patzer als gewöhnlich misslungen, außer sie entstehen durch eine ungewohnte Umgebung.}}

\newglossaryentry{bedächtig}
{
    name={Bedächtig},
    description={Wenn du in einem Rededuell abwartest, ist deine Probe um +4 erleichtert.}}

\newglossaryentry{empathie}
{
    name={Empathie},
    description={Du kannst eine \gls{in}-Probe gegen die \gls{willenskraft} deines Gegenübers ablegen. Wenn die Probe gelingt, erfährst du eine seiner Schwächen. Eine einmal abgelegte Probe, egal ob ge- oder misslungen, kannst du nur wiederholen, wenn du das Gegenüber besser kennengelernt hast.}}

%KK
\newglossaryentry{zerstörerisch I}
{
    name={Zerstörerisch I},
    description={Proben zum Zerstören oder Durchbrechen von Gegenständen sind um +4 erleichtert.}}

\newglossaryentry{zerstörerisch II}
{
    name={Zerstörerisch II},
    description={Proben zum Zerstören oder Durchbrechen von Gegenständen sind um weitere +4 erleichtert. Erlaubt das Manöver Hammerschlag gegen
Gegenstände. Bei waffenlosen Angriffen gegen Gegenstände verletzt du dich normalerweise nicht.}}

\newglossaryentry{muskelprotz}
{
    name={Muskelprotz},
    description={Einschüchtern-Proben im Kampf sind um +4 und du kannst ohne zusätzliche Erschwernis bis zu 4 Gegner einschüchtern.}}

\newglossaryentry{adrenalinschub}
{
    name={Adrenalinschub},
    description={Du kannst dir eine Probe bei einer körperlichen Tätigkeit (z.B. Schmieden, Laufen, Nahkampfangriff) um +4 erleichtern. Anschließend erleidest du einen Punkt Erschöpfung.}}
        
%KL
\newglossaryentry{scharfsinnig I}
{
    name={Scharfsinnig II},
    description={Proben bei einer Ermittlung oder Recherche sind um +2 erleichtert.}}

\newglossaryentry{scharfsinnig II}
{
    name={Scharfsinnig II},
    description={Proben bei einer Ermittlung oder Recherche sind um weitere +2 erleichtert. Bei Recherchen erhältst du einen zusätzlichen Informationsgrad, wenn der gewertete Würfel eine 12 oder höher zeigt.}}

\newglossaryentry{vorbereitung}
{
    name={Vorbereitung},
    description={Du kannst Proben auf profane Fertigkeiten besonders sorgfältig vorbereiten (sofern Vorbereitung sinnvoll ist). Wenn du die doppelte notwendige Zeit aufwendest, erhältst du +4 Punkte Erleichterung.}}

\newglossaryentry{eingebung}
{
    name={Eingebung},
    description={Pro Abenteuer kannst du den Spielleiter W3 mal um einen Tipp bitten. Diese Tipps sollen nicht das Abenteuer lösen, können dich aber auf Fehler in deinem Plan oder bisher übersehene Aspekte oder Zusammenhänge aufmerksam machen.}}

\newglossaryentry{abgehärtet I}
{
    name={Abgehärtet I},
    description={Proben zur Abwehr von Giften und Krankheiten sind um +4 erleichtert.}}

\newglossaryentry{abgehärtet II}
{
    name={Abgehärtet II},
    description={Proben zur Abwehr von Giften und Krankheiten sind um weitere +4 erleichtert und das Durchhaltevermögen steigt um 2.}}

\newglossaryentry{schnelle heilung}
{
    name={Schelle Heilung},
    description={Du regenerierst 2 Wunden pro durchschlafener Nacht und immer noch 1 Wunde, wenn die Schlafphase unterbrochen wurde.}}
        
\newglossaryentry{unverwüstlich}
{
    name={Unverwüstlich},
    description={\gls{ws} +1. Wenn \glslink{zähigkeit}{Zähigkeitsproben} zum Ignorieren von \gls{kampfunfähigkeit} misslingen, erleidest du keine \gls{erschöpfung}.}}

%MU
\newglossaryentry{willensstark I}
{
    name={Willensstark I},
    description={\gls{mr}+4.}}

\newglossaryentry{willensstark II}
{
    name={Willensstark II},
    description={\gls{mr}+4.}}

\newglossaryentry{geisterpanzer}
{
    name={Geisterpanzer},
    description={Gegen direkten Schaden aus Zaubern wie Fulminictus, Ignisphaero oder Hexengalle kannst du deinen \gls{mu} als \gls{ws} verwenden.}}

\newglossaryentry{unbeugsamkeit}
{
    name={Unbeugsamkeit},
    description={\gls{mr} +\gls{mu}/2. Mit einer \gls{aktion} \gls{konflikt} und einer \gls{konterprobe} (MU,16) kannst du einen auf dir liegenden \gls{zauber} abschütteln. Anschließend erleidest du einen Punkt \gls{erschöpfung}.}}\newglossaryentry{rüstung}
{
    name={Rüstung},
description={Der Rüstungsschutz erhöht die \gls{ws*} entscheidend, wodurch dein Charakter seltener Wunden und Wundschmerz erleidet. Dafür schränkt Rüstung die Beweglichkeit ein, was durch die \gls{be} dargestellt wird. Normalerweise ist die Behinderung gleich hoch wie der Rüstungsschutz. Jeder Punkt Behinderung erschwert Kampfwürfe um –1, senkt die \gls{gs} um –1 (bis minimal 1) und das Durchhaltevermögen um –2 (bis minimal 1, S. 34). Proben auf die Fertigkeit \gls{athletik} und \gls{ge} sind nach Spielleiterentscheid um –2 (Reiten in schwierigem Gelände) bis –16 (Schwimmen in schwerer Rüstung) erschwert. Die Behinderung kann durch die Vorteile \gls{rüstungsgewöhnung} und besonders hochwertige Herstellung gesenkt werden. Eine unpassende Rüstung erhöht die Behinderung um +1.}}

\newglossaryentry{trefferzonen}
{
    name={Optional: Trefferzonen},
    description={Diese Optionalregeln eignen sich für deine Spielrunde, wenn ihr Rüstungen und Wundschmerz detaillierter darstellen möchtet. Mit ihnen hat dein Charakter 6 Trefferzonen: \gls{kopf}, \gls{brust}, \gls{bauch}, \gls{schwertarm}, \gls{schildarm} und \gls{beine}. Die \gls{rüstung} wird weiter individualisiert, indem du 6 x RS Punkte auf die 6 Zonen verteilst. Dieser Zonen-Rüstungsschutz (\gls{zrs}) ergeben mit der \gls{ws} die 6 Zonen-Wundschwellen (\gls{zws}*). Selbstverständlich sollte diese Aufteilung dem gesunden Menschenverstand folgen, also ein Helm nicht den Bauch schützen.\\
Mit dem Manöver \gls{gezielter schlag} kannst du eine Trefferzone aussuchen, sonst wird die Trefferzone mit 1W6 zufällig bestimmt. Die TP des Angriffes werden dann mit der jeweiligen ZWS* verglichen. Die WS* verwendest du nur noch für Flächenschaden, zum Beispiel durch den Feuerodem eines Drachen. Die Regeln für Wundabzüge ändern sich mit den Trefferzonen nicht. Die Auswirkungen des Wundschmerzes unterscheiden sich aber je nach Trefferzone: Ein heftiger Kopftreffer hat wie in den Basisregeln eine betäubende Wirkung, während dich ein Treffer an den Beinen auf die Bretter schickt. Die Probe wird je nach Zone auf ein unterschiedliches Attribut abgelegt, die Schwierigkeit ist gleich der der Basisregeln (20, I).}}

\newglossaryentry{kopf}
{
    name={Trefferzone: Kopf},
    description={Bei einer 6: \gls{mu}(20, I). Betäubt: entspricht dem \gls{wundschmerz} in Basisregeln.}}

\newglossaryentry{brust}
{
    name={Trefferzone: Brust},
    description={Bei einer 5: \gls{ko}(20, I). Organtreffer: Das Opfer erleidet eine Zusatzwunde.}}

\newglossaryentry{bauch}
{
    name={Trefferzone: Bauch},
    description={Bei einer 4: \gls{ko}(20, I). Organtreffer: Das Opfer erleidet eine Zusatzwunde.}}

\newglossaryentry{schwertarm}
{
    name={Trefferzone: Schwertarm},
    description={Bei einer 3: \gls{kk}(20, I). Entwaffnet: Die mit dieser Hand geführte Waffe fällt zu Boden.}}

\newglossaryentry{schildarm}
{
    name={Trefferzone: Schildarm},
    description={Bei einer 2: \gls{kk}(20, I). Entwaffnet: Die mit dieser Hand geführte Waffe fällt zu Boden.}}

\newglossaryentry{beine}
{
    name={Trefferzone: Beine},
    description={Bei einer 1: \gls{ge}(20, I). Sturz: Das Opfer stürzt und liegt am Boden.}}